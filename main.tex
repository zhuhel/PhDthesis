% !TeX encoding = UTF-8
% !TeX program = xelatex
% !TeX spellcheck = en_US

\documentclass[doctor,english,pdf]{ustcthesis}
% doctor|master|bachelor [academic|professional] [chinese|english] [print|pdf]
% [super|numebers|authoryear]

\title{利用ATLAS探测器上ZZ 玻色子到全轻子通道的衰变事例进行电弱对称性破缺的研究}
\author{祝鹤龄}
\major{粒子与原子核物理}
\supervisor{赵政国}
\cosupervisor{马宏}
% \date{二〇一七年五月一日} % 注释掉则为今日
% \professionaltype{专业学位类型}
% \secretlevel{秘密}        % 绝密|机密|秘密,注释本行则不保密
% \secretyear{20}           % 保密年限

\entitle{Study of Electroweak Symmetry Breaking in ZZ Production in Purely Leptonic Decay with ATLAS Detector}
\enauthor{Heling Zhu}
\enmajor{Particle and Nuclear Physics}
\ensupervisor{Zhengguo Zhao}
\encosupervisor{Hong Ma}
% \endate{May 1, 2017}      % Today if commented
% \enprofessionaltype{Professional degree type}
% \ensecretlevel{Secret}    % Top secret|Highly secret|Secret


% 加载宏包和配置
\usepackage{graphicx}
\graphicspath{{figures/}}
\usepackage{booktabs}
\usepackage{longtable}
\usepackage[ruled,linesnumbered]{algorithm2e}
\usepackage{siunitx}
\usepackage{amsthm}
\usepackage{hyperref}
\usepackage{setspace}

%% sepecial defination for ATLAS
\usepackage{main-defs}
\usepackage{latex/atlasunit}
\usepackage{latex/atlasparticle}
\usepackage{latex/atlasjournal}
\usepackage{latex/atlasmisc}
\usepackage{latex/atlasphysics}

\DeclareRobustCommand\cs[1]{\texttt{\char`\\#1}}
\newcommand\pkg{\textsf}

\renewcommand\vec{\symbf}
\newcommand\mat{\symbf}
\newcommand\ts{\symbfsf}
\newcommand\real{\mathbf{R}}
%\newcommand{\doi}[1]{\textsc{doi}: \href{http://dx.doi.org/#1}{\nolinkurl{#1}}}
\newcommand{\doi}[1]{\href{http://dx.doi.org/#1}{\nolinkurl{#1}}}


\begin{document}

\maketitle
\makestatement

\frontmatter
% !TeX root = ../main.tex

\begin{abstract}
  中文摘要
\end{abstract}

\begin{enabstract}
  English abstract.

\end{enabstract}

% !TeX root = ../main.tex

\begin{acknowledgments}


\end{acknowledgments}

\tableofcontents
% \listoffigures
% \listoftables

\newpage
\newenvironment{dedication}
     {\vspace{6ex}\begin{quotation}\begin{center}\begin{em}}
     {\par\end{em}\end{center}\end{quotation}}
\begin{dedication}
\begin{center}
test
\end{center}
\end{dedication}

\mainmatter
% !TeX root = ../main.tex

\chapter{Introduction}

The research of particle physics is aiming to understand how our universe works at its fundamental level. It can be accomplished by pursuing the mysteries of the basic construction of matter and energy, probing the interactions between elementary particles, and studying the nature of time and space. 

\textbf{Elementary particles}

From around the 6th century BC, ancient Greek philosophers Leucippus, Democritus, and Epicurus brought up a philosophical idea that everything is composed of ``uncuttable" elementary particles. 
In the 19th century, John Dalton, through his work on stoichiometry, concluded that each element of nature was composed of a single, unique type of particle. 
The particle was named as ``atom" after the Greek word atomos, with the meaning of ``indivisible". 
However this Dalton's atom theory was strongly challenged later. Near the end of 19th century, physicists discovered that Dalton's atoms are not, in fact, the fundamental particles of nature, 
but conglomerates of even smaller particles. 
Electron was discovered by J. J. Thomson in 1897~\cite{Thomson1897}, and then its charge was then carefully measured by Robert Andrews Millikan and Harvey Fletcher in their ``oil drop experiment" of 1909~\cite{PhysRev.2.109}. 
In early 20th-century, Rutherford's ``gold foil experiment" showed that the most mass of atom is concentrated in a small positive charge nucleus~\cite{Rutherford1911}. 
Then the discoveries of anti-particles (the positron in 1932) and other particles (e.g. the muon in 1936) indicate that more discoveries could be expected in future experiments.

Starting from 1950s, more accelerator facilities were put into service. 
During the 1950s and 1960s, various particles were found in particle collisions from increasingly high-energy beams, informally referred to as the ``particle zoo".
In 1964, the quark model was independently proposed by physicists Murray Gell-Mann and George Zweig, and experimentally confirmed of their existence in mid-1970s. 
In 1970s, the establishment of quantum chromodynamics  (QCD) postulated the fundamental strong interaction, experienced by quarks and mediated by gluons.

The well-known Standard model (SM) was developed during the latter half of the 20th century. 
At that time, confirmation of the top quark (1995), the tau neutrino (2000), and the Higgs boson (2012) have added further credence to the SM.
Now, the quarks, leptons and gauge bosons are the elementary particles in the framework of Standard Model of particle physics, 
that theoretically describes three of the four known fundamental forces (the electromagnetic, weak, and strong interactions, and not including the gravitational force) in the universe, 
as well as classifies all known elementary particles.

\textbf{Higgs mechanics and electroweak symmetry breaking}

In 1961, Sheldon Glashow, Steven Weinberg and Abdus Salam together brought forward a unified electroweak theory to combine the electromagnetic and weak interactions. 
In the standard model, under the condition that the energy is high enough but electroweak symmetry is unbroken, all elementary particles are massless. 
But measurements show the fact that the W and Z bosons actually have masses. 
Later on, the Higgs mechanics resolves this conundrum. 
The description of the mechanism adds a Higgs field in all space of the Standard Model, 
where the field causes spontaneous symmetry breaking during interactions,
and all massive particles in the Standard Model, including the W and Z bosons, interact with Higgs boson to acquire their mass.

Over the past few decades, with the combination of electroweak theory, the Higgs mechanics has been widely accepted. 
But the Higgs boson, the essential part to explain this mechanics of the property ``mass" for gauge bosons and fermions, had been the final missing piece in the Standard Model of particle physics at that time. 
The mass of Higgs boson was not specifically predicted by the SM, and it has been searched in several large experiments (eg. LEP at CERN, Tevatron at Fermilab, and LHC at CERN) with different energy. 
In 2012, the discovery of Higgs boson was finally announced by the ATLAS and CMS collaborations at the Large Hadron Collider (LHC) with its mass round 125~\gev. 
Peter Higgs and Francois Englert were award the Nobel Prize in Physics in the year of 2013 for their theoretical discovery of 
this mechanism that contributes to our understanding of the origin of mass.

\textbf{Contents of this thesis}

This dissertation is organized as follows. 
Section 2 briefly introduces the Standard Model of particle physics, the Higgs mechanism related to the dissertation and the LHC phenomenology. 
Section 3 gives an overview of the LHC and the ATLAS detector. 
The detector simulation and the reconstruction of physics objects are described in section 4. 
And then section 5 focuses on the Standard model ZZ production cross section measurement in $ZZ \rightarrow \llll$ channel, where $\mathrm{\ell}$ stands for electron or muon, 
and the observation of its electroweak component as well as its further prospects in High luminosity LHC (HL-LHC). 
Section 6 present the search of possible heavy resonances in $ZZ \rightarrow \llll$ channel. 
In the end, section 7 gives the summary and outlook for future physics in LHC.

% !TeX root = ../main.tex

\chapter{Theory}

%\section{The Standard Model of Particle Physics}
\section{Elementary Particles and Interactions in the Standard Model}
The standard model (SM) reflects our current understanding of elementary particles and several basic interactions.
It is a gauge quantum field theory containing the internal symmetries of the unitary product group $SU(3) \times SU(2) \times U(1)$, 
in which the color group $SU(3)$ represents the strong interaction, and $SU(2) \times U(1)$ describes the electroweak interactions.
Over the past decades, the SM has been widely tested through various experiments with extremely high precision.

\subsection{Elementary particles in the Standard Model}
\label{elementaryparticles}

The elementary particles in SM can be classified into 3 class: \textit{fermions, gauge bosons and the Higgs boson} as shown in Figure~\ref{fig:eleP-1}.
\begin{figure}[!htb]
  \centering
  \includegraphics[width=0.48\textwidth]{figures/Theory/Standard_Model_of_Elementary_Particles.pdf}
  \caption{The elementary particles of the Standard Model.}
  \label{fig:eleP-1}
\end{figure}

\textbf{Fermions}
The Standard Model includes 12 elementary particles of spin-$\frac{1}{2}$ obeying the Fermy-Dirac statistics, known as fermions. 
They are classified into two types: \textit{leptons and quarks} according to the their interctions.
The \textit{leptons} include three generations: electron ($e$) and electron neutrino ($\nu_{e}$); 
muon ($\mu$) and muon neutrino ($\nu_{\mu}$); tau ($\tau$) and tau neutrino ($\nu_{\tau}$).
The $e, \mu~and~\tau$ carry electric charge of -1 and three neutrinos are electrically neutral. 
All the leptons can participate in electroweak interactions.
Also there are three generations of \textit{quarks}: up ($u$) and down ($d$); charm ($c$) and strange ($s$); top ($t$) and bottom ($b$).
The defining property of the quarks is that they carry color charge (while leptons don't), and hence interact via the strong interaction, 
leting them be strongly bound from one to another, forming color-neutral composite 
particles (hadrons) containing either a quark and an antiquark (mesons) or three quarks (baryons).
In the meantime, u, c and t-quark carry electric charge of 2/3, and d, s and b-quark carry electric charge of -1/3. 
Hence they interact via all three interactions described in SM.
Each fermion also has a corresponding antiparticles.

\textbf{Gauge bosons}
act as force carriers that mediate the strong, weak, and electromagnetic interactions in SM.
They are spin-1 particles obeying the Bose-Einstein statistics. 
There are three types of gauge bosons:
\begin{itemize}
  \item The eight massless \textit{gluons} mediate the strong interactions between color charged particles (the quarks).
  \item The massless \textit{photons} mediate the electromagnetic force between electrically charged particles.
  \item The $W^{+}, W^{-} and Z$ bosons mediate the weak interactions between particles of different flavors (all quarks and leptons). All these three bosons are masive, the $W^{\pm}$ carries an electric charge of $+1$ and $−1$ and couples to the electromagnetic interaction. Z boson is electrically neutral.
\end{itemize}
Figure~\ref{fig:eleP-2} shows the Feynman diagrams of corresponding interactions in SM.
\begin{figure}[!htb]
  \centering
  \includegraphics[width=0.48\textwidth]{figures/Theory/Standard_Model_Feynman_Diagram_Vertices.png}
  \caption{The Feyman diagrams of interactions that form the basis of the standard model.}
  \label{fig:eleP-2}
\end{figure}

\textbf{Higgs boson}
is a masive scaler elementary particle with spin-0. 
It plays a unique role in the SM by explainning the origin of masses of massive gauge bosons ($W^{\pm} and Z$) and fermions. 
And it is the last discovered particle in SM.

%\subsection{Elementary particles in the Standard Model}
\subsection{Electroweak theory}
\label{ewktheory}
The electroweak interaction is the unified description of two of the four known fundamental interactions of nature: electromagnetism and the weak interaction.
It is based on the gauge group of $SU(2)_{L} \times SU(1)_{Y}$, in which $L$ is the left-handed fields and $Y$ is the weak hypercharge \cite{Langacker:2009my}.
It follows the Lagrangian of
\begin{equation} \label{eq:Lew}
	L_{EW} = L_{gauge} + L_{Higgs} + L_{fermion} + L_{Yukawa}
\end{equation}

$L_{gauge}$ is the \textbf{gauge term} part
\begin{equation}
	L_{gauge} = -\frac{1}{4} W^{i}_{\mu\nu} W^{\mu\nu i} - \frac{1}{4} B_{\mu\nu} B^{\mu\nu}
\end{equation}
where $W^{i}_{\mu}$ and $B_{\mu}$ respectively present the $SU(2)_{L}$ and $SU(1)_{Y}$ gauge fields, with the corresponding field strength tensors of
\begin{equation}
\begin{split}
	& B_{\mu\nu} = \partial_{\mu} B_{\nu} - \partial_{\nu} B_{\mu} \\
	& W^{i}_{\mu\nu} = \partial_{\mu} W^{i}_{\nu} - \partial_{\nu} W^{i}_{\mu} - g \epsilon_{ijk} W^{j}_{\mu} W^{k}_{\nu}
\end{split}
\end{equation}
In the equations above, $g$ is the $SU(2)_{L}$ gauge coupling and $\epsilon_{ijk}$ is the totally antisymmetric tensor.
The gauge Lagrangian has three and four-point self interactions of $W^{i}$, which result in triple and quartic gauge boson couplings.

The second term of the Lagrangian is the \textbf{scaler part}:
\begin{equation} \label{eq:Lhiggs}
	{L}_{Higgs} = \left(D^{\mu}\phi\right)^{\dagger}D_{\mu}\phi - V(\phi)
\end{equation}
where $\phi = \binom{\phi^{+}}{\phi^{0}}$  is a complex Higgs scalar,
and $V(\phi)$ is the Higgs potential which is restricted into the form of 
\begin{equation} \label{eq:Vhiggs}
	V(\phi) = +\mu^{2}\phi^{\dagger}\phi + \lambda\left(\phi^{\dagger}\phi\right)^{2}
\end{equation}
due to the combination of $SU(2)_{L} \times SU(1)_{Y}$ invariance and renormalizability.
In Eq.~\ref{eq:Vhiggs}, $\mu$ is a mass-dependent parameter and $\lambda$ is the quartic Higgs scalar coupling, 
which represents a quartic self-interaction between the scalar fields.
When $\mu^{2} < 0$, there will be spontaneous symmetry breaking (more details in section~\ref{symbreaking}).
To maintain vacuum stability, $\lambda > 0$ is required.
And in Eq.~\ref{eq:Lhiggs}, the gauge covariant derivative is defined as
\begin{equation}
	D_{\mu}\phi = \left(\partial_{\mu} +ig\frac{\tau^{i}}{2}W_{\mu}^{i} + \frac{ig^{'}}{2}B_{\mu}\right)\phi
\end{equation}
in which $\tau^{i}$ represents the Pauli matrices, and $g'$ is the $U(1)_{Y}$ gauge coupling.
The square of the covariant derivative results in three and four-point interactions between the gauge and scalar fields.

The third term of the Lagrangian is the \textbf{fermion part}
\begin{equation} \label{eq:Lfermion}
\begin{split}
  	{L}_{fermion} = \sum_{m=1}^{F} & ( \bar{q}_{mL^{i}}^{0}\gamma_{\mu}D_{\mu}q_{mL}^{0} + \bar{l}_{mL^{i}}^{0}\gamma_{\mu}D_{\mu}l_{mL}^{0} + \bar{u}_{mR^{i}}^{0}\gamma_{\mu}D_{\mu}u_{mR}^{0} \\
  	& + \bar{d}_{mR^{i}}^{0}\gamma_{\mu}D_{\mu}d_{mR}^{0} + \bar{e}_{mR^{i}}^{0}\gamma_{\mu}D_{\mu}e_{mR}^{0} + \bar{\nu}_{mR^{i}}^{0}\gamma_{\mu}D_{\mu}\nu_{mR}^{0})
\end{split}
\end{equation} 
In Eq.~\ref{eq:Lfermion}, m is the family index of fermions, F is the number of families.
The subscripts $L (R)$ stand for the left (right) chiral projection $\psi_{L(R)} \equiv \left(1 \mp \gamma_{5} \right) \psi/2$.
\begin{equation}
	q_{mL}^{0} = \binom{u_{m}^{0}}{d_{m}^{0}}_{L}   \qquad    l_{mL}^{0} = \binom{\nu_{m}^{0}}{e_{m}^{-0}}_{L}
\end{equation}
are the $SU(2)$ doublets of left-hand quarks and leptons, while 
$u_{mR}^{0}$, $d_{mR}^{0}$, $e_{mR}^{-0}$ and $\nu_{mR}^{0}$ are the right-hand singlets.

The last term in Eq.~\ref{eq:Lew} is \textbf{Yukawa term}
\begin{equation}
\begin{split}
	{L}_{Yukawa} =& -\sum_{m,n=1}^{F} [\Gamma_{mn}^{u}\bar{q}_{mL}^{0}\widetilde{\phi}u_{nR}^{0} + \Gamma_{mn}^{d}\bar{q}_{mL}^{0}\phi d_{nR}^{0} \\
	& + \Gamma_{mn}^{e}\bar{l}_{mn}^{0}\phi e_{nR}^{0} + \Gamma_{mn}^{\nu}\bar{l}_{mL}^{0}\widetilde{\phi}\nu_{nR}^{0}]+h.c.
\end{split}
\end{equation}
the matrices $\Gamma_{mn}$ refer to the Yukawa couplings between single Higgs doublet ($\phi$) and the various flavors of quarks (m) and leptons (n).


%\subsection{Electroweak theory}
\subsection{Higgs mechanism and Electroweak symmetry breaking}
\label{symbreaking}

As shown in previous subsection, the Lagrangian $L_{gauge}$ does not involve any mass term due to the requirement of gauge invariance.
So all the W and B bosons should be massless. But experimental observations show that the gauge bosons are massive.
Therefore, the gauge invariance must be broken spontaneously.
The Higgs field is introduced to break the $SU(2)_{L} \times U(1)_{Y}$ symmetry and
guage bosons and fermions can interact with Higgs filed to acquire their masses.
And this specific process is named \textit{Higgs mechanism} in SM.

The Higgs field $\phi$ is a doublet and can be written in a Hermitian basis as
\begin{equation}
	\phi = \binom{\phi^{+}}{\phi^{0}} = \frac{1}{\sqrt{2}} \binom{\phi_{1} - i\phi_{2}}{\phi_{3} - i\phi_{4}}
\end{equation}
where $\phi_{i} = \phi_{i}^{+}$ stand for four Hermitian field. 
In this new basis, the Higgs potential in Eq.~\ref{eq:Vhiggs} can be expressed as:
\begin{equation}
	V(\phi) = \frac{1}{2}\mu^{2}\left(\sum_{i=1}^{4}\phi_{i}^{2}\right) + \frac{1}{4}\lambda\left(\sum_{i=1}^{4}\phi_{i}^{2}\right)^{2}
\end{equation}
To simplify the situation, the axis in this four-dimensional space can be choosen to satisfied
~$\left<0\left| \phi_{i} \right|0\right> = 0$ for $i = 1, 2, 4$, and $<0\left| \phi_{3} \right|0> = v$. Thus,
\begin{equation}
	V(\phi) \rightarrow V(v) = \frac{1}{2}\mu^{2}v^{2} + \frac{1}{4}\lambda v^{4}
\end{equation}
To minimization of this potential depends on the sign of $\mu^{2}$ as shown in figure~\ref{fig:C2_Higgs_potential}.
When $\mu^{2} > 0$ the minimum occurs at $v = 0$, namely the vacuum is empty space and $SU(2)_{L} \times U(1)_{Y}$ symmetry is unbroken.
In the case of $\mu^{2} < 0$, the $v = 0$ symmetric point is no longer stable and the minimum occurs at nonzero value of 
$v = \left( -\mu^{2}/\lambda\right)^{1/2}$ which breaks the $SU(2)_{L} \times U(1)_{Y}$ symmetry.
\begin{figure}[!htb]
  \centering
  \includegraphics[width=0.7\textwidth]{figures/Theory/Vhiggs.png}
  \caption{Higgs potential $V(\phi)$ with $\mu^{2}>0$ (left) and $\mu^{2}<0$ (right).}
  \label{fig:C2_Higgs_potential}
\end{figure}
Thus, the classical vacuum $\phi_{0}$ of Higgs doublet can be expressed by
\begin{equation}
	\phi_{0} = \frac{1}{\sqrt{2}}\binom{0}{v}
\end{equation}
And to quantize around the classical vacuum in a general form:
\begin{equation}
	\phi = \frac{1}{\sqrt{2}} \binom{0}{v+H}
\end{equation}
Where H is a Hermitian field for physical Higgs scalar.
In this guage, the Lagrangian $L_{Higgs}$ in Eq.~\ref{eq:Lhiggs} takes a simple form
\begin{equation}
\begin{split} \label{eq:Lhiggs2}
	L_{Higgs} & = \left(D^{\mu}\phi\right)^{\dagger}D_{\mu}\phi - V(\phi) \\
	& = M_{W}^{2}W^{\mu+}W_{\mu}^{-}\left(1+\frac{H}{\nu}\right)^{2} + \frac{1}{2}M_{Z}^{2}Z^{\mu}Z_{\mu}\left(1+\frac{H}{\nu}\right)^{2} \\ 
        &   + \frac{1}{2}\left(\partial_{\mu}H\right)^{2} - V(\phi)
\end{split}
\end{equation}
where the W and Z fields are
\begin{equation}
\begin{split}
	& W^{\pm} = \frac{1}{\sqrt{2}} \left(W^{1} \mp iW^{2}\right) \\
	& Z = - sin\theta_{W}B + cos\theta_{W}W^{3}
\end{split}
\end{equation}
Therefore, in Eq.~\ref{eq:Lhiggs2} spontaneous symmetry breaking brings out masses for the W and Z gauge bosons
\begin{equation}
\begin{split}
	& M_{W} = \frac{gv}{2} \\
	& M_{Z} = \sqrt{g^{2} + g'^{2}} \frac{v}{2} = \frac{M_{W}}{cos\theta_{W}}
\end{split}
\end{equation}
where $\theta_{W}$ is the weak angle defined as
\begin{equation}
	sin\theta_{W} = \frac{g'}{\sqrt{g^{2} + g'^{2}}} \qquad cos\theta_{W} = \frac{g}{\sqrt{g^{2} + g'^{2}}} \qquad tan\theta_{W} = \frac{g'}{g}
\end{equation}
Then another gauge boson photon remains massless with the field of
\begin{equation}
	A = cos\theta_{W}B + sin\theta_{W}W^{3}
\end{equation}

After the symmetry breaking, the Higgs potential in unitary gauge can be written into
\begin{equation}
	V(\phi) = -\frac{\mu^{4}}{4\lambda} - \mu^{4}H^{2} + \lambda\nu H^{3} + \frac{\lambda}{4}H^{4}
\end{equation}
The first term in $V$ is a constant, while the second term denotes a (tree-level) mass of Higgs boson
\begin{equation}
	M_{H} = \sqrt{-2\mu^{2}} = \sqrt{2\lambda}v
\end{equation}
Due to the unknown of  quartic Higgs coupling \lambda, the Higgs mass is not predicted.
The third and fourth terms in Higgs potential $V$ denote the induced cubic and quartic interactions of the Higgs scalar.

Through the Higgs mechanism, fermions can also acquire their masses.
In the unitary gauge, Yukawa Lagrangian ($L_{Yukawa}$) can be written as a simple form of \cite{Pich:2015lkh}
\begin{equation}
	L_{Yukawa} = -\left(1+\frac{H}{v}\right) \left(m_{d}\bar{d}d + m_{u}\bar{u}u + m_{l}\bar{l}l\right)
\end{equation}
in which $m_{f} = \frac{y_{f}v}{\sqrt{2}}$ for $f = d, u, l$.

%\subsection{Higgs mechanics and electroweak symmetry breaking}
\subsection{Beyond the SM Higgs sector}
\label{bsmhiggs}

After the discovery of the Higgs boson by the ATLAS and CMS Collaborations at the LHC~\cite{20121, 201230} in 2012,
one question comes out: if this Higgs boson at around 125~\gev~ is fully responsible for the unitarization of the scattering amplitudes?
%If not, additional new physics must be present to play this role.
The possibility that this discovered particle is just a part of the extended Higgs sector by various extensions cannot be ruled out.
Many models, motivated by hierarchy and naturalness arguments, 
predicted the extended Higgs sector, such as the electroweak-singlet model and the two-Higgs-doublet models (2HDM).

\textbf{Singlet scalar extension of the SM} \\
The electroweak singlet model can be considered as the minimal extension of the SM Higgs sector~\cite{Profumo_2007}, encompassing a single gauge singlet real scalar field $S$.
In this model, a heavy, real singlet is introduced in addition to the SM one.
The associated zero temperature, tree-level scalar potential can be written as:
\begin{equation}
    V = V_{SM} + V_{HS} + V_{S}
\end{equation}
where
\begin{equation}
\begin{split}
    V_{SM} =  \mu^{2} \left( H^{\dagger}H \right) + \bar{\lambda_{0}} \left( H^{\dagger}H \right) \\
    V_{HS} = \frac{a_{1}}{2}\left( H^{\dagger}H \right)S + \frac{a_{2}}{2}\left( H^{\dagger}H \right)S^{2} \\
    V_{S} = \frac{b_{2}}{2} S^{2} + \frac{b_{3}}{3} S^{3} + \frac{b_{4}}{4} S^{4} \\ 
\end{split}
\end{equation}
where $H$ stands for the SM scalar field of the original Higgs mechanism.
After electroweak symmetry breaking, this model gives rise to two $CP$-even Higgs bosons, in which the lighter one is the Higgs boson that has been discovered at around 125~\gev.
And the new heavy scalar ($S$) is allowed to have both SM and non-SM decays.
One would expect to see suppressions of the branching ratio to SM Higgs decay modes, as the branching ratio to the pair of singlet-like scalars would be considerable. 

\textbf{Two Higgs Doublet Model} \\
The two-Higgs-doublet model (2HDM)~\cite{BRANCO20121} is another extension of SM Higgs sector carried by an additional scalar doublet.
In this model, through electroweak symmetry breaking, there are five physical Higgs bosons: two CP-even, one CP-odd, and two charged ones.
The most general CP-conserving 2HDM has seven free parameters:
\begin{itemize}
    \item The Higgs boson masses: $m_{h}$, $m_{H}$, $m_{A}$ and $m_{H^{/pm}}$.
    \item $tan \beta$: the ratio of the vacuum expectation values of the two doublets.
    \item $\alpha$: the missing angle between the CP-even Higgs bosons.
    \item $m_{12}^{2}$: the potential parameter that mixes the two Higgs doublets.
\end{itemize}
where the $m_{h}$ can be identified as the mass of observed Higgs boson at around 125~\gev, and $m_{H}$ is another heavy scaler with similar properties as $h$ boson.
The coupling of the neutral Higgs bosons to the W and Z are the same: 
\begin{enumerate}
    \item The coupling of the light Higgs, h, to either $WW$ or $ZZ$ is the same as the Standard Model coupling times $sin(\beta - \alpha)$ 
    \item The coupling of the heavier Higgs, H, is the same as the Standard Model coupling times $cos(\alpha - \beta)$.
    \item The coupling of the pseudoscalar, A, to vector bosons vanishes.
\end{enumerate}
The two Higgs doublets, $\Phi_{1}$ and $\Phi_{2}$, can couple to fermions (leptons and up- and down-type quarks) in several ways, which leads to several types of 2HDM models:
\begin{itemize}
    \item Type-\uppercase\expandafter{\romannumeral1} model: all quarks and leptons couple only to $\Phi_{2}$.
    \item Type-\uppercase\expandafter{\romannumeral2} model: down-type quarks and leptons couple to $\Phi_{1}$, and up-type quarks couple to $\Phi_{2}$.
    \item The ``lepton-specific" model: leptons couple to $\Phi_{1}$, while all quarks couple to $\Phi_{2}$.
    \item The ``flipped" model: down-type quarks couple to $\Phi_{1}$, while up-type quarks and leptons couple to $\Phi_{2}$.
\end{itemize}

%\subsection{Beyond the SM Higgs sector}

\section{Phenomenology of Large Hadron Collider}
The Large Hadron Collider (LHC)~\cite{Bruning:2004ej, Buning:2004wk, Benedikt:2004wm} was built as a bridge between the experiments and the theories.
Physicists hope that the LHC can help to answer some open questions in fundamental physics, 
such as the basic laws of interactions, the forces among the elementary particles, 
the deep structure of space and time, and the interrelation between quantum mechanics and general relativity.
This section will talk about firstly the general introduction of Physics inside hadronic collision,
then followed by two important LHC phenomenologies of the Higgs physics and Diboson physics that are related closely to this dissertation.

\subsection{Physics at hadronic collision}
\label{hadroniccollision}

Protons are not the elementery particle, which actually be composed of quarks and gluons.
So in proton-proton (pp) collision at LHC, it is not protons themselves interact but quarks and gluons.
Scattering processes can then be further classified into either \textit{hard} or \textit{soft} processes
according to the momentum transfer during the interaction \cite{Dremin:2005wd}.
QCD, as an underlying theory for both two process, its approach and level of understandings in two cases are quite different.
For hard process, eg. Higgs, vector bosons and jets production, the rates and event
properties can be precisely predicted based on perturbation theory.
However, for soft processes like total cross-section, the underlying events, the rates and properties are dominated by non-perturbative QCD effects
that are less understood.
For many hard processes, the hard interactions are accompanied by soft ones.
A example of the hadronic collision is illustrated in figure~\ref{fig:C2_had_col}. 
\begin{figure}[!htb]
  \centering
  \includegraphics[width=0.7\textwidth]{figures/Theory/hh_collision.png}
  \caption{Schematic view of a hadron-hadron collision \cite{Womersley:2000cx}.}
  \label{fig:C2_had_col}
\end{figure}
and the typical features are summarized as below:
\begin{itemize}
	\item \textbf{Parton Distribution Function (PDF)}: $f_{i}\left(x, Q^{2}\right)$ gives the probability of a parton with flavor $i$ (quark or gluon), carring amomentum fraction of $x$ and at the energy of Q in a proton. Parton distribution function cannot be fully calculated by perturbative QCD because of the inherent non-perturbative nature of partons. There are many different sets of PDFs that are determined by a fit to data from experimental observables in various processes. As an example, figure~\ref{fig:C2_PDF4LHC15} for \textit{PDF4LHC15} which is based on the combination of the \textit{CT14}, \textit{MMHT14} and \textit{NNPDF3.1} NNLO PDF sets \cite{Lin:2017snn}.
	\item \textbf{Fragmentation and hadronization}: The processes to produce final state particles (or jets) from the partons produced in hard scattering.
	\item \textbf{Initial/Final state radiation}: The incoming and outgoing partons that carry color charge can emit QCD radiation, which gives rise to additional jets. Also the charged incoming and outgoing particles can emit QED radiations with photons.
	\item \textbf{Underlying events}: Products from soft processes (not come from the primary hard scattering) as the remnants of scattering interactions.
\end{itemize}
\begin{figure}[!htb]
  \centering
  \includegraphics[width=0.7\textwidth]{figures/Theory/PDF4LHC15.png}
  \caption{The PDF4LHC15 NLO PDFs at a low scale $\mu^{2} = Q^{2} = 4 GeV^{2}$ (left) and at $\mu^{2} = Q^{2} = 100 GeV^{2}$ (right) as a function of x.}
  \label{fig:C2_PDF4LHC15}
\end{figure}

\textbf{Cross section of hard scaterring} 

According to \textit{QCD factorization theorems} \cite{Collins:1989gx}, the perturbative calculations can be applied to many important
hard processes involving hadrons. The basic problem addressed by factorization theorems is how to calculate high energy cross sections.
Conder the process of scattering between two hardons A and B to produce a final state X, the cross section $\sigma$ can be obtained 
by summing over all the subprocess cross section $\hat{\sigma}$ \cite{Stirling:1194745}
\begin{equation}
	\sigma_{AB} = \int dx_{a} dx_{b} f_{a/A}\left(x_{a}\right) f_{b/B}\left(x_{b}\right) \hat{\sigma}_{ab\rightarrow X}
\end{equation}
where $f_{q/A}\left(x_{q}\right)$ is the parton distrition functions of parton $q$.
Taking into account the leading order correction:
\begin{equation} \label{eq:xs2}
	\sigma_{AB} = \int dx_{a} dx_{b} f_{a/A}\left(x_{a}Q^{2}\right) f_{b/B}\left(x_{b}Q^{2}\right) \hat{\sigma}_{ab\rightarrow X}
\end{equation}
where $Q^{2}$ represents large momentum scale that characterizes the hard scattering.
Later on, since the finite corrections were not universal and had to be calculated separately for each process,
the perturbative $O\left(\alpha_{S}^{n}\right)$ corrections to the leading logarithm cross section in Eq.~\ref{eq:xs2}
need to be applied, one can get:
\begin{equation}
	\sigma_{AB} = \int dx_{a} dx_{b} f_{a/A}\left(x_{a}\mu_{F}^{2}\right) f_{b/B}\left(x_{b}\mu_{F}^{2}\right) \hat{\sigma}_{ab\rightarrow X}\left(\alpha_{S},\mu_{R},\mu_{F}\right)
\end{equation}
in which $\mu_{F}$ is \textit{factorization scale} which can represent the scale that separates the long- and short-distance physics,
and $\mu_{R}$ is the \textit{renormalization scale} for QCD running coupling.
$\hat{\sigma}_{ab\rightarrow X}$ is the parton-level hard scattering cross section that can be calculated perturbatively in QCD with the form of
\begin{equation} \label{eq:xs3}
	\hat{\sigma}_{ab\rightarrow X}\left(\alpha_{S},\mu_{R},\mu_{F}\right) 
		= \left(\alpha_{S}\right)^{n} \left[ \hat{\sigma}^{(0)}
		+ \left(\alpha_{S}/2\pi\right) \hat{\sigma}^{(1)}\left(\mu_{R},\mu_{F}\right)
		+ \left(\alpha_{S}/2\pi\right)^{2} \hat{\sigma}^{(2)}\left(\mu_{R},\mu_{F}\right)
		+ ... \right]
\end{equation}
where $\hat{\sigma}^{(0)}$ stands for the leading-order (LO) partonic cross section,
while $\hat{\sigma}^{(1)}$ and $\hat{\sigma}^{(2)}$ are the next-to-leading-order (NLO) and
next-to-next-to-leading-order (NNLO) cross section.

$\mu_{R}$ and $\mu_{F}$ depend on the order of truncation in Eq.~\ref{eq:xs3}.
In principle, if cross section is calculated to all orders, it is invariant under changes in these parameters.
The choices of $\mu_{R}$ and $\mu_{F}$ are arbitrary. 
To avoid unnaturally large logarithms reappearing in the perturbation series,
it is sensible to choose $\mu_{R}$ and $\mu_{F}$ values of the order of the typical momentum scales of
the hard scattering process and $\mu_{R} = \mu_{F}$ is also often assumed.
Take Drell–Yan process as an example, the standard choice is $\mu_{R} = \mu_{F} = m_{ll}$, 
where $m_{ll}$ is the invariant mass of dilepton pair.

%\subsection{Physics at hadron colliders}
\subsection{Higgs physics at the LHC}
\label{higgs}

One important physics purpose of the LHC is searching for the Higgs boson, which was the last missing part in the SM.
This section will discuss both the production and decay modes of the SM Higgs boson in proton-proton collision.

%% ================================ Higgs production ===============================
\textbf{Higgs productions}

The Higgs boson can be produced through several processes.
There are 4 main production modes at the LHC: gluon-gluon Fusion (\textit{ggF}), vector boson Fusion (\textit{VBF}),
associated production with vector-bosons (\textit{VH}) (also called the Higgs Strahlung) 
and associated production with a pair of top/anti-top quarks (\textit{ttH})~\cite{Grojean:2243593}.
Figure~\ref{fig:higgs_productions_fd} shows the corresponding Feynman diagrams of each process at leading order.
\begin{figure}[!htb]
  \centering
  \includegraphics[width=0.8\textwidth]{figures/Theory/Figures_FeynmanHprod.png}
  \caption{Feynman diagrams of four Higgs production modes:
	   (a) ggF; (b) VBF; (c) VH; (d) ttH.}
  \label{fig:higgs_productions_fd}
\end{figure}
For pp collisions, the cross section of production of Higgs boson is a function of centre-of-mass energy $\sqrt{s}$. 
Figure~\ref{fig:higgs_productions_xs} depicts the cross section of SM Higgs, whose mass is 125 GeV, for several different production modes when centre-of-mass energy varying from 6 to 15~\tev.
Figure~\ref{fig:higgs_productions_xs2} shows the prospect of production cross section  
as a function of Higgs mass from 10 to 2000~\gev~for pp collision at the centre-of-mass energy of 13~\tev~ and 14~\tev~ \cite{deFlorian:2227475}. 
\begin{figure}[!htb]
  \centering
  \includegraphics[width=0.5\textwidth]{figures/Theory/Plot_Escan_H125_new_sqrt.pdf}
    \caption{The SM Higgs boson (125~\gev) production cross section for various production modes as a function of the centre-of-mass energy for pp collision.}
  \label{fig:higgs_productions_xs}
\end{figure}
\begin{figure}[!htb]
  \centering
  \includegraphics[width=0.45\textwidth]{figures/Theory/plotAll_14tev_BSM_sqrt.pdf}
  \includegraphics[width=0.45\textwidth]{figures/Theory/plotAll_13tev_BSM_sqrt.pdf}
  \caption{Higgs boson production cross section for various production modes as a function of the Higgs mass for $\sqrt{s}$ = 13~\tev~ (left) and 14~\tev~ (right) for pp collision.}
  \label{fig:higgs_productions_xs2}
\end{figure}

%% ================================ Higgs decays ===================================
\textbf{Higgs decays}

The Higgs boson can interact with gauge bosons and fermions through gauge coupling and the Yukawa coupling as introduced in section~\ref{symbreaking}.
Figure~\ref{fig:higgs_decay_fd} depicts the Feynman diagrams of various possible Higgs decay channels.
\begin{figure}[!htb]
  \centering
  \includegraphics[width=0.8\textwidth]{figures/Theory/Figures_Feynman_Hdecay.png}
  \caption{SM Higgs decay channels.}
  \label{fig:higgs_decay_fd}
\end{figure}
The branching ratio of Higgs boson decaying into different final states as a function of Higgs mass is shown in figure~\ref{fig:higgs_decay_br}.
\begin{figure}[!htb]
  \centering
  \includegraphics[width=0.5\textwidth]{figures/Theory/Higgs_BR_LM.eps}
  \caption{Branching ratio of Higgs decays in various channels as a function of Higgs mass~\cite{Heinemeyer:1559921}. }
  \label{fig:higgs_decay_br}
\end{figure} 

%% ====================================== High mass related ===========================
%(\textbf{BSM Higgs models})



%\subsection{Higgs Physics at LHC}
\subsection{Diboson physics}
\label{diboson}

The study of diboson physics is another important test for SM of particle physics in electroweak sector, 
in which vector boson scattering is a key process for probing the mechanism of the electroweak symmetry breaking (EWSB).
In the meantime, the non-resonant diboson productions are crucial backgrounds for Higgs studies at the LHC, which make the precise measurement of their cross section becomes very important.

%% ====================================== Diboson productions ========================
\textbf{Diboson productions}

About $90\%$ of diboson productions at hadron collider is from quark-antiquark annihilation,
while others are contributed from gluon initiated process.
Figure~\ref{fig:diboson_fd1} shows the tree-level Feynman diagrams of diboson production.
\begin{figure}[!htb]
  \centering
  \includegraphics[width=0.7\textwidth]{figures/Theory/diboson_prod_fey.png}
  \caption{The tree-level Feynman diagrams of diboson production at the LHC.}
  \label{fig:diboson_fd1}
\end{figure}
Then figure~\ref{fig:diboson_xs1} illuminates the total production cross-section presented by ATLAS
as a function of centre-of-mass energy $\sqrt{s}$ from 7 to 13 TeV for several diboson processes comparing 
to some other major processes in hadron collision.
The cross section for diboson processes are measured at next-to-next leading order (NNLO). 
\begin{figure}[!htb]
  \centering
  \includegraphics[width=0.7\textwidth]{figures/Theory/ATLAS_n_SMSummary_SqrtS.pdf}
  \caption{Total production cross-section presented by ATLAS as a function of centre-of-mass energy $\sqrt{s}$ from 7 to 13 TeV for some selected processes,
	   the diboson measurements are scaled by a factor 0.1 to allow a presentation without overlaps.}
  \label{fig:diboson_xs1}
\end{figure}

%% ======================================= Vector boson scattering ===================
\textbf{Vector boson scattering}

The $SU(2)_{L} \times U(1)_{Y}$ structure in SM predicts self-interactions between electroweak gauge bosons.
Those self-couplings can involve either three or four gauge bosons at a single vertex, known as triple gauge coupling (\textit{TGC}) and
quartic gauge couplings (\textit{QGC}), respectively.
Vector boson scattering or fusion (\textit{VBS} or \textit{VBF}) is carried out 
by four electroweak vector bosons, namely $Z$, $W^{\pm}$ and photon $(\gamma)$ as the Feynman diagrams shown in figure~\ref{fig:vbs_fd1}. 
And the vertexes include either those self-interactions
or the interactions with the Higgs boson described in figure~\ref{fig:vbs_fd2}.
\begin{figure}[!htb]
  \centering
  \includegraphics[width=0.4\textwidth]{figures/Theory/VBS.png} 
  \caption{Feynman diagrams of the vector boson scattering.}
  \label{fig:vbs_fd1}
\end{figure}
\begin{figure}[!htb]
  \centering
  \includegraphics[width=0.6\textwidth]{figures/Theory/vbs_tgc.png} \\
  \includegraphics[width=1.0\textwidth]{figures/Theory/vbs_qgc.png} \\
  \includegraphics[width=0.6\textwidth]{figures/Theory/vbs_higgs.png} 
  %\includegraphics[width=1.0\textwidth]{figures/Theory/VBS_vertex.png}
  \caption{Feynman disgrams of vertexes involving QGC, TGC and Higgs.}
  \label{fig:vbs_fd2}
\end{figure}

The amplitudes of leading-order (LO) VBS can be expressed as\cite{PhysRevD.87.093005}:
\begin{equation} \label{eq:amp_tgc}
\begin{split}
	& {iM}_{TGC}^{s-channel} = -i\frac{g_{1}^{2}}{4m_{W}^{4}}[s(t-u)-3m_{W}^{2}(t-u)] \\
	& {iM}_{TGC}^{t-channel} = -i\frac{g_{1}^{2}}{4m_{W}^{4}}\left[(s-u)t-3m_{W}^{2}(s-u)+\frac{8m_{W}^{2}}{s}u^{2}\right]
\end{split}
\end{equation}
\begin{equation} \label{eq:amp_qgc}
	{iM}_{QGC} = i\frac{g_{1}^{2}}{4m_{W}^{4}}\left[s^{2}+4st+t^{2}-4m_{W}^{2}(s+t)-\frac{8m_{W}^{2}}{s}ut\right]
\end{equation}
\begin{equation} \label{eq:amp_higgs}
\begin{split}
	{iM}_{Higgs} & = -i\frac{C_{\nu}^{2}g_{1}^{2}}{4m_{W}^{2}}\left[\frac{(s-2m_{W}^{2})^{2}}{s-m_{H}^{2}} + \frac{(t-2m_{W}^{2})^{2}}{t-m_{H}^{2}}\right] \\
                     & \simeq -i\frac{C_{\nu}^{2}g_{1}^{2}}{4m_{W}^{2}}(s+t)
\end{split}
\end{equation}
Combining s- and t-channel of TGC in Eq.~\ref{eq:amp_tgc} and the QGC term in Eq.~\ref{eq:amp_qgc}:
\begin{equation} \label{eq:self_coup}
	{iM}_{TGC} + {iM}_{QGC}= i\frac{g_{1}^{2}}{4m_{W}^{2}}(s+t) + {O}((s/m_{W}^{2})^{0})
\end{equation}
In Eq.~\ref{eq:self_coup}, the amplitude grows as a function of center-of-mass energy ($\sqrt{s}$),
which violates the unitarity in the TeV region.
Considering the Higgs term in Eq.~\ref{eq:amp_higgs} perfectly cancels out this growing,
and the remaining term ${O}((s/m_{W}^{2})^{0})$ depends on the total amplitude in SM.

In conclusion, the Higgs boson acts as "moderator" to unitarize high-energy longitudinal vector boson scattering
by restoring unitarity of total amplitude in high energy region.

%\subsection{Diboson physics}

\chapter{The Large Hadron Collider and the ATLAS Detector}

\section{The Large Hadron Collider}
Located near the French-Swiss border at the European Organization for Nuclear Research (CERN),
the Large Hadron Collider (LHC) is the largest and most powerful facility for particle physics in the world.
It's the proton-proton collider with the centre-of-mass energy designed up to 14~\tev.
The beams inside the LHC are made to collide at four locations around its 27-kilometer accelerator ring, 
corresponding to four particle experiments - the ATLAS, CMS, ALICE and LHCb.
With its unprecedented energy, the LHC is designed to observe physics that involve highly massive particles,
which have never been observed in previous accelerators with lower energies.

\subsection{Operation history and machine layout}

%% ================================= history ==============================
\textbf{Operation history}

The LHC \cite{Bruning:2004ej, Buning:2004wk, Benedikt:2004wm, Evans_2008} 
is a two-ring-superconducting-hadron accelerator and collider lies in a tunnel about 100 metres underground.
It's designed to provide proton-proton collisions at the centre-of-mass energy up to 14~\tev~
with a unprecedented luminosity of $10^{34} cm^{-2} s^{-1}$.
In the meantime, it can also collide heavy (Pb) ions with an energy of 2.8~\tev~ per nucleon and a peak luminosity of $10^{27} cm^{-2} s^{-1}$.
Table~\ref{tab:LHC_parameters} shows the main design parameters of the LHC for proton-proton collisions.
\begin{table}[htbp]
  \centering
  \caption{Summary of design parameters of the LHC for pp collisions.}
  \label{tab:LHC_parameters}
  \begin{tabular}{cc}
    \hline
    Circumference	& 26.7 km\\
    Beam energy at collision	& 7 ~\tev\\
    Beam energy at injection	& 0.45~\tev \\
    Dipole field at 7~\tev	& 8.33 T \\
    Luminosity		& $10^{34} cm^{-2} s^{-1}$ \\
    Beam current	& 0.56 A \\
    Protons per bunch	& $1.1 \times 10^{11}$ \\
    Number of bunches	& 2808 \\
    Nominal bunch spacing	& 24.95 ns \\
    Normalized emittance	& 3.75 $\mu$m \\
    Total crossing angle	& 300 $\mu$rad \\
    Energy loss per turn	& 6.7 keV \\
    Critical synchrotron energy	& 44.1 eV \\
    Radiated power per beam	& 3.8 kW \\
    Stored energy per beam	& 350 MJ \\
    Stored energy in magnets	& 11 GJ \\
    Operating temperature	& 1.9 K \\
    \hline
  \end{tabular}
\end{table}

The LHC was built from 1998 to 2008. 
It started its first beam in September 2008, but then was interrupted by a quench incident only after a few days running.
Then it resumed the operation in November 2009 with a low energy beams.
From March 2010, physics runs took place at the centre-of-mass energy of 7~\tev.
Later on, this energy was increased in 2012 to $\sqrt{s} = 8~\tev$, with an integrated luminosity of 20.3~\ifb, and this period is called ``run-1".
After run-1, the LHC was shut down for two years for hardware maintenance and upgrade, starting from February 2013.

The second operation period with higher centre-of-mass energy at 13~\tev~ started from 2015 called ``run-2".
And it continued to the end of 2018 with total integrated luminosity reaching about 147 $fb^{-1}$ for ATLAS experiment.
Figure~\ref{fig:lumi_vs_month} shows the cumulative luminosity as a function of time in month delivered to ATLAS experiment during stable beams 
in years from 2011 to 2018.
\begin{figure}[!htb]
  \centering
  \includegraphics[width=0.6\textwidth]{figures/Detector/intlumivsyear.pdf}
  \caption{Cumulative luminosity vs time in the years from 2011 to 2018 for ATLAS detector.}
  \label{fig:lumi_vs_month}
\end{figure}

%% ======================================= layout ==============================
\textbf{Machine layout}

The layout of CERN accelerator complex is shown in figure~\ref{cern_layout}.
The protons are accelerated by a series of machines before being injected into the main ring.
At beginning, the 50~\mev~ protons are produced in the linear particle accelerator LINAC2, 
and further accelerated to 1.4~\gev~ in Proton Synchrotron Booster (PSB).
The protons are then injected into the Proton Synchrotron (PS) to gain the energy of 26~\gev~ and further accelerated to 450~\gev~ in Super Proton Synchrotron (SPS).
At the end, they are injected into the main ring, and can reach a maximum energy of 7~\tev.

\begin{figure}[!htb]
  \centering
  \includegraphics[width=1.0\textwidth]{figures/Detector/LHC_v2017.png}
  \caption{Layout of CERN LHC complex \cite{Mobs:2197559}.}
  \label{cern_layout}
\end{figure}

The collisions can occur in 4 points, with corresponding 4 major detector experiments that are briefly described as follows:
\begin{itemize}
	\item \textbf{ATLAS}: A Toroidal LHC ApparatuS, one general-purpose particle detector experiment and the detector with largest volume at the LHC. It is designed to search for the Higgs boson, test the stardand model of particle physics and search for possible beyond SM physics.
	\item \textbf{CMS}: Compact Muon Solenoid, another large general-purpose particle physics detector, with the same physics goal as ATLAS and also cross check with ATLAS.
	\item \textbf{ALICE}: A Large Ion Collider Experiment, it is optimized to study heavy-ion (Pb-Pb nuclei) collisions at a centre-of-mass energy of 2.76~\tev~ per nucleon pair.
	\item \textbf{LHCb}: Large Hadron Collider beauty, it is a specialized b-physics experiment, designed primarily to measure the parameters of CP violation in the interactions of b-hadrons.
\end{itemize}



%\subsection{Operation history and machine layout}
\subsection{Luminosity and pile-up}

\textbf{Luminosity}

In beam–beam collisions, the event rate for a process is written as~\cite{Evans_2008}:
\begin{equation}
	N = \mathcal{L} \sigma
\end{equation}
where $\sigma$ is the cross section of the process, and $\mathcal{L}$ is the luminosity.
To study rare events, $\mathcal{L}$ must be as high as possible.
The luminosity only depends on the beam parameters as:
\begin{equation} \label{eq:lumi}
	\mathcal{L} = \frac{ N_{b}^{2} n f_{r} \gamma}{4\pi \epsilon_{n} \beta^{*}}
\end{equation}
in which $N_{b}$ represents the number of particles per bunch, $n$ denotes the number of bunches per beam,
$f_{r}$ is the revolution frequency, and $\gamma$ is relativistic $\gamma$ factor, 
$\epsilon_{n}$ is the normalized transverse emittance and $\beta^{*}$ denotes the $\beta$ function at the collision point.
To reduce the beam-beam interaction effects, the bunches must have a crossing angle,
which produces a geometrical luminosity reduction factor $F$:
\begin{equation}
	F = 1 / \sqrt{1 + \left( \frac{\theta_{c}\sigma_{Z}}{2\sigma^{*}} \right) }
\end{equation}
where $\theta_{c}$ denotes the crossing angle at the interaction point, $\sigma_{Z}$ is the root mean square (RMS) bunch length
and $\sigma^{*}$ is the transverse RMS beam size at crossing point.

The luminosity expressed in Eq.~\ref{eq:lumi} is normally the instantaneous luminosity.
In fact the running conditions usually vary with time, so the luminosity can change as well.
To take into account the time dependence, integrated luminosity is invited, by integraling the instantaneous luminosity over time:
\begin{equation}
	L = \int \mathcal{L}(t) dt
\end{equation}
The unit of integrated luminosity we commonly use is $b^{-1}$ that satisfying $1 b^{-1} = 10^{24} cm^{-2}$.
Figure~\ref{fig:lumi_vs_time} shows integrated luminosity as a function of time delivered to ATLAS (green), 
recorded by ATLAS (yellow), and certified to be good quality data (blue) during run-2 pp collisions.
For most physics analysis, the data with good quality (require to satisfy \textit{Good Run List}) is used.
\begin{figure}[!htb]
  \centering
  \includegraphics[width=0.6\textwidth]{figures/Detector/intlumivstimeRun2DQall.pdf}
  \caption{Integrated luminosity vs delivered month from 2015 to 2018 in ATLAS experiment.}
  \label{fig:lumi_vs_time}
\end{figure}

\textbf{Pile-up}

In collisions, multiple interactions can happen in one single bunch crossing, which is called ``\textit{pile-up}".
The variable $\left< \mu \right>$, representing the average number of interactions per bunch crossing that used to describe pile-up effect, is defined as:
\begin{equation}
	\left< \mu \right> = \frac{L_{tot}\sigma}{f_{r}n_{bunch}}
\end{equation}
where $L_{tot}$ is the instantaneous luminosity, $\sigma$ denotes the inelastic cross section,
$f_{r}$ represents the LHC revolution frequency and $n_{bunch}$ is the number of colliding bunches.
Usually, with increasing luminosity, the pile-up becomes more significant.
Figure~\ref{fig:run2_mu} shows the luminosity-weighted distribution of the mean number of interactions per crossing
for pp collision data from 2015 to 2018 (full run-2), the challenge of pile-up increased in each year.
\begin{figure}[!htb]
  \centering
  \includegraphics[width=0.6\textwidth]{figures/Detector/mu_2015_2018.pdf}
  \caption{Number of interactions per crossing weighted bt luminosity from 2015 to 2018 in ATLAS experiment.}
  \label{fig:run2_mu}
\end{figure}

%\subsection{Luminosity and pile-up}

\section{ATLAS detector}
\subsection{Detector overview}

ATLAS (A Toroidal LHC ApparatuS) is the largest volume detector ever constructed for a particle collider.
It has the dimensions of a cylinder with 46 meters long, 25 meters in diameter, and sits in a cavern 100 meters below ground.
The detector contains about 3000km of cables and it weights 7000 tonnes.

This paragraph briefly summarizes the coordinate system and nomenclature used to describe the ATLAS detector \cite{Collaboration_2008}.
As shown in figure~\ref{fig:coordinate}, we define the nominal interaction point as the origin of the coordinate system, the beam direction as the \textit{z}-axis and the \textit{x-y} plane is transverse to the beam direction.
The positive \textit{x}-axis is defined to be the direction pointing to the center of the LHC ring, 
while the positive \textit{y}-axis is pointing upwards.
\begin{figure}[!htb]
  \centering
  \includegraphics[width=0.9\textwidth]{figures/Detector/Coordinate_system_atlas.png}
  \caption{Coordinate system used by the ATLAS experiment at the LHC \cite{Perez:phdthesis}.}
  \label{fig:coordinate}
\end{figure}
There are two sides of detector A and C, in which A(C)-side is defined as with positive (negative) \textit{z}.
The azimuthal angle $\phi$ is measured as usual around the beam axis, while the polar angle $\theta$ is the angle from the beam axis
In physics analysis, we usually use the pseudorapidity instead of $\theta$ angle, which is designed as $\eta = - ln [tan\left( \frac{\theta}{2}\right)]$. 
For massive objects (eg. jets), the rapidity $y = \frac{1}{2} ln[ \frac{E+p_{z}}{E-p_{z}}]$ is used.
In addition, the \textit{transverse} momentum $p_{T}$, \textit{transverse} energy $E_{T}$ and the missing \textit{transverse} energy $E_{T}^{miss}$ are defined in \textit{x-y} plane.
The commonly used distance measurement $\Delta R$, is defined in the pseudorapidity-azimuthal angle space as $\Delta R = \sqrt{ \Delta\eta^{2} + \Delta\phi^{2}}$.

The overall ATLAS layout is shown in figure~\ref{fig:atlas_layout}, which is forward-backward symmetric with respect to the interaction point.
\begin{figure}[!htb]
  \centering
  \includegraphics[width=1.0\textwidth]{figures/Detector/atlas_layout.jpg}
  \caption{Cut-away view of the ATLAS detector \cite{Pequenao:1095924}.}
  \label{fig:atlas_layout}
\end{figure}
The magnet configuration comprises a thin superconducting solenoid surrounding the inner-detector cavity, 
and three large superconducting toroids (one barrel and two end-caps) arranged with an eight-fold azimuthal symmetry around the calorimeters.

\textbf{The inner detector}, which is the innermost part of ATLAS, is immersed in a 2 T solenoidal magnetic field.
It's used for pattern recognition, momentum and vertex measurements and electron identification, with the combination of tracking system.

\textbf{The calorimeter} is outside the inner detector, for electromagnetic and hadronic energy measurements.
High granularity liquid-argon (LAr) electromagnetic sampling calorimeters is used to measure energy and position resolution with range up to $|\eta| < 3.2$ for electrons and photons.
For hadronic calorimetry, a scintillator-tile calorimeter is used in the range of $|\eta| < 1.7$.
The LAr forward calorimeters provide both electromagnetic and hadronic energy measurements with the coverage up to $|\eta| = 4.9$.

\textbf{The muon spectrometer} is in the outermost side.
The air-core toroid system, with a long barrel and two inserted end-cap magnets, provides strong bending power in a large volume within a light and open structure.
Multiple-scattering effects are minor, and excellent muon momentum resolution can be achieved.

%\subsection{Dector overview}
\subsection{Physics requirement}

As mentioned previously, ATLAS is one of two general-purpose particle detector experiment at LHC.
It's designed to take advange of the unprecedented energy at LHC.
The Higgs boson was discovered as one of its benchmark, and lots of precise tests and measurements of SM is on going.
In the meantime, ATLAS is also designed to abserve the phenomena that involve highly massive particles, such as heavy beyond standard model (BSM) gauge bosons $Z'$ and $W'$.
It can also explore the possibility of extra dimensions proposed by several models in TeV region.
To fulfil many diverse physics goals, a set of general requirements are needed:
\begin{itemize}
	\item The speed-fast and radiation-hard electronics are required due to the experimental conditions at LHC. 
	\item High detector granularity is needed to reduce the overlapping events and handle the particle fluxes.
	\item Large acceptance in pseudorapidity and azimuthal angle coverage is needed.
	\item For inner detector, good charged-paricle momentum resolution and reconstruction efficiency are crucial. And the vertex detectors close to the interaction region are required to be able to observe secondary vertices for offline tagging of $\tau$-lepton and $b$-jets.
	\item Good electromagnetic (EM) calorimetry for electron and photon, as well as full-coverage hadronic calorimetry for accurate jet and missing transverse energy measurements, are importantly required, since these measurements form the basis of many studies.
	\item Good muon spectrometer is also required for muon identification and momentum resolution measurement over a wide range of momenta.
	\item Highly efficient but with sufficient background rejection triggers are also needed and extramely important for objects with low transverse-momentum. 
\end{itemize}

%Table~\ref{tab:detector_goal} summarizes the major performance for different parts of ATLAS detector.
More detailed descriptions of each sub-system will be given in the following subsections.

%\subsection{Physics requirement}

\subsection{Magnet system}

A strong magnetic field is required for precise measurement of charged particle momenta.
The ATLAS detector uses two large superconducting magnet systems, a hybrid system of a central superconducting solenoid and three outer superconducting toroids, to bend charged particles~\cite{McFayden:phdthesis}.
The total magnet system is 22 m in diameter and 26 m in length as shown in figure~\ref{fig:megnet_sys}.
\begin{figure}[!htb]
  \centering
  \includegraphics[width=0.7\textwidth]{figures/Detector/magnetSystems.png}
  \caption{Schematic diagram of the ATLAS magnet system.}
  \label{fig:megnet_sys}
\end{figure}

The central solenoid produces two Tesla (T) magnetic field surrounding the inner Detector.
When obtaining such high field strength, at the same time, the solenoid needs to be thin in order to reduce the material in front of the calorimeter.

The outer toroid system comprises one barrel superconducting toroid and two end-caps.
The barrel one is composed of eight coils encased in individual racetrack-shaped, stainless-steel vacuum vessels and produces the magnetic field in the cylindrical volume surrounding the calorimeters.
Each end-cap toroid consists of a single cold mass built up from eight flat, square coil units and eight keystone wedges and provides a magnetic field of approximately 1 T for the muon detectors in the end-cap regions.

%\subsection{Magnet system}
\subsection{Inner detector}

The inner detector, as shown in figure~\ref{fig:inner_dec}, is the detector closest to beam pipe.
It's used to measure the position of charged particle tracks in high precision together with good momentum resolution,
in which the measurement of primary and secondary vertices and electron identification are especially important.
Due to the extremely high luminosity produced by the LHC, the precise measurements of vertex and momentum becomes tough and fine-granularity detectors are crucial.
The inner detector consists of three subdetectors that will be described as below.
\begin{figure}[!htb]
  \centering
  \includegraphics[width=0.8\textwidth]{figures/Detector/ID_newTRT_d3.png}
  \caption{Schematic diagram of the ATLAS inner detector\cite{Aad:1698966}.}
  \label{fig:inner_dec}
\end{figure}

%% ============================== Pixel detector ====================================
\textbf{Pixel detector}

The pixel detector is the innermost part of ATLAS tracking system.
With finest granularity of materials, it has the best spatial resolution and 3-dimensional space-point measurement in inner detector.
ATLAS Pixel Detector for the LHC run-2 is composed of 4 layers of barrel pixel detector and two end-caps with three pixel disks each, as shown in figure~\ref{fig:inner_pixel}.
There are three outer layers that originally installed for run-1 and one additional layer called Insertable B-Layer (IBL) that newly constructed in run-2~\cite{Mullier:2016}.
Now the 4-layer pixel detector has very good reconstruction of primary and secondary vertices, which is even crucial for long-lived particles like $\tau$-lepton and b-quark.
\begin{figure}[!htb]
  \centering
  \includegraphics[width=1.0\textwidth]{figures/Detector/inner_pixel.png}
  \caption{Schematic diagram of the ATLAS 4-Layer Pixel Detector.}
  \label{fig:inner_pixel}
\end{figure}

%% ================================ SCT ===========================================
\textbf{Semiconductor Tracker}

The Semiconductor Tracker (SCT) is the middle component of the inner detector that outside the pixel detector.
It has similar function as pixel detector but with long and narrow strips instead of small pixels, which makes a much larger coverage than pixel detector.
The SCT consists of 4088 modules, it contains four concentric layers in barrel (2112 modules) and nine disks in each of the two end-caps (1976 modules) as shown in figure~\ref{fig:inner_sct}.
And it measures particles over a large area with 6.3 million readout channels and a total area of 61 square meters.
The SCT is the most critical part of the inner detector for 2D track hit reconstruction.
In barrel, the hit precision is 17 $\mu$m in the \textit{r}-$\phi$ coordinate and 580 $\mu$m in \textit{z} coordinate.
In end-caps, it have accuracies of 17 $\mu$m in the \textit{z}-$\phi$ coordinate and 580 $\mu$m in \textit{r} coordinate.
\begin{figure}[!htb]
  \centering
  \includegraphics[width=0.8\textwidth]{figures/Detector/inner_SCT.png}
  \caption{SCT (a) barrel module and (b) end-cap\cite{Sultan:phdthesis}.}
  \label{fig:inner_sct}
\end{figure}

%% ================================= TRT ======================================
\textbf{Transition radiation tracker}

The transition radiation tracker (TRT)\cite{TRT_2008} is the outermost part of inner detector.
It has a very different design with the two previously sub-detectors. It's composed of thin-walled drift tubes called straw, also in three parts: a barrel and two end-cap regions.
There are 73 barrel layers and 224 end-cap layers (112 in each) with 372000 straws in total, and about 351000 readout channels for TRT.
The TRT provides better \textit{z} resolution but much worse \textit{r}-$\phi$ resolution (about 130 $\mu$m) compared to the pixel detector and SCT per straw.
But the straw hits still make significant contributions to momentum measurement, since its lower precision per point (compared to silicon) can be compensated by the large number of measurements and long track length.

%\subsection{Inner dector}
\subsection{Calorimeters}

The calorimeters are designed to measure the energy from particles by absorbing them.
They are located outside the solenoidal magnet that surrounds the inner detector.
The ATLAS calorimeters are comprised of a number of sampling calorimeters with full $\phi$-symmetry and the pseudorapidity range of $|\eta|<4.9$.
Figure~\ref{fig:calo_dec} shows the layout of the ATLAS calorimeter system.
As mentioned in overview section, there are two basic calorimeter systems: an inner electromagnetic (EM) calorimeter and an outer hadronic calorimeter.
The EM calorimeter is designed for precise measurements for electrons and photons, so that with fine granularity;
while the hardronic one with relative coarser granularity but satisfied the physics requirements for jets reconstructions and $E_{T}^{miss}$ measurements.
Two different sampling techniques are used, the EM calorimeter is purely based on liquid-argon (LAr) technology, hardronic calorimeter use both LAr and scintillating tiles calorimeters. 
More details are described as belows.
\begin{figure}[!htb]
  \centering
  \includegraphics[width=0.8\textwidth]{figures/Detector/calo_layout.png}
  \caption{Cut-away view of the ATLAS calorimeters. The LAr calorimeters are seen inside the scintillator- based Tile hadronic calorimeters\cite{Buchanan:2008}.}
  \label{fig:calo_dec}
\end{figure}

%% ================================ electromagnetic calorimeter ===================
\textbf{Liquid Argon calorimeter}

%Liquid-argon calorimeter is used for EM calorimeter in barrel and end-caps regions and for hardronic calorimeter in end-caps.
The Liquid Argon sampling calorimeter technique with "accordion-shaped" electrodes is used for all electromagnetic calorimetry covering the pseudorapidity range of $|\eta|<3.2$;
and for hadronic calorimetry from $|\eta| = 1.4$ up to the acceptance limit $|\eta| = 4.9$\cite{CERN-LHCC-96-041}.
Figure~\ref{fig:calo_lar} shows the shape of a barrel module as accordion geometry.
For barrel EM calorimeter, the absorbing material is lead-liquid argon, while the hadronic end-cap calorimeter use copper plates as the absorbing material.
In addition, the forward calorimeter is splited into three parts, an EM sector in which copper is used as absorbing material and two hadronic sectors using tungsten ouside the EM sector.
\begin{figure}[!htb]
  \centering
  \includegraphics[width=0.6\textwidth]{figures/Detector/calo_lar.png}
  \caption{Diagram of a LAr EM calorimeter barrel module\cite{Sanchez:2010}.}
  \label{fig:calo_lar}
\end{figure}


%\subsection{Calorimeters}
\subsection{Muon spectrometer}

Muon spectrometer~\cite{CERN-LHCC-97-022} is the outermost part of the ATLAS detector with an extremely large tracking system.
It measures a large range of muon momentum, and the accuracy is about 3\% at 100 GeV and 10\% at 1 TeV.
The muon spectrometer comprises three main parts: a magnetic field produced by three toroidal magnets;
a set of chambers measuring the tracks of muons with high spatial precision; and triggering chambers with accurate time-resolution. 
Figure~\ref{fig:muon_dec} shows the schematic of ATLAS muon spectrometer that consists of four types of muon chambers 
(\textit{MDT, CSC, RPC, TGC}) as well as the magnet systems (barrel and end-cap toroid).
\begin{figure}[!htb]
  \centering
  \includegraphics[width=0.8\textwidth]{figures/Detector/muon_all.png}
  \caption{Cut-away view of the ATLAS muon spectrometer\cite{Sliwa:2013oua}.}
  \label{fig:muon_dec}
\end{figure}

More details of four chambers are given as below:
\begin{itemize}
	\item \textbf{Monitored Drift Tubes (MDT)}. MDTs provide the precise momentum measurement with the $|\eta|$ range up to 2.7, except in the innermost end-cap layer where the coverage is limited to $|\eta| < 2.0$. The chambers comprises three or four layers of drift tubes, with a diameter of 29.970 mm, operated with Ar/CO2 gas (93/7) at 3 bar. The average resolution can reach 80 $\mu$m per tube and 30 $\mu$m per chamber.
	\item \textbf{Cathode strip chambers (CSC)}. CSCs are used in the forward region of $2 < |\eta| < 2.7$ in the innermost tracking layers, due to their good time resolution and high rate capability. The CSCs are multi-wire proportional chambers (MWPC) with the cathode planes segmented into strips in orthogonal directions, which allows both coordinates to be measured from the induced-charge distribution. The resolution of a chamber is about 40 $\mu$m for bending plane and 5 mm for the transverse plane.
	\item \textbf{Resistive plate chambers (RPC)}. The RPCs serves as fast triggers in the barrel region of $|\eta| < 1.05$ due to the high rate capability and good spatial and time resolution. It is a gaseous parallel electrode-plate detector without any wires. There are three concentric cylindrical layers around the beam axis, as three trigger stations. Each stations consists of two independent layers to measure the transverse coordinates of $\eta$ and $\phi$.
	\item \textbf{Thin gap chambers (TGC)}. TGCs are used as trigger system for the end-cap region of $1.5 < |\eta| < 2.4$, and works based on the same principle as multi-wire proportional chambers. In addition, they can also provide the second azimuthal coordinate to complement the measurement of MDT in bending direction.
\end{itemize}

%\subsection{Muon spectrometer}
\subsection{Trigger system}

Trigger system in ATLAS is a very essential component, which is responsible for deciding whether to keep a given collision event for later study or not.
In LHC run-2, higher energy, luminosity and pile-up lead to an large increase of event rate by up to a factor of five, which cause to a even larger challenge and more strict requirement of trigger system.

The trigger system in run-2 is comprised of a hardware-based first level trigger (Level-1) and a software-based high level trigger (HLT) \cite{Ruiz-Martinez:2133909}.
As depicted in figure~\ref{fig:trig_syst}, in Level-1, the inputs from coarse granularity calorimeter and muon detector information together with some other subsystems are sent to the Central Trigger Processor to determine Regions-of-Interest (RoIs) in the detector. 
\begin{figure}[!htb]
  \centering
  \includegraphics[width=0.9\textwidth]{figures/Detector/tdaq-run2-schematic2017.png}
  \caption{Schematic diagram of the ATLAS trigger and data acquisition system in Run-2.}
  \label{fig:trig_syst}
\end{figure}
The events rate can be reduced by Level-1 triggers from 30 MHz to 100 kHz. 
After that, the RoI information from Level-1 is sent to HLT, in which more sophisticated selection algorithms are run for regional reconstruction.
The HLT reduces the rate from Level-1 of 100 kHz to about 1 kHz on average.
At the end, the events that accepted by HLT are transfered to local storage at experimental site for offline reconstruction.
Details about Level-1 and HLT trigger systems will be described as belows.

\textbf{Level-1 trigger}

Substantial upgrades have been delivered in ATLAS Level-1 trigger system for Run-2 data taking.
The upgrades took place in both hardware and detector readout, allows the trigger rate increasing from 70 kHz (run-1) to 100 kHz (run-2).
As mentioned above, there are two major parts of Level-1 triggers, which include Level-1 calorimeter (L1calo) trigger and Level-1 muon (L1mu) trigger.

Level-1 Calorimeter trigger uses the reduced granularity information from the electromagnetic and hadronic calorimeters to search for electrons, photons, taus and jets and missing transverse energy ($E_{T}^{miss}$).
It can identify an Region-of-Interest (RoI) as a $2 \times 2$ trigger tower cluster in the EM calorimeter as shown in figure~\ref{fig:trig_tower}, 
and $4 \times 4$ or $8 \times 8$ trigger tower for Jet RoIs.
\begin{figure}[!htb]
  \centering
  \includegraphics[width=0.6\textwidth]{figures/Detector/trig_tower.png}
  \caption{An examples of L1 calorimeter trigger tower for electron and photon triggers\cite{Pasztor:2063746}.}
  \label{fig:trig_tower}
\end{figure}
One important upgrade is that, the new FPGA-based (field-programmable gate array) Multi-Chip Modules are used to replace the ASICs (application-specific integrated circuits) included in the modules used in run-1,
which allows the use of auto-correlation filters to suppress pile-up.

The Level-1 Muon trigger system includes one barrel section (RPC) and two enf-cap section (TGC), which provides fast trigger signals from the muon detectors for the Level-1 trigger decision.
By requiring a coincidence with hits from the innermost muon chambers, it can reduce the $L1_MU15$ rate by about 50\% in the region of $1.3 < |\eta| < 1.9$ while only loss around 2\% signal efficiency.
In addition, the coverage is extended by around 4\% due to installing new chambers in the feet region of the muon detector.

\textbf{High Level Trigger}

The ATLAS trigger system separated the Level-2 and Event Filter computer clusters in run-1, but for run-2, they have been merged into a single HLT event processing.
The new arrangement helps to reduce the complexity and duplication of algorithm, which leads to a more flexible high level trigger system.
During the long-shutdown between LHC run-1 and run-2, lots of re-optimizations have been done for trigger reconstruction algorithms as well as the offline analysis selections,
which can improve the efficiency by more than a factor of two in some cases like in hadronic tau triggers.
For some triggers, the HLT processing performed within RoIs can also allows to agregrate from RoIs to single objects. 
This improvement reduces the CPU processing for events with overlapping RoIs, and the average output rate has been increased from 400 Hz to 1 kHz.
The HLT reconstruction algorithm can be divided into fast and precision online reconstruction steps. 
As depicted by figure~\ref{fig:trig_alg}, the initial fast reconstruction helps to reduce the event rate early, and be seeded into precision reconstruction.
Then the final online precision reconstruction is improved and uses offline-like algorithms as much as possible.
In particular, multivariate analysis techniques (based on machine learning) have been introduce online in many aspects.
\begin{figure}[!htb]
  \centering
  \includegraphics[width=0.5\textwidth]{figures/Detector/trig_alg.png}
  \caption{ The HLT trigger algorithm sequence\cite{Pasztor:2063746}.}
  \label{fig:trig_alg}
\end{figure}

%\subsection{Trigger system}

\chapter{Simulation and Object Reconstruction for the ATLAS Experiment}

In current LHC pp collison, bunches of protons collide every 25 nanoseconds (ns), which gives a large challenge to event reconstruction and selections.
To predict and model each process, the Monte Carlo simulations of physics events are essential for high-energy physics experiments.
This section will briefly discuss the event simulation and reconstruction programs based on the ATLAS software framework. 

\section{Event sumilation}
\subsection{Simulation framework}

The ATLAS simulation program is integrated into the ATLAS software framework called \textit{Athena}\cite{atlas:athena},
which uses Python as an object-oriented scripting and interpreter language to configure and load C++ algorithms and objects.
Figure~\ref{fig:frame_overview} shows the overview of ATLAS simulation data flow\cite{Aad:2010ah}.
In the diagrams, the square-cornered boxes represents algorithms and applications to be run and round-cornered boxes denote data objects.
\begin{figure}[!htb]
  \centering
  \includegraphics[width=1.0\textwidth]{figures/Simulation/outline_atalsSimulation_v2.png}
  \caption{The flow of the ATLAS simulation software.}
  \label{fig:frame_overview}
\end{figure}

First of all, events are produced by MC generators in standard HepMC format and then read into the simulation.
During the simulation, particles are propagated through the full ATLAS detector whose configurations can be set by users via GEANT4 toolkit.
The energies deposited in the sensitive regions of the detector are recorded as \textit{hits}, which contains the total energy deposition,
position, and time, and are written to a simulation hit file.
In the meatime, the events in "truth" format are also recorded to contain the history of the interactions from the generator, including incoming and outgoing particles.
Simulated Data Objects (SDOs) are created from truth, which are maps between hits in sensitive portions of the detector and truth information of particles in simulation.
The files are then sent to digitization, which firstly constructs "digits" inputs to the read out drivers (RODs) in the detector electronics.
Then the ROD is emulated, from which one can get a Raw Data Object (RDO) file used for reconstruction.

In conclusion, there are three main parts of framework: \textit{Generation}, \textit{Simulation} and \textit{Reconstruction}. 
More details are given below.

\textbf{Event generation}

As shown in figure~\ref{fig:mc_event_structure}\cite{Hoche:2014rga}, at hardon colliders, multiple scattering and rescattering effects arise, which must be simulated by Monte Carlo (MC) event generators to reflect the full complexity of those event structures.
Several MC event generators can be used to generate events originally in HepMC format.
The events can be filtered at generation time with some certain requirements (eg. decay channel or missing energy above a certain threshold).
The generator is responsible for any prompt decays (e.g. W or Z bosons) but stores any "stable" particle expected to propagate through a part of the detector. 
During the generation steps, any interactions with detector are ignored and only immediate decays are considered.

There are several MC generators that have been widely-used with general purpose, which include Sherpa\cite{Gleisberg_2009}, Herwig++\cite{Bahr2008}, PowhegBox\cite{Nason:2004rx}, MC@NLO\cite{Frixione_2002} and Pythia8\cite{Sjostrand:2007gs}.

\begin{figure}[!htb]
  \centering
  \includegraphics[width=0.5\textwidth]{figures/Simulation/mc_event_structure.png}
  \caption{Sketch of a hardon-hardon collision simulated by MC event generator. The red blob in center denotes the hard collision, surrounded by tree-like structures representing Bremsstrahlung which is simulated by Parton Showers. The purple blob stands for a secondary hard scattering event. The light green blobs indecate the parton-to-hardon transitions and the dark green blobs represents hardon decays. The yellow lines are soft photon radiations.}
  \label{fig:mc_event_structure}
\end{figure}

%\subsection{Simulation framework}

\chapter{Observation of electroweak ZZ production and measurement of SM ZZ cross section in the \llll final state using pp collisions data collected by ATLAS detector from 2015$~$2018}

\section{Introduction}

After the discovery of Higgs boson~\cite{20121, 201230}, the examine of electroweak symmetry breaking (EWSB) becomes a main focus at the LHC.
In addition to measuring the properties of Higgs boson directly, the vector boson scattering (VBS) process is another key avenue to probe EWSB~\cite{Lee:1977yc, Chanowitz:1985hj, Szleper:2014xxa}.
As introduced in section~\ref{symbreaking}, in Standard Model (SM), the Higgs boson acts as ``moderator" to unitarize high-energy longitudinal VBS amplitudes at the ~\tev~ scale.
Therefore, studying high-energy behaviours of VBS is crucial to understand the mechanism of EWSB.

Since there was no VBS process was observed prior to the LHC era, LHC provides an unexceptionable opportunity to study them due to its unprecedented high energy and luminosity.
At LHC, the VBS process is typically studied through the measurements of electroweak (EW) production of two vector bosons radiated from quark-quark initial state, 
plus a pair hadronic jets with high energy in the back and forward regions (denoted as EW-$VVjj$).
The quantum chromodynamics (QCD) production of $VVjj$ contains two QCD vertices at the lowest order (denoted as QCD-$VVjj$) is an ireducible background to the search of EW-$VVjj$ production.
The features of EW-$VVjj$ production include a large invariant mass of jet pair ($m_{jj}$) and a significant separation of rapidity between two jets ($\Delta y_{jj}$).
Figure~\ref{fig:vbszz_diagrams} presents some typical Feynman diagrams of EW- and QCD- $ZZjj$ processes.
\begin{figure}[!htbp]
\begin{center}
\includegraphics[width=0.22\textwidth]{figures/VBSZZ/diagram-EWZZjj-Schn-Higgs.pdf}
\includegraphics[width=0.22\textwidth]{figures/VBSZZ/diagram-EWZZjj-Tchn-Higgs.pdf}
\includegraphics[width=0.22\textwidth]{figures/VBSZZ/diagram-EWZZjj-QGC.pdf}
\includegraphics[width=0.22\textwidth]{figures/VBSZZ/diagram-EWZZjj-TGC.pdf}\\
\includegraphics[width=0.22\textwidth]{figures/VBSZZ/diagram-QCDZZjj-qq.pdf}
\includegraphics[width=0.22\textwidth]{figures/VBSZZ/diagram-QCDZZjj-qg.pdf}
\includegraphics[width=0.22\textwidth]{figures/VBSZZ/diagram-QCDZZjj-gg.pdf}
\includegraphics[width=0.22\textwidth]{figures/VBSZZ/diagram-QCDZZjj-box.pdf}\\
\end{center}
\caption{Typical diagrams for the production of $ZZjj$, including the relevant EW VBS diagrams (first row) and QCD diagrams (second row).}
\label{fig:vbszz_diagrams}
\end{figure}

The first evidence of the EW-$VVjj$ process was seen in same-sign $WW$ channel (EW-$W^{\pm}W^{\pm}jj$) by ATLAS collaboration with 20.3~\ifb~8~\tev~data~\cite{PhysRevLett.113.141803},
in which a 3.6$\sigma$ excess was observed in data over the background-only prediction.
In the LHC run-2, the observation (with > 5 $\sigma$ statistical significance) of EW-$W^{\pm}W^{\pm}jj$ process has been reported in both ATLAS and CMS collaboration with 36~\ifb~13~\tev~data\cite{PhysRevLett.123.161801, Sirunyan:2017ret}.
In $WZ$ channel (EW-$WZjj$), an observation with 5.3 $\sigma$ excess was also reported by the ATLAS collaboration recently~\cite{2019469}.
As for the EW-$ZZjj$ production, it was searched by CMS using 35.9~\ifb~13~\tev~data but no evidence was found\cite{2017682}.
The EW production in $ZZ$ final state (EW-$ZZjj$) is typically rare, whose fiducial cross section has an order of \textit{O}(0.1)~\ifb in the final state where both $Z$ bosons decay leptonically.
But in the meantime, $ZZ \rightarrow \llll$ process offers a extremely clean channel than all the others. So with more data collected in the LHC, the observation of EW-$ZZjj$ becomes possible.

This section presents the first observation of EW-$ZZjj$ production decaying to four charged leptons with two jets (\lllljj) by ATLAS collaboration using the complete set of the LHC run-2 data with 139~\ifb luminosity.
It is a new milestone in the study of EWSB at the LHC, and completes the last missing part of observation of weak boson scattering for massive bosons.
In the meatime, the measurement of fiducial cross-sections for SM $ZZ$ production including both EW and QCD processes is also reported.
The $ZZjj$ production involving intermediate $\tau$-leptons from $Z$ decays is considered as signal but has a negligible contribution to the selected events.
Reducible backgrounds give minor contributions in the \lllljj channel are also studied.
To further separate the EW signal and the QCD background, multivariate discriminant (MD) is trained using event kinematic information from simulated samples. 
The MD distribution is then used as discriminant in statistical fit to evaluate the signal strength of EW process.

\section{Data and MC samples}

\subsection{Data samples}
\label{sec:vbszz_data}

The datasets for this analysis include the full run-2 pp collision data collected by the ATLAS experiment during the years from 2015 to 2018.
Data event is only used if it passed the latest Good Run List (GRL) released by the Data Quality group from ATLAS experiment,
corresponding to an integrated luminosity of $139.0~\pm~2.4$~\ifb.


%{\footnotesize
%\begin{verbatim}
%data15_13TeV.periodAllYear_DetStatus-v89-pro21-02_Unknown_PHYS_StandardGRL_All_Good_25ns.xml
%data16_13TeV.periodAllYear_DetStatus-v89-pro21-01_DQDefects-00-02-04_PHYS_StandardGRL_All_Good_25ns.xml
%data17_13TeV.periodAllYear_DetStatus-v99-pro22-01_Unknown_PHYS_StandardGRL_All_Good_25ns_Triggerno17e33prim.xml
%data18_13TeV.periodAllYear_DetStatus-v102-pro22-04_Unknown_PHYS_StandardGRL_All_Good_25ns_Triggerno17e33prim.xml
%\end{verbatim}
%}

%% =======================================================
\subsection{MC simulations}
\label{sec:mc}

The EW-$ZZjj$ production is modelled using \MGMCatNLO~2.6.1~\cite{Alwall:2014hca} with the matrix elements (ME) calculated in the leading-order (LO) approximation
in perturbative QCD (pQCD) and with the NNPDF2.3LO~\cite{Ball:2012cx} parton distribution functions (PDF).
The VBF Higgs process is also included.

The QCD-$ZZjj$ production is modelled using \textsc{Sherpa} 2.2.2~\cite{Gleisberg:2008ta} with the NNPDF3.0NNLO~\cite{ball2015parton} PDF,
where events with up to one (three) outgoing partons are generated at NLO (LO) in pQCD.
The production of $ZZjj$ from the gluon-gluon initial state with a four-fermion loop or with an exchange of the Higgs boson has an order of $\alpha_{S}^{4}$ in QCD,
and is not included in the \textsc{Sherpa} simulation.
A separate $gg$ induced $ZZ$ + 2jets sample is modelled using \textsc{Sherpa} 2.2.2 with the NNPDF3.0NNLO PDF
and with an additional 1.7 k-factor~\cite{PhysRevD.92.094028} being applied.
Then the interference between EW- and QCD-$ZZjj$ is modelled with \MGMCatNLO~2.6.1 calculated at LO.

The diboson productions from QCD $WW \rightarrow \ell \nu qq$ as well as QCD and EW $WZ \rightarrow \ell\ell qq$ are modelled using \textsc{Sherpa} 2.2.2 with the NNPDF3.0NNLO PDF.
The productions of semileptonic decays ($WW \rightarrow \ell\nu qq$ and $WZ \rightarrow qq\ell\ell$) are modelled using \textsc{Powheg-Box}~v2~\cite{Frixione:2007nw} with the CT10 PDF~\cite{Lai:2010vv}.
%Other diboson processes are not included due to negligible contributions.
The triboson production is modelled using \textsc{Sherpa} 2.2.2 with the NNPDF3.0NNLO PDF.

For top-quark pair (\ttbar) production, the \textsc{Powheg-Box}~v2 is used with the CT10 PDF.
The single top-quark production in $t$-channel, $s$-channel and $Wt$-channel are simulated using the \textsc{Powheg-Box}~v1 event generator~\cite{Alioli:2009je,Frederix:2012dh,Re:2010bp}.
The productions of \ttbar~in association with vector boson(s) ($ttV$) are modelled with \MGMCatNLO~2.3.3 for $ttW$ and $ttZ$ with $Z \rightarrow \nu\nu/qq$ decays,
with \textsc{Sherpa} 2.2.1 for $ttZ$ where the $Z$ decays to dilepton,
and with \MGMCatNLO~2.2.2 for $ttWW$ respectively.

The \Zjet processes are modelled using \textsc{Sherpa} 2.2.1 with the NNPDF3.0NNLO PDF, 
in which the ME is calculated for up to two partons with next-to-leading-order (NLO) accuracy in pQCD and up to four partons with LO accuracy.

For all the samples except those from \textsc{Sherpa}, 
the parton showering is modelled with \textsc{Pythia8}~\cite{Sjostrand:2007gs} using the NNPDF2.3~\cite{Ball:2012cx} PDF set,
and the A14 set of tuned parameters~\cite{ATL-PHYS-PUB-2014-021}.
While for \textsc{Sherpa} samples, the parton showering is simulated within the programme.

All simulated events are processed with detector response simulation based on \textsc{Geant4} described in section~\ref{sec:simulation_framework}.
In addition, simulated inelastic pp collisions are overlaid to model additional pp collision in the same and neighbouring bunch crossings (pile-up),
and reweighted to match the pile-up conditions in data.
Moreover, all simulated events are processed using the same reconstruction algorithms as data.
And the leptons and jets reconstruction, energy scale and resolution, and the leptons identification, isolation, trigger efficiencies for simulated events,
as described in section~\ref{sec:reconstruction}, are all corrected to match the data measurements.

\section{Objects and Event selection}

\subsection{Objects defination}

The selection of analysis relies on the defination of multiple objects: \textit{electrons}, \textit{Muons}, and \textit{jets}.
Details of definations for each objects are described as below:

\textbf{Muon:} 
To increase the acceptance range in reco-level for \llll jj channel, all four types of muons 
(CB, ST, CT, ME muons, described in section~\ref{sec:muon}) are used.
The identified muons are then required to pass $p_{T} > 7 \gev$ and $|\eta| < 2.7$,
and satisfy the \textit{Loose} identification criterion (see defination in sec~\ref{sec:muon}).
The impact parameter cuts are further applied to suppress the contribution from cosmic muons and non-prompt muons,
with the value of: $|d_{0}/\sigma(d_{0})| < 3.0$ and $|z_{0} sin\theta| < 0.5 mm$.
In order to avoid muons associated with jets, all muons are required to be isolated that pass \textit{FixedCutLoose} isolation criteria.

\textbf{Electron:} 
As described in section~\ref{sec:electron}, electrons are reconstructed from energy deposits in the EM calorimeter matched to a track in the inner detector.
The electron candidates must satisfy the \textit{Loose} criterion valuing by the likelihood-based (LH) method.
And electrons are required to have $p_{T} > 7 \gev$ and $|\eta| < 2.47$.
Moreover, the impact parameter requirements of $|d_{0}/\sigma(d_{0})| < 5.0$ and $|z_{0} sin\theta| < 0.5 mm$ are applied.

\section{Background estimation}
\label{sec:background}

Table~\ref{tab:yield_prefit} summarizes the background yields for $ZZjj \rightarrow \lllljj$ channel in 139~\ifb.
Uncertainties on the predictions include both statistical and systematic components.
"Others" includes minor contributions from non-$ZZ$ processes including \Zjet, top-quark, triboson and $ttV$ processes.
Detail of estimation for each source will described below.

\begin{table}[!htbp]
\sisetup{
table-number-alignment = center,
table-align-uncertainty=true
}
\begin{center}
   \begin{tabular}{
   c
   S[table-format = 3.1]@{$\,\pm\,$}
   S[table-format = 2.1]
   }
   \hline
   Process                 & \multicolumn{2}{c}{\lllljj}       \\
   \hline
   EW-$ZZjj$               &  20.6 &  2.5  \\
   QCD-\qqZZ               &  77   & 25    \\
   QCD-\ggZZ               &  13.1 &  4.4  \\
   Others                  &   3.2 &  2.1  \\
   \hline
   Total                   & 114   & 26    \\
   \hline
   Data                    &  \multicolumn{2}{l}{127}           \\
   \hline
   \end{tabular}
\end{center}
\caption{
Observed data and expected signal and background yields in 139~\ifb{} of luminosity.
Minor backgrounds are summed together as `Others'.
Uncertainties on the predictions include both statistical and systematic components.
}
\label{tab:yield_prefit}
\end{table}

\subsection{QCD backgrounds}

The QCD-$ZZjj$ production, which include both $qq$ and $gg$ induced processes, is an irreducible background in the search of EW-$ZZjj$ production.
A QCD-enriched control region (CR) is defined to constrain the contribution by reverting either the $\mjj$ or $\dyjj$ requirements:\\
	$\mjj <$ 300~\GeV{} or $\dyjj <$ 2 \\
Then the normalization factor of QCD-$ZZjj$ process is included into statistical fit as a float parameter to properly treat the uncertainty correlations between SR and CR, 
while the shapes are taken from MC simulation.
Table~\ref{tab:yield_qcdcr} shows the event yields of each background components in this CR.
Uncertainties are statistical one only.
\begin{table}[!htbp]
\sisetup{
table-number-alignment = center,
table-align-uncertainty=true
}
\begin{center}
   \begin{tabular}{
   c
   S[table-format = 3.1]@{$\,\pm\,$}
   S[table-format = 2.1]
   }
   \hline
   Process                 & \multicolumn{2}{c}{\lllljj}       \\
   \hline
   EW-$ZZjj$               &   3.9 &  0    \\
   QCD-$ZZjj$              & 136.9 &  0.6  \\
   QCD-$ggZZjj$            &  16.8 &  0.1  \\
   Diboson                 &   0.3 &  0.1  \\
   Triboson                &   1.6 &  0.1  \\
   \Zjet                   &   \multicolumn{2}{l}{0}    \\
   \ttbar                  &   \multicolumn{2}{l}{0}    \\
   \hline
   Total                   & 159.5 &  0.62 \\
   \hline
   Data                    &  \multicolumn{2}{l}{152}           \\
   \hline
   \end{tabular}
\end{center}
\caption{
Observed data and expected signal and background yields in 139~\ifb{} of luminosity.
Diboson background in table includes all the other diboson processes discussed in section~\ref{sec:mc}, except those with four-lepton final state.
Uncertainties include only MC statistic.
No events from \Zjet and \ttbar MC samples pass the selection, and are indicated as 0 in the table.
}
\label{tab:yield_qcdcr}
\end{table}
The distributions of 4l and di-jet invariant mass in QCD CR are shown in figure~\ref{fig:qcdcr_prefit}.
\begin{figure}[!htb]
  \centering
  \includegraphics[width=0.42\textwidth]{figures/VBSZZ/QCDCR/MZZ_4l_QCD_CR.pdf}
  \includegraphics[width=0.42\textwidth]{figures/VBSZZ/QCDCR/MJJ_4l_QCD_CR_fullSyst.pdf}
  \caption{Pre-fit $\mzz$ and $\mjj$ distribution in QCD-enriched CR.}
  \label{fig:qcdcr_prefit}
\end{figure}

\subsection{Fake backgrounds}

Backgrounds from \Zjet, top-quark and $WZ$ processes are estimated by data-driven method.
These events usually contain two or three leptons from Z/W decays, together with heavy-flavor jets or misidentified components of jets reconstructed as leptons called "fake leptons".
A \textit{fake factor} method is used to estimate this backgrounds, in which the lepton misidentification is measured in data regions 
with enhanced contributions from \Zjet and top-quark processes:
\begin{enumerate}
	\item Define a dedicated background dominant region to derive the fake factor for this background. 
The \textit{fake factor} is defined as:
\begin{equation}
	\mathcal{F} = \mathcal{N}_{good} / \mathcal{N}_{pool}
\end{equation}
where $\mathcal{N}_{good}$ refers to the number of good leptons passing all SR selection, while $\mathcal{N}_{pool}$ denotes the number of poor leptons passing most SR selection but fail one certain requirement.
	\item Define a \lllljj fake control region, where one or two leptons pass \textit{poor} requirement while all the other leptons are required to have SR selection.
	\item The number of fake events are calculated as:
\begin{equation}
	\mathcal{N}_{fake} = \left( N_{gggp} - N_{gggp}^{ZZ} \right) \times \mathcal{F} - \left( N_{ggpp} - N_{ggpp}^{ZZ} \right) \times \mathcal{F}^{2}
\end{equation}
with the subtraction of $ZZ$ contribution, and the double counting between ($N_{gggp}$ and $N_{ggpp}$).
\end{enumerate}

For the definition of \textit{poor} leptons:
The poor electrons are defined as failing "FixedCutLoose" isolation requirement or "LooseLH" electron ID requirement but satisfying "VeryLooseLH" WP.
The poor muons are required to fail the "FixedCutLoose" isolation requirement or invert the impact parameter cut to be $3 < d_{0}/\sigma(d_{0}) < 10$.
The dedicated \Zjet and \ttbar dominant regions are defined to calculate the fake factor respectively in the following subsections.
%For other minor fake contributions like $WZ$, $W + jets$ and $W^{+}W^{-} + jets$ without additional dedicated CR, the estimations are included into 

\subsubsection{Fake factor for \Zjet}

Fake factor for \Zjet background is calculated in \Zjet-enriched region, where events with one SFOS lepton pair around $Z$ mass associated with two jets are selected.
The value of fake factor is driven from data, and as a function of $p_{T}$ and $\eta$ as shown in figure~\ref{fig:fake_zjet_el} for electrons and figure~\ref{fig:fake_zjet_mu} for muons.
During calculation, the contributions from non-\Zjet backgrounds (\ttbar, $ZZ$, $WZ$) have been subtracted from data.
The values calculated from \Zjet MC directly are also shown in plots for comparison.
\begin{figure}[!htb]
  \centering
  \includegraphics[width=0.42\textwidth]{figures/VBSZZ/fakebkg/Electron_2Dff_ptfakeFactorAddElectron_etapt_pavgy.pdf}
  \includegraphics[width=0.42\textwidth]{figures/VBSZZ/fakebkg/Electron_2Dff_etafakeFactorAddElectron_etapt_pavgx.pdf}
  \caption{Fake factor for $\Zjet$ background, constructed with additional electron, as a function of $p_{T}$ (left) and $\eta$ (right).}
  \label{fig:fake_zjet_el}
\end{figure}

\begin{figure}[!htb]
  \centering
  \includegraphics[width=0.42\textwidth]{figures/VBSZZ/fakebkg/Muon_2Dff_ptfakeFactorAddMuon_etapt_pavgy.pdf}
  \includegraphics[width=0.42\textwidth]{figures/VBSZZ/fakebkg/Muon_2Dff_etafakeFactorAddMuon_etapt_pavgx.pdf}
  \caption{Fake factor for $\Zjet$ background, constructed with additional muon, as a function of $p_{T}$ (left) and $\eta$ (right).}
  \label{fig:fake_zjet_mu}
\end{figure}

\subsubsection{Fake factor for \ttbar}

The fake factor for \ttbar are calculated in \ttbar dominanted region by selecting one $e\mu$-pair with additional two jets.
For events with three leptons, $m_{T}^{W} <$ 60~\gev cut is applied to reject the constribution from \ttbar + W events.
The $m_{T}^{W}$ is defined as below:
\begin{equation}
	m_{T}^{W} = \sqrt{ 2p_{T}^{l_{3}} E_{T}^{miss} \left[1-cos\left(\Delta\phi\left(p_{T}^{l_{3}}, E_{T}^{miss}\right)\right)\right] }
\end{equation}
In addition, at least one b-jet is required to enhance the top component.
The fake factors of \ttbar calculated from data as the function of $p_{T}$ and $\eta$ are shown in figure~\ref{fig:fake_tt_el} for electrons and ~\ref{fig:fake_tt_mu} for muons.
The non-\ttbar contributions, which include \Zjet, $ZZ$ and $WZ$, are subtracted from data.
\begin{figure}[!htb]
  \centering
  \includegraphics[width=0.42\textwidth]{figures/VBSZZ/fakebkg/Electron_2Dff_ptttbarFakeFactorAddElectron_etapt_pavgy.pdf}
  \includegraphics[width=0.42\textwidth]{figures/VBSZZ/fakebkg/Electron_2Dff_etattbarFakeFactorAddElectron_etapt_pavgx.pdf}
  \caption{Fake factor for $\ttbar$ background, constructed with additional electron, as a function of $p_{T}$ (left) and $\eta$ (right).}
  \label{fig:fake_tt_el}
\end{figure}

\begin{figure}[!htb]
  \centering
  \includegraphics[width=0.42\textwidth]{figures/VBSZZ/fakebkg/Muon_2Dff_ptttbarFakeFactorAddMuon_etapt_pavgy.pdf}
  \includegraphics[width=0.42\textwidth]{figures/VBSZZ/fakebkg/Muon_2Dff_etattbarFakeFactorAddMuon_etapt_pavgx.pdf}
  \caption{Fake factor for $\ttbar$ background, constructed with additional muon, as a function of $p_{T}$ (left) and $\eta$ (right).}
  \label{fig:fake_tt_mu}
\end{figure}

\subsubsection{Combination}

The fake factors calculated from each dedicated region are then combined together according to their contributions in fake control region described previously.
Figure~\ref{fig:fake_mjj} shows the \mjj distribution with data and major fake backgrounds in three different 4l channels.
\begin{figure}[!htb]
  \centering
  \includegraphics[width=0.32\textwidth]{figures/VBSZZ/fakebkg/15161718_mva_dijet_mass_zjet_ttbar_ratio_electrons_mva_dijet_mass.pdf}
  \includegraphics[width=0.32\textwidth]{figures/VBSZZ/fakebkg/15161718_mva_dijet_mass_zjet_ttbar_ratio_mix_mva_dijet_mass.pdf}
  \includegraphics[width=0.32\textwidth]{figures/VBSZZ/fakebkg/15161718_mva_dijet_mass_zjet_ttbar_ratio_muons_mva_dijet_mass.pdf}
  \caption{$\mjj$ distributions in fake control region in 4e (left), 2e2$\mu$ (middle) and 4$\mu$ (right) channel.
The ratios between $\Zjet$ and $\ttbar$ ($\Zjet / \ttbar$) in each individual channel are: 2.59, 0.95, 0.74.}
  \label{fig:fake_mjj}
\end{figure}

\subsubsection{Systematics of fake estimation and results}
\label{sec:fake_syst}

The systematics of fake factor method can be measured by varying the parameters and selection requirements in fake factor calculation.
In addition, due to the very limited data statistic in \llll channel, to be more conservative, 
the difference between data measurement and MC simulation are also considered as another systematics component.
In detail, the sources of systematics that have been included are listed as below:
\begin{itemize}
	\item Variations of isolation cut for the poor lepton definition up and down scaled by a factor of two.
	\item Variations of the yields of those subtracted MC in fake control region scaled by 30\% up and down.
	\item The difference of fake factors between driven from data and from MC simulation.
	\item The difference of fake factors when changing to one bin measurement (instead of $p_{T}$ or $\eta$ dependent).
	\item The statistical uncertainties on fake factor in fake control region.
\end{itemize}

Table~\ref{tab:fake_uncer} summarizes the contribution of fake backgrounds in signal region under different systematic conditions mentioned above as well as the nominal one.
Uncertainties of each value in table are statistical one.
\begin{table}[h]
    \centering
    \resizebox{1.0\columnwidth}{!}{
        \begin{tabular}{lrrrr}
        channel & 4e &   2e2$\mu$ &    4$\mu$ &   inclusive \\ 
	\hline
        Nominal estimate                  & 0.678$\pm$ 0.652 & 1.023$\pm$ 0.740 & 0.566$\pm$ 0.240 & 2.268$\pm$ 1.015 \\
        $F$ stat. uncertainty varied down & 0.698$\pm$ 0.622 & 0.872$\pm$ 0.652 & 0.509$\pm$ 0.214 & 2.079$\pm$ 0.926 \\
        $F$ stat. uncertainty varied up   & 0.657$\pm$ 0.685 & 1.173$\pm$ 0.840 & 0.622$\pm$ 0.267 & 2.452$\pm$ 1.116 \\
        One bin $F$                       & 0.653$\pm$ 0.590 & 0.594$\pm$ 0.558 & 0.646$\pm$ 0.313 & 1.892$\pm$ 0.870 \\
        MC $F$                            & 0.534$\pm$ 0.471 & 1.415$\pm$ 0.993 & 0.439$\pm$ 0.184 & 2.389$\pm$ 1.114 \\
        Isolation varied down             & 0.938$\pm$ 0.686 & 0.552$\pm$ 0.466 & 0.215$\pm$ 0.107 & 1.704$\pm$ 0.837 \\
        Isolation varied up               & 0.723$\pm$ 0.646 & 1.104$\pm$ 0.739 & 0.559$\pm$ 0.237 & 2.386$\pm$ 1.010 \\
        MC corr. varied down              & 0.697$\pm$ 0.695 & 1.048$\pm$ 0.811 & 0.832$\pm$ 0.385 & 2.577$\pm$ 1.136 \\
        MC corr. varied up                & 0.660$\pm$ 0.614 & 0.984$\pm$ 0.687 & 0.316$\pm$ 0.159 & 1.961$\pm$ 0.935 \\
        \hline
        \end{tabular}
        }
    \caption{
    Fake background estimations in the SR. For the nominal value the 2D fake factor together with the \Zjet and \ttbar combination applied.
    The other lines show the estimations with different uncertainty variations.
    }
    \label{tab:fake_uncer}
\end{table}


\section{Systematics}

The analysis performances both the statistical fit to MD distribution to extract the EW-ZZjj contributions
and the cross section measurements in fiducial volume.
Therefore, theoretical and experimental uncertainties may affect the predictions background yields and shapes, 
correction factors from detector-level to particle-level measurement, as well as the $ZZjj$ MD shapes and so on.
Moreover, the statistical uncertainties of simulated samples are also taken into account.
And due to the extramely low cross section of \llll channel, the analysis is still data statistic dominanted.
This section will described the measurement of both theoretical and experimental systematics for $ZZjj$ productions.
The systematics for fake backgrounds have been elaborated in section~\ref{sec:fake_syst}.

\subsection{Theoretical systematics}



\section{Measurement of fiducial cross section}
\label{sec:xsec}

The fiducial cross section for inclusive $ZZjj$ production, which includes both EW and QCD components, is then measured.
The definition of fiducial volume, which is used for cross section measurement, follows closely to the detector-level selection
but use physics objects in particle-level, which are reconstructed in simulation from stable final-state particles,
prior to their interactions with the detector.

For electrons and muons, QED final-state radiation is for the most part recovered 
by adding the four-momenta of surrounding photons that are not originating from hadrons and within an angular distance $\Delta R < 0.1$
to the lepton four-momentum, called lepton ``dressing" in truth level.
Particle-level jets are built with anti-$k_{T}$ algorithm with radius parameter $R = 0.4$ using all final-state particles except leptons and neutrinos as inputs.
Comparing to the events selection in detector-level in section~\ref{sec:vbszz_selection},
in particle-level, the selected dilepton pair mass required is relaxed to be within 60 to 120~\gev~ for the reasons of reducing the migration effect,
as well as being more compatibility with CMS publication~\cite{2017682}.
All other kinematic selections are the same as the definition in detector-level.

\subsection{Calculation of C-factor}
\label{sec:cf}

C-factor is defined as the ratio between the number of selected events in detector-level and the number of particle-level events in fiducial volume (FV):
\begin{equation}
	\mathcal{C} = \frac{N_{detector-level}}{N_{FV.}}
\end{equation}
The value of C-factor for each $ZZjj$ process are calculated from each individual simulation samples as listed in table~\ref{tab:xs_cf} together with their systematics.
\begin{table}[H]
\begin{center}
   \begin{tabular}{|c|c|c|c|c|}
   \hline
   Process          & $\mathcal{C}$ & $\Delta$C (stats.) & $\Delta$C (sys.)        & $\Delta$C (theo.)       \\
   \hline
   EWK $ZZjj$         & 0.663         & $\pm$0.002       & $\pm^{0.032}_{0.031}$ & NA                    \\
   \hline
   QCD \qqZZ        & 0.702         & $\pm$0.003       & $\pm^{0.061}_{0.051}$ & $\pm^{0.015}_{0.018}$ \\
   \hline
   QCD \ggZZ        & 0.741         & $\pm$0.021       & $\pm^{0.143}_{0.072}$ & $\pm{0.002}$          \\
   \hline
\end{tabular}
\end{center}
\caption{C Factor of different $ZZjj$ processes.}
\label{tab:xs_cf}
\end{table}

Then the $\mathcal{C}$ from different processes are combined together to be used as inputs for cross section calculation:
\begin{equation}
	\mathcal{C} = \Sigma_{i} \frac{N_{FV.}^{i}}{\Sigma_{j} N_{FV.}^{j}} \times \mathcal{C}_{i} = 0.699\pm0.003(stats.)\pm^{0.011}_{0.013}(theo.)\pm0.028(exp.)
\end{equation}
The stats. refers to the statistical uncertainty from MC simulation statistics.
The theo. and exp. denote the theoretical and experimental uncertainties described in section~\ref{sec:systematics}.

\subsection{Result of fiducial cross section}

The cross section in fiducial volume is computed as:
\begin{equation}\label{eq:xs}
	\sigma^{FV.} = \frac{N_{data} - N_{bkg}}{\mathcal{C} \times Lumi}
\end{equation}
where $N_{data}$ and $N_{bkg}$ denote the number of events selected from detector-level selection from data and sum of backgrounds,
and $\mathcal{C}$ is the C-factor calculated above, Lumi represents the integrated luminosity of data from 2015 to 2018 of 139~\ifb.
%As shown in table~\ref{tab:yield_prefit}, in inclusive measurement, only the ``Others" represents background, 
%processes of EW-$ZZjj$, QCD-\qqZZ and QCD-\ggZZ are signals.
Table~\ref{tab:xs} shows the fiducial cross section for \llll final state measured from equation~\ref{eq:xs}, 
as well as the predicted cross section measured from MC simulation directly.

\begin{table}[!htbp]
\begin{center}
\scalebox{0.90}{
\begin{tabular}{ c | c}
\hline
\hline\noalign{\smallskip}
Measured fiducial $\sigma$ [fb] & Predicted fiducial $\sigma$ [fb] \\
\noalign{\smallskip}\hline\noalign{\smallskip}
$1.27 \pm 0.12(\mathrm{stat}) \pm 0.02(\mathrm{theo}) \pm 0.07(\mathrm{exp}) \pm 0.01(\mathrm{bkg}) \pm 0.03(\mathrm{lumi})$ & $1.14 \pm 0.04(\mathrm{stat}) \pm 0.20(\mathrm{theo})$ \\
\noalign{\smallskip}\hline
\hline
\end{tabular}}
\end{center}
\caption{
Measured and predicted fiducial cross-sections in \lllljj final-state.
Uncertainties due to different sources are presented.
}
\label{tab:xs}
\end{table}
The measured cross section has a total uncertainty of 11\%, and is found to be compatible with SM prediction.
This measurement is still dominant by data statistic.

\section{Search for EW-$ZZjj$}

\subsection{MD discriminant}

To further separate the EW-$ZZjj$ component from QCD-$ZZjj$, a MD based on Gradient Boosted Decision Tree (BDT) algorithm~\cite{Coadou_BDT} 
is trained with simulated events via TMVA framework\cite{Speckmayer_2010}.
Training is performed between EW (signal) and QCD (background) processes.
Twelve event kinematic variables sensitive to the characteristics of the EW signal are used as input features in training. 
Table~\ref{tab:bdt_features} lists those input variables with the order of their importance in BDT response provided by TMVA tool.
One can see the jet-related information provides larger sensitivity.
Then the MD distributions in both SR and QCD CR region are used for statistical fit.

\begin{table}[h]
\begin{center}
\renewcommand\arraystretch{1.8}
\begin{spacing}{0.8}
\begin{tabular}{p{1cm}|p{2cm}|p{8cm}}
\hline
\hline
Rank & Variables                    & Description 			\\ \hline
1    & \mjj                         & Dijet invariant mass 		\\ \hline
2    & $p_{T}^{j1}$                 & \pT of the leading jet		\\ \hline
3    & $p_{T}^{j2}$                 & \pT of the sub-leading jet	\\ \hline
4    & $\frac{p_{T}(ZZjj)}{H_{T}(ZZjj)}$  & \pT of the $ZZjj$ system divided by the scalar \pT sum of Z bosons and two jets \\ \hline
5    & $y_{j1} \times y_{j2}$       & Product of jet rapidities		\\ \hline
6    & \dyjj                        & Rapidity difference between two jets \\ \hline
7    & $Y_{Z2}^{*}$                 & Rapidity of the second Z boson \\ \hline
8    & $Y_{Z1}^{*}$                 & Rapidity of the Z boson reconstructed from the lepton pair with the mass closer to the Z boson mass \\ \hline
9    & $p_{T}^{ZZ}$                 & \pT of 4l system \\ \hline
10   & $m_{ZZ}$                     & Invariant mass of 4l system \\ \hline
11   & $p_{T}^{Z1}$                 & \pT of the Z boson reconstructed from the lepton pair with the mass closer to the Z boson mass \\ \hline
12   & $p_{T}^{\ell 3}$             & \pT of the third lepton \\ \hline
\hline
\hline
\end{tabular}
\end{spacing}
\caption{Input features for the training of MD. }
\label{tab:bdt_features}
\end{center}
\end{table}

\iffalse
\subsection{Statistical procedure}

To examine the compatibility between data and the signal-plus-background hypothesis, 
a test statistic is based on the profile likelihood ratio method.
%The likelihood function is the product of all the Poisson probability density functions built in individual MD bins given as:
The binned likelihood function is given as
\begin{equation}
	\mathcal{L}(\mu,\sigma) = \prod_{i}^\mathrm{bins} \mathcal{L}_{\mathrm{poiss}}(N_{\mathrm{data}}\,|\,\mu s(\theta)+b(\theta))_{i} \times \mathcal{L}_{\text{gauss}}(\theta)_{i}
\end{equation}
where the Poisson term presents the statistical fluctuations of the data 
and a Gaussian term models the pdf of auxiliary measurement to constrain the systematics.
$\mu$ denotes the signal strength of EW-$ZZjj$ process, computed as the ratio between measured (expected) cross section to the SM prediction.
$\theta$ presents the nuisance parameter, which is the set of parameters that parameterize the effect of systematic uncertainties described in section~\ref{sec:systematics} following the Gaussian distribution.
$N_{data}$ is the number of selected data events, while the $s(\theta)$ is the expected signal yield and $b(\theta)$ is the expected background yield as the function of nuisance parameters.

The test statistic $q_{\mu}$ is defined as:
\begin{equation}
	q_\mu = -2 \ln \left( \dfrac{\mathcal{L}(\mu,\hat{\hat{\theta}}_{\mu})}{\mathcal{L}(\hat{\mu},\hat{\theta})} \right)
\end{equation}
in which $\mathcal{L}(\hat{\mu},\hat{\theta})$ is the unconditional likelihood with respect to both $\mu$ and $\theta$,
and $\mathcal{L}(\mu,\hat{\hat{\theta}}_{\mu})$ is the conditional likelihood for a constant $\mu$.
Signal-like data distributions are more likely to have a low test-statistic ($q_\mu$ close to 0) 
while the contributions of background-like data have a larger $q_\mu$.
Under the background-only hypothesis, the compatibility of the observed (Asimov) data with the prediction 
is calculated to obtain the observed (expected) significance respectively.
\fi

\subsection{Fitting procedure}


A profile likelihood fit, as described in chapter~\ref{sec:statisticalfit}, is performed on MD discriminant to extract the EW-$ZZjj$ signal from backgrounds.
%The fit combined the observed and expected measurements from both \llll and \llvv channel from EW-ZZ production to gain more statistic.
The binning of MD distributions in SR is optimized to maximize the sensitivity for detecting EW signal.
The normalization of QCD-$ZZjj$ production ($\mu_{QCD}^{llll}$) in \llll channel is determined by data from simultaneously fit in SR and QCD CR as described in section~\ref{sec:background}.
The signal strength of EW-$ZZjj$ production ($\mu_{EW}$) is taken as parameter of interest and floated in the fit.
The effects of the uncertainties related to normalizations and shapes described previously in section~\ref{sec:systematics} 
of background processes in the MD distribution are all taken into account.

In most case, a common nuisance parameter is used for each source of systematic in all bins and all categories.
The statistical uncertainties for simulated samples are uncorrelated among all bins, and the background uncertainties only applied to their corresponding backgrounds.
%For combination between two channels, the theoretical uncertainties between \llll and \llvv are uncorrelated due to different fiducial volumes definition.
Furthermore, to be more conservative, the generator modelling uncertainty for QCD-$ZZjj$ production mentioned in section~\ref{sec:systematics}
is separated to be two nuisance parameters in low and high MD region.

\subsection{Result of fit}

The statistical fit is performed both in individual \llll channel, as well as the combination between \llll and \llvv channel to gain more statistic.
The results of statistical fit in \llll final state, 
and the one in combined channel are presented in table~\ref{tab:fit_result}.
The \llvv analysis will not be described in this dissertation, but more details can refer to~\cite{ATLAS:2019vrv}.
To drive expected results, the observed data is used for QCD CR to extract normalization factor of QCD component ($\mu_{QCD}^{\llll}$),
while in SR, asimov data built from background prediction and signal model with SM assumed cross section is used.

\begin{table}[!htbp]
\begin{center}
\begin{tabular}{c|c|c|c}
\hline
                 & $\mu_{\mathrm{EW}}$ &  $\mu^{\llll}_{\mathrm{QCD}}$   &  Significance Obs. (Exp.) \\
\hline
\llll          & $1.54 \pm 0.42$     &  $0.95 \pm 0.22$                  &  5.48 (3.90) $\sigma$     \\
\hline
Combination of \llll and \llvv         & $1.35 \pm 0.34$     &  $0.96 \pm 0.22$                  &  5.52 (4.30) $\sigma$     \\
\hline
\end{tabular}
\end{center}
\caption{
Observed \muEW and \muQCD, as well as the observed and expected significance from the individual \llll channel.% and the combined fits.
The full set of systematic uncertainties are included.
}
\label{tab:fit_result}
\end{table}

As a conclusion, in \llll channel, the background-only hypothesis is rejected at 5.5$\sigma$ (3.9$\sigma$) for observed (expected) data,
which leads to the observation of EW-$ZZjj$ production.

Figure~\ref{fig:fit_MD} shows the post-fit MD distributions for \llll events after performing a combined fit in SR (left) and QCD CR (right).
The EW-$ZZjj$ cross section measured in \llll channel is extracted to be $0.94 \pm 0.26$~fb.
\begin{figure}[!htbp]
\begin{center}
\includegraphics[width=0.42\textwidth]{figures/VBSZZ/fit/BDT_4l_SR_postFit.pdf}
\includegraphics[width=0.42\textwidth]{figures/VBSZZ/fit/BDT_4l_QCD_CR_postFit.pdf}
\end{center}
\caption{Observed and post-fit expected multivariate discriminant distributions after the statistical fit in the \llll SR (left) and QCD CR (right).
        The error bands include the experimental and theoretical uncertainties,
        as well as the uncertainties in \muEW and \muQCD.
        The error bars on the data points show the statistical uncertainty on data.
        }
\label{fig:fit_MD}
\end{figure}

Figure~\ref{fig:scaled_mjj} shows the $\mjj$ distribution in SR (left) and QCD CR (right),
where the normalization of EW and QCD processes are scaled according to their observed value in table~\ref{tab:fit_result}.
High $\mjj$ region is more sensitive for EW-$ZZjj$ events detection from this figure.
Figure~\ref{fig:scaled_mzz} shows the spectrum of invariant mass of \llll system ($\mzz$) in SR
also with the normalization of EW and QCD processes scaled.

\begin{figure}[!htbp]
\begin{center}
\includegraphics[width=0.32\textwidth]{figures/VBSZZ/fit/MJJ_4l_SR.pdf}
\includegraphics[width=0.32\textwidth]{figures/VBSZZ/fit/MJJ_4l_QCD_CR.pdf}
\end{center}
\caption{Observed and post-fit expected \mjj distributions in SR (left) and QCD CR (right).
        The error bands include the expected experimental and theoretical uncertainties.
        The error bars on the data points show the statistical uncertainty.
        The contributions from the QCD and EW production of $ZZjj$ events are scaled by 0.96 and 1.35, respectively,
        corresponding to the observed normalization factors in the statistical fit.
        The last bin includes the overflow events.
        }
\label{fig:scaled_mjj}
\end{figure}

\begin{figure}[!htbp]
\begin{center}
\includegraphics[width=0.4\textwidth]{figures/VBSZZ/fit/MZZ_4l_SR.pdf}
\end{center}
\caption{Observed and post-fit expected $\mzz$ spectrum in SR.
        The error bands include the expected experimental and theoretical uncertainties.
        The error bars on the data points show the statistical uncertainty.
        The contributions from the QCD and EW production of $ZZjj$ events are scaled by 0.96 and 1.35, respectively,
        corresponding to the observed normalization factors in the statistical fit.
        The last bin includes the overflow events.
        }
\label{fig:scaled_mzz}
\end{figure}

Figure~\ref{fig:event_display} is the display of one event candidate of EW-$ZZjj$ production in $2e2\mu$ final state with two jets in forward and backward region.
\begin{figure}[!htbp]
\begin{center}
\includegraphics[width=1.0\textwidth]{figures/VBSZZ/fit/resize_340368_454611985_v3.pdf}
\end{center}
\caption{Display of an event candidate of EW-$ZZjj$ production in $2e2\mu$ channel in last MD bin (0.875 < MD < 1.0).
         The invariant mass of the di-jet (four-lepton) system is 2228 (605)~\gev. }
\label{fig:event_display}
\end{figure}

\section{Prospect study of EW-$ZZjj$ production in HL-LHC}

\subsection{Introduction}
The High-Luminosity Large Hadron Collider (HL-LHC) project aims to increase luminosity by a factor of 10 beyond the LHC’s design value 
to increase the potential for discoveries after 2025.
The designed luminosity will reach 3000~\ifb with the centre-of-mass energy of 14~\tev.

As introduced in previous sections, with full run-2 data of 139~\ifb collected by ATLAS detector at LHC, 
the EW-$ZZjj$ production is the last channel of observation for VBS processes with massive boson 
due to its very low cross section in $ZZ$ decay.
So we expect that this channel will benefit significantly from the increased luminosity at the high-luminosity LHC (HL-HLC),
and can be studied in great details for this known mechanism.

In this section, a prospect study has been performed for EW-$ZZjj$ production at the HL-LHC in the llll channel with the ATLAS detector.
The study uses 3000~\ifb of simulated pp collisions at a centre-of-mass energy of 14~\tev~that is expected to be recorded by the ATLAS detector at HL-LHC.
All simulated events are produced at particle-level, 
and the detector effects of lepton and jet reconstruction and identification are estimated by corrections, 
assuming the mean number of interactions per bunch crossing ($<\mu>$) of 200.

\subsection{The ATLAS detector at HL-LHC}

As the expectation of HL-LHC, the new Inner Tracker (ITk)\cite{Collaboration:2285585}
 will extend the acceptance capability of ATLAS detector to pseudorapidity ($|\eta|$) up to 4.0.
By including a forward muon trigger, the upgraded Muon Spectrometer\cite{Collaboration:2285580} is also expected to provide 
muon identification capabilities to $|\eta|$ up to 4.0.
The new high granularity timing detector (HGTD)\cite{Collaboration:2623663} designed to mitigate the pile-up (PU) effects 
is also foreseen in the forward region of $2.4 < |\eta| < 4.0$.
The expected performance of the upgraded ATLAS detector at HL-LHC has been studied as reported in Ref.\cite{ATL-PHYS-PUB-2016-026}.

\subsection{Simulation}

The analysis is performed using particle-level events for simulated samples.
The samples are generated at $\sqrt{s} = 14~\tev$~and with a fast simulation based on setting for ATLAS detector at HL-LHC.
The signal in this analysis is EW-$ZZjj$ process, while only the dominanted ireducible background of QCD-$ZZjj$ is considered.
Both signal and background are generated using \textsc{Sherpa} with NNPDF3.0NNLO PDF set.
The signal sample is generated with two jets at Matrix Element (ME) level.
The background is qq-initial process, in which events with up to one (three) outgoing partons are generated at NLO (LO) in perturbative QCD.
As a quick study, other irreducible backgrounds like fake backgrounds from \Zjet and top-quark processes, as well as Diboson without 4l final-state and Triboson processes are not included into this analysis.
Furthermore, for hard scattering events, the pile-up collisions are set with a mean value of 200 interactions per bunch crossing.
In studies, signal and background yields are then scaled to 3000~\ifb for HL-LHC.

\subsection{Event selection}

The analysis selection follows closely to the one in ATLAS run-2 analysis as described in section~\ref{sec:selection}.
Here are some changes according to the expectation of HL-LHC scenario for ATLAS detector:
\begin{itemize}
	\item Extend the lepton indentification in forward with $|\eta| <$ 4.0
	\item Pile-up (PU) jet suppression is applied with a PU rejection factor of 50 for all PU jets in the region of $|\eta| <$ 3.8, based on the expected ATLAS detector performance at the HL-LHC.
	\item The jets are reuiqred to have $\pT >$ 30 (70)~\gev~in the $|\eta| <$ 3.8 ($3.8 < |\eta| < 4.5)$ region.
	\item For two selected jets, tight the $\mjj$ requirement to > 600~\gev, and require $\Delta \eta_{jj} >$ 2.
\end{itemize}
In addition, a fiducial volume, which is used to study the expected precision of the cross-section measurements,
 is defined at particle-level with the same kinematic requirements listed above.

Table~\ref{tab:event_yield} summarized the number of selected signal and background events normalized to 3000~\ifb.
In addition to the \textit{baseline} selection listed above, two alternative selections are also studies:
\begin{itemize}
	\item Reduce the lepton $\eta$ region to 2.7, to uncerstand the effect due to forward lepton reconstruction and identification with the upgraded ATLAS detector.
	\item Only apply the PU jet suppression with region $|\eta| < 2.4$, to measure the improvement of \textit{baseline} by extending the rejection range of PU jets at the HL-LHC.
\end{itemize}
\begin{table}[htbp]
  \small
  \centering
  \begin{tabular}{|c|c|c|c|}
    \hline
    Selection & $N_{\mathrm{EW-ZZjj}}$ & $N_{\mathrm{QCD-ZZjj}}$ & $N_{\mathrm{EW-ZZjj}}$ / $\sqrt{N_{\mathrm{QCD-ZZjj}}}$ \\
    \hline
    Baseline                                 & 432 $\pm$ 21 & 1402 $\pm$ 37   & 11.54 $\pm$ 0.58 \\
    \hline
    Leptons with $|\eta|<$ 2.7               & 373 $\pm$ 19 & 1058 $\pm$ 33   & 11.46 $\pm$ 0.62 \\
    \hline
    PU jet suppression only in $|\eta|<$ 2.4 & 536 $\pm$ 23 & 15470 $\pm$ 120 &  4.31 $\pm$ 0.19  \\
    \hline
  \end{tabular}
  \caption{
    Comparison of event yields for signal ($N_{\mathrm{EW-ZZjj}}$) and background ($N_{\mathrm{QCD-ZZjj}}$) processes, 
    and expected significance of EW-$ZZjj$ processes,
    normalized to 3000~\ifb{} data at 14~\TeV{},
    with baseline and alternative selections.
    Uncertainties in the table refer to expected data statistical uncertainty at 14~\TeV{} with 3000~\ifb{}.
  }
  \label{tab:event_yield}
\end{table}
From this table, one can see the extended track coverage increases the \lllljj events by 15$~$30\%, by improving the lepton efficiency.
But the significance of searching for EW-$ZZjj$ process does not improve so much due to the large increment of background events.

Figure~\ref{fig:kine} shows the kinematic distributions of dijet invariant mass (\mjj), the ZZ invariant mass (\mzz) and 
the $\phi$ separation of two Z bosons ($|\Delta\phi(ZZ)|$) as well as the centrality of the ZZ system.
The ZZ centrality is defined as:
\begin{equation}
  ZZ~\text{centrality} = \frac{|y_{ZZ} - (y_{j1} + y_{j2})/2|}{|y_{j1} - y_{j2}|}
\end{equation}
To measure the event yield, the top panel shows the stack distribution for EW- and QCD-$ZZjj$ processes,
while bottom panel is the ratio between EW and QCD.
\begin{figure}[!htbp]
\centering
\subfloat[]{
\includegraphics[width=0.42\textwidth]{figures/VBSZZ/hllhc/TagJJM_final_noshape_0_ratio.pdf}
}
\subfloat[]{
\includegraphics[width=0.42\textwidth]{figures/VBSZZ/hllhc/MVV_noshape_0_ratio.pdf}
}
\\
\subfloat[]{
\includegraphics[width=0.42\textwidth]{figures/VBSZZ/hllhc/dPhiZZ_noshape_0_ratio.pdf}
}
\subfloat[]{
\includegraphics[width=0.42\textwidth]{figures/VBSZZ/hllhc/ZZCen_noshape_0_ratio.pdf}
}
\caption{
Detector-level distributions of EW- and QCD-$ZZjj$ processes with selected events in defined phase space at 14~\tev~of 
(a) \mjj,
(b) \mzz,
(c) $|\Delta\phi(ZZ)|$,
(d) ZZ centrality,
normalized to 3000~\ifb{}.
}
\label{fig:kine}
\end{figure}


\subsection{Systematics}

According to studies in section~\ref{sec:systematics}, the dominanted systematic in \llll channel is from theoretical systematic for QCD-$ZZjj$ background process.
Different sizes of systematics have been tested, at 5, 10 and 30\% level on background modeling.
The 5\% uncertainty is an optimal estimation when there is enough data events from QCD-enrich control region at HL-LHC to constrain the theoretical modeling on QCD-$ZZjj$ process.
The 30\% one is a conservative estimation, in which the uncertainties are calculated from different PDF sets and QCD renormalization and factorization
scales, following recommendation from PDF4LHC mentioned in section~\ref{sec:systematics}.

For experimental sources, the jet systematic has been checked following the setting provided by HL-LHC in Ref.\cite{ATL-PHYS-PUB-2016-026},
and the uncertainties are within 5\% level, which is smaller than run-2 measurement at 10\%.
Figure~\ref{fig:jet_uncer} depicts the up and down variations for jet uncertainty provided by HL-LHC performance tool as function of dijet invariant mass.
\begin{figure}
  \centering
  \includegraphics[width=0.42\textwidth]{figures/VBSZZ/hllhc/Uncer_baseline_TagJJM_ewk_linear.pdf}
  \includegraphics[width=0.42\textwidth]{figures/VBSZZ/hllhc/Uncer_baseline_TagJJM_qcd_linear.pdf}
  \caption{Jet variations on $\mjj$ distribution for EW-ZZjj (left) and QCD-ZZjj (right) processes
           with luminosity of 3000~\ifb at 14~\tev.
	   \textit{Upgrade Performance Function} is used to extract the uncertainties with \textit{baseline} setting.}
  \label{fig:jet_uncer}
\end{figure}
Therefore, a conservative 5\% uncertainty is used as experimental uncertainty.

The final results rely largely on the uncertainties, especially the theoretical uncertainties on QCD-$ZZjj$ production.
So results with different uncertainty conditions will be shown:
\begin{itemize}
	\item The case with statistical uncertainty of simulated samples only.
	\item The case with statistical and experimental uncertainties (5\%)
	\item The case with statistical, experimental and additional theoretical uncertainties at 5\%, 10\% and 30\% respectively.
\end{itemize}
Three different sources of uncertainties are treated as uncorrelated.

\subsection{Results}

In this analysis, the expected significance of EW-$ZZjj$ production is calculated as:
\begin{equation}
  \text{Significance} = \frac{S}{\sqrt{\sigma(B)_{stat.}^2 + \sigma(B)_{syst.}^2}},
\end{equation}
where $S$ presents the number of selected signal events,
and $\sigma(B)_{stat.}$ and $\sigma(B)_{syst.}$ denote the statistical and systematic (exp. + theo.) uncertainties from background processes.
The statistical is computed from expected yield at 3000~\ifb.

Base on baseline selection of $\mjj > 600~\gev$, a additional scan over different \mjj cuts are performed with a step of 50~\gev
for luminosity of 3000~\ifb under different systematic conditions, as shown in figure~\ref{fig:mjj_scan}.
\begin{figure}[!htbp]
\centering
\includegraphics[width=0.48\textwidth]{figures/VBSZZ/hllhc/significance_noshape_0_noratio.pdf}
\caption{
The expected significance of EW-ZZjj processes as a function of different \mjj cut with 3000~\ifb,
under conditions of different sizes of theoretical uncertainties on the QCD-ZZjj background modelling.
The statistical uncertainty is estimated from expected data yield at 14~\TeV{} with 3000~\ifb.
Different uncertainties are summed up quadratically.
}
\label{fig:mjj_scan}
\end{figure}

In addition, the expected differential cross section of EW-$ZZjj$ process is measured in the defined phase space at 14~\tev,
as a function of \mzz and \mjj, shown in figure~\ref{fig:xs_mjj_mzz}.
\begin{figure}[!htbp]
\centering
\includegraphics[width=0.48\textwidth]{figures/VBSZZ/hllhc/TagJJM_final_all_linear.pdf}
\includegraphics[width=0.48\textwidth]{figures/VBSZZ/hllhc/MZZ_all_linear.pdf}
\caption{
The projected differential cross-sections at 14~\TeV{} for the EW-ZZjj processes as a function of \mjj (left) and \mzz (right).
The top panel shows measurement with statistical only case,
where statistical uncertainty is estimated from expected data yield at 14~\TeV{} with 3000~\ifb.
The bottom panel shows impact of different sizes of systematic uncertainties.
}
\label{fig:xs_mjj_mzz}
\end{figure}
The expected differential cross sections are calculated as:
\begin{equation}
\begin{split}
  \sigma = \frac{N_{pseudo-data} - N_{QCD-ZZjj}}{L*C_{EW-ZZjj}}\\
  C_{EW-ZZjj} = \frac{N_{EW-ZZjj}^{det.}}{N_{EW-ZZjj}^{part.}}
\end{split}
\end{equation}
where $N_{pseudo-data}$ denotes the expected number of data events with 3000~\ifb{} luminosity at 14~\tev,
and $N_{QCD-ZZjj}$ and $N_{EW-ZZjj}$ are the number of predicted events of QCD-ZZjj and EW-ZZjj processes in particle-level.
The $C_{EW-ZZjj}$ factor represents the detector efficiency for EW-ZZjj processes introduced in section~\ref{sec:cf}.
The interference between EW- and QCD-ZZjj processes is ignored due to its minor contribution.

The number of expected integrated cross section as well as its uncertainty under different uncertainty conditions are shown in table~\ref{tab:xsec}
in 3000~\ifb luminosity at 14~\tev.
The statistical uncertainty is at 10\% level when with such large luminosity.
The result is dominanted by systematics and can reach 100\% level when theoretical modeling uncertainty is 30\% for QCD-$ZZjj$ processees.
\begin{table}[htbp]
  \small
  \centering
  \begin{tabular}{c|c|c|c|c|c|c}
    \hline
     & Cross section [fb] & Stat. only & Plus exp. & Plus $5\%$ theo. & Plus $10\%$ theo. & Plus $30\%$ theo. \\
    \hline
    EW-ZZjj & 0.21 & $\pm0.02$ & $\pm0.04$ & $\pm0.05$ & $\pm 0.08$ & $\pm 0.21$ \\
    \hline
  \end{tabular}
  \caption{
  Summary of expected cross-section measurements with different theoretical uncertainties.
  The statistical uncertainty is estimated from expected data yield at 14~\TeV{} with 3000~\ifb.
  Different uncertainties are summed up quadratically.
  }
  \label{tab:xsec}
\end{table}



\clearpage
\section{Conclusion}

The fiducial cross section for inclusive $ZZjj$ production is measured in this section, with a total relative uncertainty of 11\% for the \lllljj final state,
and found to be compatible with the SM prediction.
The observation of electroweak production of two jets in association with a Z-boson pair decay to \llll final state 
using 139~\ifb of 13~\tev pp collision data collected by ATLAS experiment at LHC is presented in this section.
The search for electroweak production of two jets in association with a Z-boson pair is based on multivariate discriminants (MD) to enhance the separation between the signal and backgrounds.
In \llll final state, the background-only hypothesis is rejected with an observed (expected) significance of 5.5 (3.9)~$\sigma$,
which gives the first observation of electroweak production in $ZZjj$ channel.

In addition, the prospect study for the EW-$ZZjj$ production at the HL-LHC in the \llll channel, using 3000~\ifb simulated pp collision at a centre-of-mass energy of 14~\tev has been presented.
The precision of the expected measurements of the integrated and differential cross sections as a function of dijet or 4l invariant mass are shown.
Under the assumption of theoretical uncertainty being constraint at 5\% level for the QCD-$ZZjj$ processes, 
the observation of the EW-$ZZjj$ process can be reached with a significance of 7~$\sigma$.


\chapter{Search for heavy $ZZ$ resonances in the \llll final state using pp collisions data collected by ATLAS detector from 2015$~$2018}

\section{Introduction}

A new particle was discovered by the ATLAS and CMS Collaborations at the LHC~\cite{20121, 201230} in 2012.
Both experiments have confirmed that the properties including spin, couplings and parity of this new particle are consistent with 
Higgs boson predicted in the Standard Model (SM), which is an important milestone in understanding of the mechanism of EWSB.
Nevertheless, the possibility that this newly discovered particle is just a part of the extended Higgs sector
as predicted by various extensions in the SM cannot be ruled out.
Many models predicted the existence of new heavy resonances decaying into dibosons, 
such as a heavy spin-0 neutral Higgs boson~\cite{PhysRevD.36.3463}
and the two-Higgs-doublet models (2HDM)~\cite{BRANCO20121}, 
as well as the spin-2 Kaluza–Klein (KK) excitations of the graviton ($G_{KK}$)~\cite{DAVOUDIASL200043}.

Though with smaller branching ratio compared to semileptonic or fully hadronic decay channels, the \llll final state has its unique sensitivity in mass range smaller than 1~\tev~region 
due to its good mass resolution and relative smaller experimental and theoretical systematics.
This section presents the search for heavy resonance decaying into a pair of $Z$ bosons to the \llll final state, in which $\ell$ denotes to either an electron or a muon~\cite{Aaboud:2017rel, Aad:2020fpj}. 
Several signal hypothesises are considered.
The first hypothesis is a heavy Higgs boson (spin-0 resonance) under the narrow-width approximation (NWA).
Then as several theoretical models prefer non-negligible natural widths, the models under large-width approximation (LWA), 
assuming widths of 1\%, 5\%, 10\% and 15\% of the resonance mass, are also studied.
In addition, the graviton excitations (spin-2 resonance) under the Randall–Sundrum model are also searched.
It is assumed that the heavy resonance is produced predominantly via the gluon-gluon Fusion (ggF) and the Vector Boson Fusion (VBF) productions, 
but with the unknown ratio of two production rates.
So the results are separated for ggF and VBF production modes.
To gain more sensitivity, the \llll events are classified into  ggF- and VBF-enriched categories.
Moreover, for the NWA model, the categorizations are studied under both cut-based and multivariate (MVA) -based methods, the details of categorization are shown in following sections.

The search uses the four-lepton invariant mass in the range of 200~\gev~to~2000~\gev~for signal hypothesis of spin-0 resonance under the NWA model,
and from 400~\gev~to 2000~\gev~for the one under the LWA models.
And the spin-2 graviton signals are searched in the mass range from 600~\gev~ to 2000~\gev.
The data collected by ATLAS detector at the LHC from 2015 to 2018 at the centre-of-mass energy of 13~\tev~is used.
In case of no excess, upper limits on the production rate of different signal hypotheses are computed from statistical fits to $m_{4l}$ distribution.


\section{Data and MC samples}

\subsection{Data samples}

The data used in this analysis are collected by ATLAS detector at the centre-of-mass energy of 13~\tev~ during the years of 2015 to 2018.
Only events passing the latest Good Run List (GRL) released by the Data Quality group from ATLAS experiment %as listed in section~\ref{sec:vbszz_data} 
corresponding to an integrated luminosity of $139.0~\pm~2.4$~\ifb~ are used.
Table~\ref{tab:data_info} listed the recorded integrated luminosity, average and peak pile-up of each year's data.
\begin{table}[htbp]
  \centering
  \caption{Summary of the recorded integrated luminosity (lumi), average and peak pile-up (PU) of data from 2015 to 2018.}
  \label{tab:data_info}
  \begin{spacing}{0.65}
  \begin{tabular}{ccccc}
    \toprule
    Year & recorded integrated lumi  & lumi after GRL & average PU & peak PU  \\
    \midrule
    2015 & 3.86~\ifb & \multirow{2}{*}{36.2~\ifb} & 13.4 & 28.1 \\
    2016 & 35.6~\ifb & & 25.1 & 52.2 \\
    2017 & 46.9~\ifb & 44.3~\ifb & 37.8 & 79.8 \\
    2018 & 60.6~\ifb & 58.5~\ifb & 36.1 & 88.6 \\
    \bottomrule
  \end{tabular}
  \end{spacing}
\end{table}

\subsection{Background MC simulations}

Background processes considered in this analysis include $ZZ$ (\qqZZ, \ggZZ), triboson ($WWZ$, $WZZ$, $ZZZ$), \Zjet and top-quark (\ttbar, ttV) processes.

The QCD \qqZZ process is modelled using \textsc{Sherpa} 2.2.2~\cite{Gleisberg:2008ta} with the NNPDF3.0NNLO~\cite{ball2015parton} PDF,
where events with up to one (three) outgoing partons are generated at NLO (LO) in pQCD.
The production of $ZZ$ from the gluon-gluon initial state with a four-fermion loop or with an exchange of the Higgs boson, which has an order of $\alpha_{S}^{4}$ in QCD, is not included in this \textsc{Sherpa} simulation.
So a separate $gg$ induced $ZZ$ sample including the continuum background, the SM Higgs boson, and the interference contribution 
is modelled using \textsc{Sherpa} 2.2.2 with the NNPDF3.0NNLO PDF set,
and with an additional k-factor~\cite{PhysRevD.92.094028} of 1.7 applied.
The EW-$ZZjj$ production is simulated using \textsc{Sherpa} 2.2.2 with the NNPDF3.0NNLO PDF, and the $ZZZ \rightarrow \llll qq$ process is also taken into account in this sample.

The \Zjet events are generated using \textsc{Sherpa} 2.2.2 with the NNPDF3.0NNLO PDF,
in which the ME is calculated for up to two partons with NLO accuracy in pQCD and up to four partons with LO accuracy.
The \Zjet events are normalized using the next-to-next-to-leading-order (NNLO) cross section.
The triboson processes with full leptonic decays and at least four prompt charged leptons are generated using \textsc{Sherpa} 2.1.1.
For top-quark pair (\ttbar) production and the single top-quark productions in $t$-channel, $s$-channel and $Wt$-channel, \textsc{Powheg-Box}~v2 is used with the CT10 PDF.
The productions of \ttbar~in association with $Z$ boson(s) ($ttZ$) is modelled with \MGMCatNLO.

\subsection{Signal MC simulations}
\label{sec:hmhzz_signal_mc}

One model considered in this analysis is heavy spin-0 resonance under the Narrow Width Approximation (NWA) simulated using \textsc{Powheg-Box}~v2 MC event generator with the CT10 PDF.
The gluon-gluon fusion (ggF) production mode and vector-boson fusion (VBF) production mode are calculated separately with matrix elements up to NLO in QCD.
The \textsc{Powheg-Box} is interfaced to \textsc{Pythia8} for parton showering, and for decaying the Higgs boson into the $H \rightarrow ZZ \rightarrow \llll$ final states.
Events of NWA signal are generated at mass points between 200~\gev~to 2000~\gev~using the step of 100 (200)~\gev~up to (above) 1~\tev~in both ggF and VBF production modes.

In addition, heavy Higgs boson events under the Large Width Approximation (LWA) with widths of 1\%, 5\%, 10\% and 15\% of the boson mass are generated using \MGMCatNLO~2.3.2 interfaced to \textsc{Pythia8}.
Only ggF production is considered.
Mass points between 400~\gev~to 2000~\gev~are simulated with the step of 100 (200)~\gev~ up to (above) 1~\tev.
To describe jet multiplicity, \MGMCatNLO~is used to simulated process of $pp\to H + \geq2\text{jets}$ at NLO in QCD with the FxFx merging scheme~\cite{Frederix2012}.

Spin-2 Kaluza–Klein (KK) gravitons (\Graviton) from the Bulk Randall–Sundrum model~\cite{graviton} are also studied in this analysis.
Events are generated by \MGMCatNLO at LO in QCD, which is then interfaced to \textsc{Pythia8} for parton showering.
The \Graviton-gluon coupling \kOverMpl, where $k$ is the curvature scale of the extra dimension and \Mpl~is the reduced Planck mass, is set to 1.
The width of the resonance is correlated with the coupling \kOverMpl~and in this configuration is around ∼ 6\% of its mass. 
The mass of the \Graviton~is the only free parameter in this simplified model.
Mass points between 600~\gev~ to 2~\tev~ with 200~\gev~ spacing were generated.

%\section{Analysis selections}
\label{sec:hmhzz_selection}

\subsection{Objects selection}
\label{sec:hmhzz_objsel}

Similar as described in section~\ref{sec:vbszz_selection}, the selection of this analysis relies on the definition of multiple objects: \textit{electrons}, \textit{Muons}, and \textit{jets}.
Details of definitions for each objects are described as below:

\textbf{Electron:}
The reconstruction of electrons is described in section~\ref{sec:electron}.
In this analysis, the electron candidates satisfying \textit{Loose} working point (WP) are selected,
with a selection efficiency ranging from 90\% for transverse momentum $\pt = 20~\gev$ to 96\% for $\pt > 60~\gev$.
Also, electrons are required to have $p_{T} > 7 \gev$ and $|\eta| < 2.47$.

\textbf{Muon:}
To increase the acceptance range in reco-level for \lllljj channel, all four types of muons
(CB, ST, CT, ME muons, described in section~\ref{sec:muon}) are used.
But at most one CT, ST or ME muon is allowed in one \llll quadruplet.
The Muon candidates are required to pass $p_{T} > 5 \gev$ and $|\eta| < 2.7$,
and satisfy the \textit{Loose} identification criterion with an efficiency of 98.5\%.

\textbf{Jets:}
Jets are clustered using the anti-$k_t$ algorithm with radius parameter $R$ = 0.4 implemented in the \textsc{FastJet} package described in section~\ref{sec:jet}. 
The particle flow (PFlow) objects~\cite{PERF-2015-09}, which are the ensemble of positive energy topo-clusters surviving the energy subtraction step at the PFlow algorithm within $|\eta| < 2.5$,
and the selected tracks that are matched to a primary vertex, are used as inputs to the \textsc{FastJet} package.
Before jet-finding, the topo-cluster $\eta$ and $\phi$ are recomputed pointing to the primary vertex position, instead of the centre of detector.
For the region of $|\eta| > 2.5$ that outside the geometrical acceptance of the tracker, only the calorimeter information is available.
Therefore, the topo-clusters, formed from calorimeter cells with significant energy depositions, are used as inputs to jet reconstruction.
The jets used in this analysis are required to pass $\pt > 30~\GeV$ and $|\eta |<4.5$.
To further reduce the effects of pile-up jets, a jet vertex tagger (JVT) is applied to jets with $p_{T} <$ 60~\gev~and $|\eta| < 2.4$.

\textbf{Overlap removal:}
As the selected jet and lepton candidates can be reconstructed from same detector information, an overlap-removal procedure is applied.
For electron and muon sharing the same ID track, the electron is selected in the case that the muon is calorimeter-tagged and does not have a MS track, or is a segment-tagged muon, otherwise the muon is selected.
The jet overlapping with electron (muon) within a cone of size of $\Delta R\equiv \sqrt{(\Delta \eta)^2 + (\Delta \phi)^2}= 0.2(0.1)$ are removed.

%% ========================================================================
\subsection{Event selection}
\label{sec:hmhzz_eventsel}

First of all, the four-lepton events are required to pass single or multi-lepton triggers, with the multi-lepton ones including electron(s)-muon(s) triggers.
Due to the increasing of peak luminosity and pile-up, the \pt and \et thresholds of triggers increase slight during the data-taking periods from 2015 to 2018.
Table~\ref{tab:hmhzz_triggers} summarizes the triggers used for \lllljj channel. 
The overall trigger efficiency for selected signal events passing final selection is around 98\%.

\begin{table}[!htbp]
\begin{center}
\caption{Summary of the $\pt$ ($E_T$) trigger thresholds (in GeV) employed for the muon (electron) trigger selection in the year of 2015, 2016, 2017, and 2018.}
\label{tab:hmhzz_triggers}
\adjustbox{max width=\textwidth}{%
\vspace{0.2cm}
\begin{tabular}{|l|c|c|c|c|}
\hline
Trigger item& \multicolumn{4}{c|}{Trigger threshold}\\
& \multicolumn{1}{c|}{2015}& \multicolumn{1}{c|}{2016}& \multicolumn{1}{c|}{2017}& \multicolumn{1}{c|}{2018}\\
\hline
single muon&        $\mu 20$;~~$\mu 50$;~~$\mu 60$&
                    $\mu 24$;~~$\mu 26$;~~$\mu 40$;~~$\mu 50$&
                    $\mu 26$;~~$\mu 50$;~~$\mu 60$&
                    $\mu 26$;~~$\mu 50$;~~$\mu 60$\\
single electron&    $e24$;~~$e60$;~~$e120$&
                    $e26$;~~$e60$;~~$e140$;~~$e300$&
                    $e26$;~~$e60$;~~$e140$;~~$e300$&
                    $e26$;~~$e60$;~~$e140$;~~$e300$\\
\hline
dimuon&             $2\mu 10$;~~$\mu 18 \_\mu8$&
                    $2\mu 10$;~~$2\mu 14$;~~$\mu 22 \_\mu8$&
                    $2\mu 14$;~~$\mu 22 \_\mu8$&
                    $2\mu 14$;~~$\mu 22 \_\mu8$\\
dielectron&         $2e12$&
                    $2e15$;~~$2e17$&
                    $2e17$;~~$2e24$&
                    $2e17$;~~$2e24$\\
\hline
electron-muon&
                    $e24 \_\mu 8$&
                    $e24 \_\mu 8$;~~$e26 \_\mu 8$&
                    $e26 \_\mu 8$&
                    $e26 \_\mu 8$\\
                    $ $&
                    \multicolumn{4}{c|}{$e17\_\mu 14$;~~$e7 \_\mu24$;~~$2e12 \_\mu 10$;~~$e12 \_2\mu10$}\\
\hline

trimuon&
                    $\mu18\_2\mu4$&
                    $\mu11\_2\mu4$;~~$\mu6\_2\mu4$;~~$\mu 20\_2\mu 4$;~~$3\mu4$&
                    $4\mu4$;~~$\mu 20\_2\mu 4$;~~$3\mu4$&
                    $\mu 20\_2\mu 4$\\
                    $ $&
                    \multicolumn{4}{c|}{$3\mu6$}\\
\hline

trielectron&        $e17\_2e9$&
                    $e17\_2e9$;~~$e17\_2e10$&
                    $e24\_2e12$&
                    $e24\_2e12$\\

\hline

\end{tabular}}
\end{center}
\end{table}

The \llll quadruplets are formed by two opposite-sign, same-flavour (OSSF) lepton pairs ($l^{+}l^{-}$), selected as described in section~\ref{sec:hmhzz_objsel}.
The $\pT$ threshold of first three leading muons are required to be 20, 15 and 10~\gev.
If there are more than one combination of lepton pairing in quadruplet, the pairing is selected by keeping it with the lepton pairs closest (leading pair, refers as $m_{12}$) and second closest (sub-leading pair, refers as $m_{34}$) to Z boson mass.
The mass of leading pair is required to satisfy $50 < m_{12} < 106~\gev$, while the sub-leading pair is required to be less than 115~\gev and larger than a threshold that is $12~\gev$ for $\mfl \leq 140~\gev$, rises linearly from $12~\gev$ to $50~\gev$ with \mfl in the interval of [$140~\gev$, $190~\gev$] and is fixed to $50~\gev$ for $\mfl > 190~\gev$. 

The two lepton pairs in quadruplet are required to have angular separation with $\Delta R > 0.1$.
To suppress the contribution from $J/\psi \rightarrow ll$ decays, for $4\mu$ and $4e$ quadruplets, the event is rejected if any opposite-sign same-flavour lepton pair is found with mass below 5~\gev.
If there are more than one quadruplets from different channels in event at the point, the quadruplet with highest expected signal rate is selected in the order of $4\mu$, $2e2\mu$, $4e$.
The transverse impact-parameter significance ($|d_0|/\sigma_{d_0}$) for muons (electrons) is than required to be smaller than 3 (5) to suppress the backgrounds from heavy-flavour hadrons.

In addition, the track- (\pt) and calorimeter-based isolation criteria is required for all electrons and muons to further suppress the reducible backgrounds of \Zjet and \ttbar.
For lepton isolation selection, the two track- and calorimeter-based variables, $E_{T}^{topocone}$ and $p_{T}^{varcone}$ as described in section~\ref{sec:muon} (section~\ref{sec:electron}) for muons (electrons), are vulnerable to pileup.
For track-based variable, this is because of additional tracks in the event.
The definition of $p_{T}^{varcone}$ attempts to limit the tracks used in the calculation to those from the vertex via a loose cut of $|z_0\sin(\theta)| < 3$,
which proved to be too loose in new pile-up regime 2017 and 2018 datasets.
So new track-based variable is used, 
by adding a requirement that the track be used in determining the vertex, or that, if not, it both pass the cut on $|z_0\sin(\theta)|$ and not be used in determining any other vertex,
which makes the track-based variable to be more isolation-robust in the high pile-up regime.
The new variable is named as ptvarcone[cone]$\_$TightTTVA$\_$pt[\pt cut], where [cone] is the cone size and [\pt cut] is the cutoff for including tracks in the calculation.

For calorimeter-based variable, the calculation of $E_{T}^{topocone}$ corrects the pile-up effects by subtracting an average pileup contribution computed over the whole detector.
But with the increasing of energy density of pile-up events, the root mean square (RMS) of $E_{T}^{topocone}$ variable increases,
which leads to the increment of possibility that the pile-up fluctuations are not be accounted for correctly.
One possible solution is using particle-flow (PFlow) method to calculate the calorimeter isolation.
As part of PFlow reconstruction process, it assigns the clusters to tracks which improves the track-cluster association for better determination of the raw value of the \et in the cone.
And using PFlow jets to calculate the pileup correction provides a further improvement.
So a resulting variable named neflowisol[cone] is used.
Finally, a requirement of isolation, called \textit{FixedCutPFlowLoose}, which gives better performance in high piup-up condition is applied to electrons and muons:\\
(max(ptcone20\_TightTTVA\_pt500, ptvarcone30\_TightTTVA\_pt500) + 0.4 \times neflowisol20) / \pt < 0.16

On the top of impact parameter cut and lepton isolation cut, the four-lepton candidates are also required to originate from a common vertex to reduce \Zjet and \ttbar backgrounds.
This is ensured by applying a vertex fit $\chi^2$ cut of 4 ID tracks of lepton candidates satisfying $\chi^2 / N_{dof} < 6~(9)$ for events in 4$\mu$ (4$e$ and 2$e$2$\mu$) channel(s).

To improve the mass resolution, the QED process of final state radiation (FSR) photons in $Z$ boson decays are token into account in the reconstruction of Z bosons.
The Four-momentum of any reconstructed photon that is consistent with having been radiated from lepton(s) in leading pair are added into final state.
Moreover, the four-momenta of leptons in both (leading and sub-leading) pairs are recomputed by performing a Z-mass-constrained kinematic fit,
whieh uses a Breit–Wigner Z boson line-shape and Gaussian function with width set to the expected lepton resolution per lepton to model the momentum response function.
The Z-mass-constrained mass improves the $\mfl$ resolution by up to 15\% depending on $m_{H}$.

In summary, table~\ref{tab:hmhzz_selections} lists a comprehensive object and event level selection as described above.
Table~\ref{tab:hmhzz_cutflow} shows the cutflow of NWA ggF signal samples at several different mass points as examples.

\begin{table}[!htbp]
  \centering
  \caption{Summary of the object and event selection requirements.
  \label{tab:hmhzz_selections}}
  \vspace{0.2cm}
  \resizebox{\textwidth}{!}{%
    \begin{tabular}{lccc}
      \hline\hline
      \multicolumn{4}{c}{\textbf \textsc{\textbf{Physics Objects}}} \\
      \hline\hline
      \multicolumn{4}{c}{\textsc{Electrons}} \\
      \multicolumn{4}{c}{Loose Likelihood quality electrons with hit in innermost layer, $\et > 7$~\GeV\ and $|\eta| < 2.47$} \\
      \multicolumn{4}{c}{Interaction point constraint: $|z_{0} \cdot \sin\theta| < 0.5$~mm (if ID track is available)}  \\
      \hline
      \multicolumn{4}{c}{\textsc{Muons}} \\
      \multicolumn{4}{c}{Loose identification with $\pt > 5$~\GeV\ and $|\eta| < 2.7$} \\
      \multicolumn{4}{c}{Calo-tagged muons with $\pt > 15$~\GeV\ and $|\eta| < 0.1$, segment-tagged muons with $|\eta| < 0.1$ } \\
      \multicolumn{4}{c}{Stand-alone and silicon-associated forward restricted to the $2.5 < |\eta| < 2.7$ region} \\
      \multicolumn{4}{c}{Combined, stand-alone (with ID hits if available) and segment-tagged muons with $\pt > 5$~\GeV} \\
      \multicolumn{4}{c}{Interaction point constraint: $|d_0| < 1$~mm and $|z_{0} \cdot \sin\theta| < 0.5$~mm (if ID track is available)} \\
      \hline
      \multicolumn{4}{c}{\textsc{Jets}} \\
      \multicolumn{4}{c}{anti-$k_T$ jets with \emph{bad-loose} identification, $\pt > 30 $~\GeV\ and $|\eta| < 4.5$} \\
      \hline
      \multicolumn{4}{c}{\textsc{Overlap removal}} \\
      \multicolumn{4}{c}{Jets within $\Delta R < 0.2$ of an electron or $\Delta R < 0.1$ of a muon are removed} \\
      \hline
      \multicolumn{4}{c}{\textsc{Vertex}} \\
      \multicolumn{4}{c}{At least one collision vertex with at least two associated track} \\

      \hline
      \multicolumn{4}{c}{\textsc{Primary vertex}} \\
      \multicolumn{4}{c}{Vertex with the largest $p_T^2$ sum} \\

      \hline\hline
      \multicolumn{4}{c}{\textbf \textsc{\textbf{Event Selection}}} \\
      \hline\hline
      \textsc{Quadruplet}     & \multicolumn{3}{l}{- Require at least one quadruplet of leptons consisting of two pairs of same-flavour} \\
      \textsc{Selection}      & \multicolumn{3}{l}{opposite-charge leptons fulfilling the following requirements:} \\
      & \multicolumn{3}{l}{- \pt~thresholds for three leading leptons in the quadruplet: $20, 15\text{ and } 10$~\GeV} \\
      & \multicolumn{3}{l}{- Maximum one calo-tagged or stand-alone muon or silicon-associated forward per quadruplet} \\
      & \multicolumn{3}{l}{- Leading di-lepton mass requirement: $50 < m_{12} < 106$~GeV} \\
      & \multicolumn{3}{l}{- Sub-leading di-lepton mass requirement: $m_{\textrm{threshold}}< m_{34} < 115$~\GeV} \\
      & \multicolumn{3}{l}{- $\Delta R(\ell,\ell')>0.10$ for all leptons in the quadruplet} \\
      & \multicolumn{3}{l}{- Remove quadruplet if alternative same-flavour opposite-charge} \\
      & \multicolumn{3}{l}{di-lepton gives $m_{\ell\ell} < 5$~\GeV} \\
      & \multicolumn{3}{l}{- Keep all quadruplets passing the above selection } \\
      \hline
      \textsc{Isolation}
      & \multicolumn{3}{l}{- Contribution from the other leptons of the quadruplet is subtracted} \\
      & \multicolumn{3}{l}{- FixedCutPFlowLoose WP for all leptons} \\
      \hline
      \textsc{Impact}         & \multicolumn{3}{l}{- Apply impact parameter significance cut to all leptons of the quadruplet} \\
      \textsc{Parameter}      & \multicolumn{3}{l}{- For electrons: $d_0/\sigma_{d_0}<5$} \\
      \textsc{Significance}   & \multicolumn{3}{l}{- For muons: $d_0/\sigma_{d_0}<3$} \\
      \hline
      \textsc{Best}           & \multicolumn{3}{l}{- If more than one quadruplet has been selected, choose the quadruplet} \\
      \textsc{Quadruplet}     & \multicolumn{3}{l}{ with highest Higgs decay ME according to channel: 4$\mu$, 2$e$2$\mu$, 2$\mu$2$e$ and 4$e$} \\
      \hline
      \textsc{Vertex}         & \multicolumn{3}{l}{- Require a common vertex for the leptons:} \\
      \textsc{Selection}      & \multicolumn{3}{l}{- $\chi^{2} / \mathrm{ndof} < 5$ for $4 \mu$ and $<9$ for others decay channels} \\
      \hline\hline
  \end{tabular}%
  }
\end{table}

\begin{table}[htbp]
  \centering
  \caption{Cutflow table for NWA ggF signal samples at different mass points. $N_{\text{event}}$ represents the
  number of MC events selected after each cut is applied. The proportion of events selected relative to the initial
  number of events is also included.}
  \label{tab:hmhzz_cutflow}

  \adjustbox{max width=\textwidth}{%
  \begin{tabular}{
      l
      S[table-format=6] S[table-format=3.2] !{\quad}
      S[table-format=6] S[table-format=3.2] !{\quad}
      S[table-format=6] S[table-format=3.2] !{\quad}
      S[table-format=6] S[table-format=3.2]}
    \toprule

    \multirow{2}[3]{*}{Cut} & \multicolumn{2}{c}{$\mH = 400~\GeV$} & \multicolumn{2}{c}{$\mH = 600~\GeV$} & \multicolumn{2}{c}{$\mH = 1000~\GeV$} & \multicolumn{2}{c}{$\mH = 2000~\GeV$}  \\
    \cmidrule(lr){2-3} \cmidrule(lr){4-5} \cmidrule(lr){6-7} \cmidrule(lr){8-9}
    ~ & {$N_{\text{event}}$}  & {Rel. [\%]}  & {$N_{\text{event}}$}  & {Rel. [\%]}  & {$N_{\text{event}}$}  & {Rel. [\%]}   & {$N_{\text{event}}$}  & {Rel. [\%]}    \\
    \midrule
    Initial              & 150000 & 100.00   &   150000 & 100.00   &   149000 & 100.00   &   120000 & 100.00   \\
    Lepton selection     &  47422 &  31.61   &    52345 &  34.90   &    56932 &  38.21   &    48644 &  40.54   \\
    SFOS                 &  44086 &  29.39   &    48247 &  32.16   &    51701 &  34.70   &    43228 &  36.02   \\
    Kinematic cuts       &  44024 &  29.35   &    48215 &  32.14   &    51677 &  34.68   &    43197 &  36.00   \\
    $Z_1$ Mass           &  43857 &  29.24   &    47975 &  31.98   &    51368 &  34.48   &    42749 &  35.62   \\
    $Z_2$ Mass           &  39359 &  26.24   &    42834 &  28.56   &    45602 &  30.61   &    37479 &  31.23   \\
    $J/\psi$ Veto        &  39354 &  26.24   &    42828 &  28.55   &    45597 &  30.60   &    37477 &  31.23   \\
    $\Delta R$           &  39346 &  26.23   &    42823 &  28.55   &    45588 &  30.60   &    37473 &  31.23   \\
    Isolation            &  37088 &  24.73   &    40753 &  27.17   &    43615 &  29.27   &    35971 &  29.98   \\
    Impact parameters    &  36461 &  24.31   &    40186 &  26.79   &    43066 &  28.90   &    35610 &  29.68   \\
    Vertex requirement   &  36372 &  24.25   &    40100 &  26.73   &    42967 &  28.84   &    35529 &  29.61   \\
    Trigger              &  36333 &  24.22   &    40076 &  26.72   &    42952 &  28.83   &    35503 &  29.59   \\
    ``Badjet'' veto      &  36202 &  24.13   &    39908 &  26.61   &    42779 &  28.71   &    35350 &  29.46   \\
    \bottomrule
  \end{tabular}
  }
\end{table}

%% ======================================== Categorization ===========================

\subsection{Event categorizations}
To improve the sensitivity of search in both VBF and ggF production mode in NWA model, events are classified into the VBF- and ggF- enriched categories.
With the increment of statistic with full run-2 data, a deep neural network (DNN-) based classifier has been studied, 
while in the meantime the traditional cut-based classifier is also used as cross check.

\subsubsection{Cut-based categorization}
There are four categories in total: one VBF-enriched category and three ggF-enriched ones with different lepton-flavor channels.
The categorization is defined based on kinematic cuts:
\begin{itemize}
	\item VBF-enriched category: Events have at least two selected jets as defined in section~\ref{sec:hmhzz_objsel}, with the two leading jets being separated by $|\Delta \eta_{jj}| > 3.3$ and invariant mass satisfying $\mjj > 400~\gev$;
	\item ggF-enriched categories: The remaining events that are not classified into VBF-enriched category. Then events are categorized into three channels based on lepton-flavor, namely ggF\_2$e$2$\mu$, ggF\_4$e$ and ggF\_4$\mu$. 
\end{itemize}

\subsubsection{DNN-based categorization}
In order to target different production modes, two types of classifiers, one dedicate to VBF production while the other one for ggF, have been trained.
Details of two classifiers are described as below.

\textbf{DNN models} \\
Figure~\ref{fig:dnn_arch} shows the architecture of VBF (left) and ggF (right) network.
The VBF network includes three parts: two recurrent networks (RNNs) and one multilayer perceptron (MLP).
One RNN (and another one) takes the \pt and \eta of \pt-ordered four leptons (two leading jets) as input features, which intends to study the time relationship from particle decay between leptons (jets).
The ggF network consists of one RNN and one MLP.

\begin{figure}[htbp]
        \centering
        \subfloat[]{\includegraphics[width=0.49\textwidth]{figures/HMHZZ/selection/model_vbf_architecture.pdf}}
        \subfloat[]{\includegraphics[width=0.49\textwidth]{figures/HMHZZ/selection/model_ggf_architecture.pdf}}
        \caption{(a) VBF DNN architecture diagram. (b) ggF DNN architecture.}
        \label{fig:dnn_arch}
\end{figure}

For training, the VBF and ggF signal samples at the masses of 200, 300, 400, 500, 600, 700, 800, 900, 1000, 1200, 1400~\gev are used with positive label.
The VBF (ggF) signals are only used for VBF (ggF) classifier.
The background (with negative labels) uses simulated samples of QCD and EW \qqZZ processes as well as \ggZZ process summed according to their cross section.
In addition to the selections described in section~\ref{sec:hmhzz_eventsel}, the events used for VBF network are required to have $N_\mathrm{jets} \geq 2$, while $N_\mathrm{jets} < 2$ is required to events for ggF network.

In order to assign equivalent importance to signals and background, during the training, signal events are reweighted to follow the \mfl distribution from background, as shown in figure~\ref{fig:dnn_rwt_vbf} (figure~\ref{fig:dnn_rwt_ggf}) before (left) and after(right) reweighting for VBF (ggF) samples.

\begin{figure}[htbp]
        \centering
        \subfloat[]{\includegraphics[width=0.48\textwidth]{{figures/HMHZZ/selection/vbf_input/m4l_before_reweighting.pdf}}}
        \subfloat[]{\includegraphics[width=0.48\textwidth]{{figures/HMHZZ/selection/vbf_input/m4l_after_reweighting.pdf}}}
        \caption{(a) \mfl distribution of raw (unweighted) training events for VBF signal (blue) and background (black); (b) \mfl distribution of weighted VBF signal (blue) and background (black) used at training time.}
        \label{fig:dnn_rwt_vbf}
\end{figure}

\begin{figure}[htbp]
        \centering
        \subfloat[]{\includegraphics[width=0.48\textwidth]{{figures/HMHZZ/selection/ggf_input/m4l_all_before_reweighting.pdf}}}
        \subfloat[]{\includegraphics[width=0.48\textwidth]{{figures/HMHZZ/selection/ggf_input/m4l_all_after_reweighting.pdf}}}
        \caption{(a) \mfl distribution of raw (unweighted) training events for ggF signal (blue) and background (black); (b) \mfl distribution of weighted ggF signal (blue) and background (black) used at training time.}
        \label{fig:dnn_rwt_ggf}
\end{figure}

%After all these preparation, then the training is performed over 20 epochs with batch size of 512 (256) for VBF (ggF) network.


\textbf{Input features} \\
Table~\ref{tab:dnn_features_vbf} (table~\ref{tab:dnn_features_ggf}) lists the input features used for VBF (ggF) network during the training.
For VBF network, one RNN (the other one) takes the \pt and \eta of \pt-ordered four leptons (two leading jets) as input features, which intends to study the time relationship from particle decay between leptons (jets).
For ggF network, the only one RNN model takes as input features of the \pt and \eta of \pt-ordered four leptons.

\begin{table}[htbp]
        \centering
        \caption{Input features for the VBF network.}
        \label{tab:dnn_features_vbf}
        \begin{tabular}{c | c}
                \toprule
                Variable & Description \\
                \midrule
                $m_{4\ell}$ & 4$\ell$ invariant mass \\
                $m_{jj}$ & dijet invariant mass \\
                $p_\mathrm{T}^{jj}$ & dijet transverse momentum\\
                $\Delta\eta_{H,j}$ & difference in pseudorapidities between the 4$\ell$ system and the leading jet \\
                $\min \Delta R _{jZ}$ & minimum angular separation between one of the two $\ell\ell$ pairs and a jet\\
                $p_\mathrm{T}^j$ & transverse momenta of the two leading jets \\
                $\eta^j$ & pseudorapidities of the two leading jets \\
                $p_\mathrm{T}^\ell$ & transverse momenta of the four leptons \\
                $\eta^\ell$ & pseudorapidities of the four leptons  \\
                \bottomrule
        \end{tabular}
\end{table}

\begin{table}[htbp]
        \centering
        \caption{Input features for the ggF network.}
        \label{tab:dnn_features_ggf}
        \begin{tabular}{c | c}
                \toprule
                Variable & Description \\
                \midrule
                $m_{4\ell}$ & 4$\ell$ invariant mass \\
                $\cos\theta_1$ & decay angle of the leading $Z$ \\
                $\cos\theta_2$ & decay angle of the sub-leading $Z$ \\
                $\cos\theta^*$ & production angle of the $ZZ$ system \\
                $\Delta R _{jH}$ & angular separation between the 4$\ell$ system and the leading jet\\
                $\phi$ & azimuthal angle of the $ZZ$ system \\
                $p_\mathrm{T}^{4\ell}$ & transverse momentum of the $4\ell$ system \\
                $\eta^{4\ell}$ & pseudorapidity of the $4\ell$ system \\
                $p_\mathrm{T}^j$ & transverse momentum of up to one jet \\
                $\eta^j$ & pseudorapidity of up to one jet \\
                $p_\mathrm{T}^\ell$ & transverse momenta of the four leptons \\
                $\eta^\ell$ & pseudorapidities of the four leptons  \\
                \bottomrule
        \end{tabular}
\end{table}

%Figure~\ref{fig:dnn_vbf_distribution} (figure~\ref{fig:dnn_ggf_distribution}) shows the distributions of input features with events before training reweighting for VBF (ggF) network of background and 4 signal samples at mass points of 300, 700, 1400 and 2000~\gev.
%
%\begin{figure}[htbp]
%        \captionsetup[subfigure]{labelformat=empty}
%        \centering
%        \subfloat[]{\includegraphics[width=0.19\textwidth]{figures/HMHZZ/selection/vbf_input/input_comparison_300_to_2000_0_score_dijet_invmass}}
%        \subfloat[]{\includegraphics[width=0.19\textwidth]{figures/HMHZZ/selection/vbf_input/input_comparison_300_to_2000_2_score_dijet_pt}}
%        \subfloat[]{\includegraphics[width=0.19\textwidth]{figures/HMHZZ/selection/vbf_input/input_comparison_300_to_2000_3_score_eta_zepp_ZZ}}
%        \subfloat[]{\includegraphics[width=0.19\textwidth]{figures/HMHZZ/selection/vbf_input/input_comparison_300_to_2000_4_score_min_dR_jZ}}
%        \subfloat[]{\includegraphics[width=0.19\textwidth]{figures/HMHZZ/selection/vbf_input/input_comparison_300_to_2000_5_score_m4l_unconstrained}}\\
%
%        \subfloat[]{\includegraphics[width=0.24\textwidth]{figures/HMHZZ/selection/vbf_input/input_comparison_300_to_2000_23_score_j_1_pt}}
%        \subfloat[]{\includegraphics[width=0.24\textwidth]{figures/HMHZZ/selection/vbf_input/input_comparison_300_to_2000_24_score_j_1_eta}}
%        \subfloat[]{\includegraphics[width=0.24\textwidth]{figures/HMHZZ/selection/vbf_input/input_comparison_300_to_2000_25_score_j_2_pt}}
%        \subfloat[]{\includegraphics[width=0.24\textwidth]{figures/HMHZZ/selection/vbf_input/input_comparison_300_to_2000_26_score_j_2_eta}}\\
%
%        \subfloat[]{\includegraphics[width=0.24\textwidth]{figures/HMHZZ/selection/vbf_input/input_comparison_300_to_2000_15_score_lep_1_pt}}
%        \subfloat[]{\includegraphics[width=0.24\textwidth]{figures/HMHZZ/selection/vbf_input/input_comparison_300_to_2000_16_score_lep_1_eta}}
%        \subfloat[]{\includegraphics[width=0.24\textwidth]{figures/HMHZZ/selection/vbf_input/input_comparison_300_to_2000_17_score_lep_2_pt}}
%        \subfloat[]{\includegraphics[width=0.24\textwidth]{figures/HMHZZ/selection/vbf_input/input_comparison_300_to_2000_18_score_lep_2_eta}}\\
%
%        \subfloat[]{\includegraphics[width=0.24\textwidth]{figures/HMHZZ/selection/vbf_input/input_comparison_300_to_2000_19_score_lep_3_pt}}
%        \subfloat[]{\includegraphics[width=0.24\textwidth]{figures/HMHZZ/selection/vbf_input/input_comparison_300_to_2000_20_score_lep_3_eta}}
%        \subfloat[]{\includegraphics[width=0.24\textwidth]{figures/HMHZZ/selection/vbf_input/input_comparison_300_to_2000_21_score_lep_4_pt}}
%        \subfloat[]{\includegraphics[width=0.24\textwidth]{figures/HMHZZ/selection/vbf_input/input_comparison_300_to_2000_22_score_lep_4_eta}}\\
%        \caption{Distributions of input features as listed in table~\ref{tab:dnn_features_vbf} for the VBF network of signals at mass points of 300, 700, 1400, 2000~\gev (coloured) and the background (grey). Only events satisfying the training selection of $N_\mathrm{jets}\geq2$ are shown.}
%        \label{fig:dnn_vbf_distribution}
%\end{figure}

%\begin{figure}[htbp]
%        \centering
%        \captionsetup[subfigure]{labelformat=empty}
%        \subfloat[]{\includegraphics[width=0.24\textwidth]{figures/HMHZZ/selection/ggf_input/input_comparison_300_to_2000_5_score_m4l_unconstrained}}
%        \subfloat[]{\includegraphics[width=0.24\textwidth]{figures/HMHZZ/selection/ggf_input/input_comparison_300_to_2000_6_score_dR_jH}}
%        \subfloat[]{\includegraphics[width=0.24\textwidth]{figures/HMHZZ/selection/ggf_input/input_comparison_300_to_2000_7_score_cth1_unconstrained}}
%        \subfloat[]{\includegraphics[width=0.24\textwidth]{figures/HMHZZ/selection/ggf_input/input_comparison_300_to_2000_8_score_cth2_unconstrained}}\\
%
%        \subfloat[]{\includegraphics[width=0.24\textwidth]{figures/HMHZZ/selection/ggf_input/input_comparison_300_to_2000_9_score_cthstr_unconstrained}}
%        \subfloat[]{\includegraphics[width=0.24\textwidth]{figures/HMHZZ/selection/ggf_input/input_comparison_300_to_2000_10_score_phi_unconstrained}}
%        \subfloat[]{\includegraphics[width=0.24\textwidth]{figures/HMHZZ/selection/ggf_input/input_comparison_300_to_2000_11_score_pt4l_unconstrained}}
%        \subfloat[]{\includegraphics[width=0.24\textwidth]{figures/HMHZZ/selection/ggf_input/input_comparison_300_to_2000_12_score_eta4l_unconstrained}}\\
%
%        \subfloat[]{\includegraphics[width=0.24\textwidth]{figures/HMHZZ/selection/ggf_input/input_comparison_300_to_2000_15_score_lep_1_pt}}
%        \subfloat[]{\includegraphics[width=0.24\textwidth]{figures/HMHZZ/selection/ggf_input/input_comparison_300_to_2000_16_score_lep_1_eta}}
%        \subfloat[]{\includegraphics[width=0.24\textwidth]{figures/HMHZZ/selection/ggf_input/input_comparison_300_to_2000_17_score_lep_2_pt}}
%        \subfloat[]{\includegraphics[width=0.24\textwidth]{figures/HMHZZ/selection/ggf_input/input_comparison_300_to_2000_18_score_lep_2_eta}}\\
%
%        \subfloat[]{\includegraphics[width=0.24\textwidth]{figures/HMHZZ/selection/ggf_input/input_comparison_300_to_2000_19_score_lep_3_pt}}
%        \subfloat[]{\includegraphics[width=0.24\textwidth]{figures/HMHZZ/selection/ggf_input/input_comparison_300_to_2000_20_score_lep_3_eta}}
%        \subfloat[]{\includegraphics[width=0.24\textwidth]{figures/HMHZZ/selection/ggf_input/input_comparison_300_to_2000_21_score_lep_4_pt}}
%        \subfloat[]{\includegraphics[width=0.24\textwidth]{figures/HMHZZ/selection/ggf_input/input_comparison_300_to_2000_22_score_lep_4_eta}}\\
%
%        \caption{Distributions of input features as listed in table~\ref{tab:dnn_features_ggf} for the ggF network of signals at mass points of 300, 700, 1400, 2000~\gev (coloured) and the background (grey). Events with any jet multiplicity are shown, as this model is evaluated in both $N_\mathrm{jets}\geq2$ and $N_\mathrm{jets}<2$.}
%        \label{fig:dnn_ggf_distribution}
%\end{figure}

\textbf{Evaluation of models} \\
Figure~\ref{fig:dnn_output_score} shows the classifier response output of background samples (QCD and EW \qqZZ and \ggZZ) as well as VBF (left) and ggF (right) signal sample at 700~\gev.

\begin{figure}[htbp]
        \includegraphics[width=0.48\textwidth]{figures/HMHZZ/selection/vbf_input/clf_output.pdf}
        \includegraphics[width=0.48\textwidth]{figures/HMHZZ/selection/ggf_input/clf_output.pdf}
        \centering
        \caption{VBF (left) and ggF (right) output of the background samples (filled) and the $700~\gev$ signal sample (black).}
        \label{fig:dnn_output_score}
\end{figure}

Then the optimal cut at DNN output score is chosen based on an overall good performance of classifier to have a large significance improvement while retaining a high signal efficiency.
Figure~\ref{fig:dnn_significance} shows the significance improvements of DNN-based cuts when comparing with cut-based one at different VBF (left) and ggF (right) mass samples,
where the significance is calculated as:
\begin{equation}
Z = \sqrt{2\left(n\ln \left[ \frac{nb+\sigma^2}{b^2+n\sigma^2}\right]
        - \frac{b^2}{\sigma^2}\ln\left[1+\frac{\sigma^2(n-b)}{b(b+\sigma^2)}\right]\right)}
\end{equation}
Cut at 0.5 (0.8) for VBF (ggF) classifier is chosen as shown in solid lines.

\begin{figure}[htbp]
        \includegraphics[width=0.48\textwidth]{figures/HMHZZ/selection/VBF_significance_improvement.pdf}
        \includegraphics[width=0.48\textwidth]{figures/HMHZZ/selection/ggf_significance_improvement.pdf}
        \centering
        \caption{Significance improvements of the DNN-based over the cut-based categorization of the VBF (ggF) category for VBF (ggF) signal samples between 300 and 2000~\gev for seven different cuts on the VBF (ggF) DNN score. 
	The optimal cut of 0.8 (0.5) for VBF (ggF) DNN is chosen by a solid line, while other alternative cuts are plotted with dashed lines. 
	For VBF category, results at 2000~\gev for cuts of 0.8 and 0.9 are missing due to a lack of background events passing this tight selection.}
        \label{fig:dnn_significance}
\end{figure}

Then the events passing VBF classifier are categorized into VBF-enriched category.
Otherwise, the events failing VBF classifier but passing ggF classifier are categorized into ggF-enriched category, which is further split into 3 channels.
All remaining events are sorted into one additional category called 'rest'.
Thus there are five categories defined in DNN-based categorization, named: VBF, ggF\_2$e$2$\mu$, ggF\_4$e$, ggF\_4$\mu$ and rest.
In summary, cuts applied in categorization are defined as follow, and these phase spaces are also illustrated in figure~\ref{fig:hmhzz_dnncate}.

\begin{itemize}
	\item VBF-enriched category: Events have at least two selected jets ($\Njets \geq 2$), and with \DNNVBF > 0.8;
	\item ggF-enriched categories: $(\Njets \geq 2 \:\&\&\: \DNNVBF \leq 0.8 \:\&\&\: \DNNggF > 0.5) \:||\: (\Njets < 2 \:\&\&\: \DNNggF > 0.5)$; 
	\item rest category: All remaining events that fail VBF and ggF cuts mentioned above.
\end{itemize}

\begin{figure}[h]
\centering
\subfloat[]{\includegraphics[width=0.43\textwidth]{figures/HMHZZ/selection/classifier_diagram_c1_njets_lt2.eps}}
\subfloat[]{\includegraphics[width=0.43\textwidth]{figures/HMHZZ/selection/classifier_diagram_c1_njets_gt2.eps}}\\
\caption{Illustration of the DNN-based VBF and ggF event classification for events with (a) $\Njets < 2$ and (b) $\Njets \geq 2$.}
\label{fig:hmhzz_dnncate}
\end{figure}

\subsection{Signal acceptance} 
\label{sec:hmhzz_signal_acc}
The signal acceptance is defined as the ratio of events passing all analysis selection in each category to the total number of simulated events in whole phase space.
In denominator, the events with $\tau$ final states are not token into account.
And the contribution of $\tau$-lepton decay to electrons and muons final states is found to be negligible.

Figure~\ref{fig:hmhzz_acc_dnn} and ~\ref{fig:hmhzz_acc_cut} show the acceptance of NWA signal in DNN- and Cut- based categorization, estimated by merging the three signal MC campaigns, mc16a, mc16d and mc16e.
A 3-rd order polynomial fit is applied for each category.

\begin{figure}[h]
\centering
\subfloat[]{\includegraphics[width=0.43\textwidth]{figures/HMHZZ/selection/acc_dnn_ggF.pdf}}
\subfloat[]{\includegraphics[width=0.43\textwidth]{figures/HMHZZ/selection/acc_dnn_VBF.pdf}}\\
\caption{NWA acceptance as a function of $m_{H}$ for the DNN-based categorization for the samples of
(a) ggF production mode;
(b) VBF production mode. 
}
\label{fig:hmhzz_acc_dnn}
\end{figure}

\begin{figure}[h]
\centering
\subfloat[]{\includegraphics[width=0.43\textwidth]{figures/HMHZZ/selection/acc_cut_ggF.pdf}}
\subfloat[]{\includegraphics[width=0.43\textwidth]{figures/HMHZZ/selection/acc_cut_VBF.pdf}}\\
\caption{NWA acceptance as a function of $m_{H}$ for the Cut-based categorization for the samples of
(a) ggF production mode;
(b) VBF production mode. }
\label{fig:hmhzz_acc_cut}
\end{figure}

%\input{chapters/HMHZZ/dnn}
%\section{Background estimation}
\label{sec:hmhzz_bkg}

In this analysis, 97\% of total expected background events are from irreducible ZZ backgrounds, which includes about 86\% quark-antiquark annihilation (\qqZZ), 10\% of gluon-induced production (\ggZZ) and around 1\% of EW vector boson scattering (\qqZZ EW) contribution.
For \qqZZ EW, although it has small contribution in total background events after analysis selection, it's important for VBF category with about 16\% contribution.

In addition to irreducible backgrounds, events from \Zjet and \ttbar processes, represent as reducible backgrounds, contribute at a few percent level and can be measured using data driven method that will later be described briefly.
Additional background called `Others', including ttV and triple-V (VVV) processes, has tiny contribution and is estimated from MC simulation directly.

\subsection{Irreducible backgrounds}
The Irreducible backgrounds have events with four prompt leptons.
The normalization of two dominant backgrounds \qqZZ and \ggZZ are taken from data by statistical fit, and the normalization of small \qqZZ EW background is taken directly from MC simulation.

The \mfl shapes of all three background components are taken from MC samples and then parameterized by an empirical function for each of them in each category respectively.
Details of background modellings are illustrated as below:

The empirical function used for background parameterization is:
\begin{equation}
    f(\mfl) = C_0 H(m_{0} - \mfl) f_{1}(\mfl) + H(\mfl - m_{0}) f_{2}(\mfl),
    \label{eq:bkg_model}
\end{equation}
where,
\begin{align*}
    f_1(x) &= \left( \frac{x - a_4}{a_3} \right)^{a_1 - 1} \left( 1 + \frac{x - a_4}{a_3} \right)^{-a_1 - a_2}, \\
    f_2(x) &= \exp \left[ b_0 \left( \frac{x - b_4}{b_3} \right)^{b_1 - 1} \left( 1 + \frac{x - b_4}{b_3} \right)^{-b_1 - b_2} \right], \\
    C_0    &= \frac{f_{2} (m_0)} {f_{1} (m_0)}.
    \label{eq:bkg_model_full}
\end{align*}

The function consists of two parts, the first part $f_{1}$ describes the \mfl spectrum in low mass region where both $Z$ bosons decay on-shell, while the second one $f_{2}$ covers distribution at high mass tail.
The transition between the low- and high- mass parts is presented in function~\ref{eq:bkg_model} by the Heaviside step function $H(x)$ at the transition point $m_0$.
The $m_0$ is chosen to optimize the smoothness of the function, and practically $m_0 = 260~(350)~\gev$ is used for \qqZZ (\ggZZ and \qqZZ EW).
Besides, the continuity of two functions at $m_0$ is ensured by the factor $C_0$ applied to $f_{1}$.
The coefficients $a_{i}$ in $f_{1}$ and $b_{i}$ in $f_{2}$ are shape parameters obtained by fitting to \mfl distribution from each MC simulated sample.

Figure~\ref{fig:qqZZ_m4l_shape_all_cut_based} to ~\ref{fig:qqZZEW_m4l_shape_all_cut_based} shows the fitting results of \qqZZ, \ggZZ, \qqZZ EW backgrounds in four cut-based categories (ggF-CBA-enriched-2$e$2$\mu$, ggF-CBA-enriched-4$e$, ggF-CBA-enriched-4$\mu$ and VBF-CBA-enriched).
Figure~\ref{fig:qqZZ_m4l_shape_all_DNN} to ~\ref{fig:qqZZEW_m4l_shape_all_DNN} shows the fitting results of those backgrounds in five MVA-based categories (ggF-MVA-high-2$e$2$\mu$, ggF-MVA-high-4$e$, ggF-MVA-high-4$\mu$, ggF-MVA-low and VBF-MVA-enriched).

\begin{figure}[htbp]
    \centering
    \includegraphics[width=0.32\textwidth]{figures/HMHZZ/background/cut_based/bkg_shape_qqZZ_ggF_4mu_190_to_2200_log.pdf}
    \includegraphics[width=0.32\textwidth]{figures/HMHZZ/background/cut_based/bkg_shape_qqZZ_ggF_4e_190_to_2200_log.pdf} \\
    \includegraphics[width=0.32\textwidth]{figures/HMHZZ/background/cut_based/bkg_shape_qqZZ_ggF_2mu2e_190_to_2200_log.pdf}
    \includegraphics[width=0.32\textwidth]{figures/HMHZZ/background/cut_based/bkg_shape_qqZZ_VBF_incl_190_to_2200_log.pdf}
    \caption{Distributions of the \mfl invariant mass fit projections of the \qqZZ background samples for the $4\mu$,
    $4e$ and $2\mu 2e$ final states in the ggF-CBA-enriched category, and the $4\ell$ inclusive VBF-CBA-enriched category.
    Cut-based categorization is used.} 
    \label{fig:qqZZ_m4l_shape_all_cut_based}
\end{figure}

\begin{figure}[htbp]
    \centering
    \includegraphics[width=0.32\textwidth]{figures/HMHZZ/background/cut_based/bkg_shape_ggZZ_ggF_4mu_180_to_2200_log.pdf}
    \includegraphics[width=0.32\textwidth]{figures/HMHZZ/background/cut_based/bkg_shape_ggZZ_ggF_4e_185_to_2200_log.pdf} \\
    \includegraphics[width=0.32\textwidth]{figures/HMHZZ/background/cut_based/bkg_shape_ggZZ_ggF_2mu2e_185_to_2200_log.pdf}
    \includegraphics[width=0.32\textwidth]{figures/HMHZZ/background/cut_based/bkg_shape_ggZZ_VBF_incl_180_to_2200_log.pdf}
    \caption{Distributions of the \mfl invariant mass fit projections of the \ggZZ background samples for the $4\mu$,
    $4e$ and $2\mu 2e$ final states in the ggF-CBA-enriched category, and the $4\ell$ inclusive VBF-CBA-enriched category. 
    Cut-based categorization is used.} 
    \label{fig:ggZZ_m4l_shape_all_cut_based}
\end{figure}

\begin{figure}[htbp]
    \centering
    \includegraphics[width=0.32\textwidth]{figures/HMHZZ/background/cut_based/bkg_shape_qqZZEW_ggF_4mu_180_to_2200_log.pdf}
    \includegraphics[width=0.32\textwidth]{figures/HMHZZ/background/cut_based/bkg_shape_qqZZEW_ggF_4e_180_to_2200_log.pdf} \\
    \includegraphics[width=0.32\textwidth]{figures/HMHZZ/background/cut_based/bkg_shape_qqZZEW_ggF_2mu2e_180_to_2200_log.pdf}
    \includegraphics[width=0.32\textwidth]{figures/HMHZZ/background/cut_based/bkg_shape_qqZZEW_VBF_incl_180_to_2200_log.pdf}
    \caption{Distributions of the \mfl invariant mass fit projections of the \qqZZ (EW) background samples for the
    $4\mu$, $4e$ and $2\mu 2e$ final states in the ggF-CBA-enriched category, and the $4\ell$ inclusive VBF-CBA-enriched category. 
    Cut-based categorization is used.} 
    \label{fig:qqZZEW_m4l_shape_all_cut_based}
\end{figure}

%===========================================================================================================================
\begin{figure}[htbp]
    \centering
    \includegraphics[width=0.32\textwidth]{figures/HMHZZ/background/dnn/bkg_shape_qqZZ_ggF_4mu_190_to_2200_log.pdf}
    \includegraphics[width=0.32\textwidth]{figures/HMHZZ/background/dnn/bkg_shape_qqZZ_ggF_4e_190_to_2200_log.pdf}
    \includegraphics[width=0.32\textwidth]{figures/HMHZZ/background/dnn/bkg_shape_qqZZ_ggF_2mu2e_190_to_2200_log.pdf} \\
    \includegraphics[width=0.32\textwidth]{figures/HMHZZ/background/dnn/bkg_shape_qqZZ_VBF_incl_190_to_2200_log.pdf}
    \includegraphics[width=0.32\textwidth]{figures/HMHZZ/background/dnn/bkg_shape_qqZZ_rest_190_to_2200_log.pdf}
    \caption{Distributions of the \mfl invariant mass fit projections of the \qqZZ background samples for the $4\mu$,
    $4e$ and $2\mu 2e$ final states in the ggF-MVA-high category, the $4\ell$ inclusive ggF-MVA-low category and VBF-MVA-enriched category.
    DNN-based categorization is used.} 
    \label{fig:qqZZ_m4l_shape_all_DNN}
\end{figure}

\begin{figure}[htbp]
    \centering
    \includegraphics[width=0.32\textwidth]{figures/HMHZZ/background/dnn/bkg_shape_ggZZ_ggF_4mu_180_to_2200_log.pdf}
    \includegraphics[width=0.32\textwidth]{figures/HMHZZ/background/dnn/bkg_shape_ggZZ_ggF_4e_185_to_2200_log.pdf}
    \includegraphics[width=0.32\textwidth]{figures/HMHZZ/background/dnn/bkg_shape_ggZZ_ggF_2mu2e_185_to_2200_log.pdf} \\
    \includegraphics[width=0.32\textwidth]{figures/HMHZZ/background/dnn/bkg_shape_ggZZ_VBF_incl_180_to_2200_log.pdf}
    \includegraphics[width=0.32\textwidth]{figures/HMHZZ/background/dnn/bkg_shape_ggZZ_rest_190_to_2200_log.pdf}
    \caption{Distributions of the \mfl invariant mass fit projections of the \ggZZ background samples for the $4\mu$,
    $4e$ and $2\mu 2e$ final states in the ggF-MVA-high category, the $4\ell$ inclusive ggF-MVA-low category and VBF-MVA-enriched category.
    DNN-based categorization is used.} 
    \label{fig:ggZZ_m4l_shape_all_DNN}
\end{figure}

\begin{figure}[htbp]
    \centering
    \includegraphics[width=0.32\textwidth]{figures/HMHZZ/background/dnn/bkg_shape_qqZZEW_ggF_4mu_180_to_2200_log.pdf}
    \includegraphics[width=0.32\textwidth]{figures/HMHZZ/background/dnn/bkg_shape_qqZZEW_ggF_4e_180_to_2200_log.pdf}
    \includegraphics[width=0.32\textwidth]{figures/HMHZZ/background/dnn/bkg_shape_qqZZEW_ggF_2mu2e_180_to_2200_log.pdf} \\
    \includegraphics[width=0.32\textwidth]{figures/HMHZZ/background/dnn/bkg_shape_qqZZEW_VBF_incl_180_to_2200_log.pdf}
    \includegraphics[width=0.32\textwidth]{figures/HMHZZ/background/dnn/bkg_shape_qqZZEW_rest_180_to_2200_log.pdf}
    \caption{Distributions of the \mfl invariant mass fit projections of the \qqZZ (EW) background samples for the
    $4\mu$, $4e$ and $2\mu 2e$ final states in the ggF-MVA-high category, the $4\ell$ inclusive ggF-MVA-low category and VBF-MVA-enriched category.
    DNN-based categorization is used.} 
    \label{fig:qqZZEW_m4l_shape_all_DNN}
\end{figure}

\subsection{Reducible backgrounds}

Similar to section~\ref{sec:background}, the reducible backgrounds include \Zjet (consisting of both heavy- and light-flavour jets), top quark pair, and $WZ$ production, which contain fake and non-isolated leptons.
The simulations are not very robust in terms of the selection efficiencies.
Thus, the data-driven method is applied to estimated the normalization of those processes in different control regions (CRs).
The estimations in this analysis are performed separately for \llmumu and \llee final states, with slightly different approaches for ``muon'' and ``electron'' backgrounds.

The ``electron'' backgrounds mostly come from process of a $Z$ boson with light-flavour jets ($Z$+LF) misidentified as electrons.
The large contribution of ``muon'' backgrounds come from heavy-flavour jets produced in association with a $Z$ boson ($Z$+HF) or in the decays of top quark.
The estimations are done following the common H4l studies without a specific \mfl range requirement~\cite{PhysRevD.91.012006}, and then the corresponding fraction of event yield in $\mfl > 200~\gev$ is calculated from MC simulation.

\textbf{\llmumu final states} 

The normalizations of ``muon'' backgrounds are extracted from simultaneous fits of the leading lepton pair's invariant mass ($m_{12}$) in four orthogonal CRs:
\begin{itemize}
	\item \textbf{Inverted $d_{0}$ CR}: this CR is formed by inverting the $d_{0}$ selection for at least one lepton in subleading lepton pair while the leptons in leading pair are required to pass all standard selection.
This CR enhances $Z$+HF and \ttbar as leptons from heavy-flavour hadronic decays are characterised by large d0.
	\item \textbf{$e\mu+\mu\mu$ CR}: this CR is formed using an opposite-sign different-flavour dilepton in leading pair.
It aims to enhance \ttbar background as the leading lepton pair cannot come from $Z$ boson decay.
	\item \textbf{Inverted isolation CR}: in this CR, leptons in leading pair are required to satisfy all standard analysis selection, but for leptons in subleading pair, they are required to pass $d_{0}$ selection but have at least one of them failing isolation selection.
This CR enhances the events from $Z$+LF processes while suppress $Z$+HF by $d_{0}$ cut.
	\item \textbf{Same-sign CR}: in this CR, the leptons in subleading pair are required to have same-charge, while the leading pair still passes standard selection.
This CR is not dominant by any specific background since all reducible backgrounds could have sizable contribution to it.
\end{itemize}

The fit results of normalizations are then propagated to signal region (SR) by applying transfer factors to account for the difference of selection efficiencies between SR and CRs.
The transfer factors are computed using $Z+\mu$ MC samples.

\textbf{\llee final states} 

The ``electron'' backgrounds are estimated in $3\ell+X$ CR, where $X$ denotes the lower \pt electron in the subleading pair.
The selection and identification criterias for $X$ are relaxed , while other three leptons must satisfy the standard selection.
In this case, $X$ could be a light-flavour jet, a photon conversion or an electron from heavy-flavour hadron decay.
Moreover, the subleading pair is required to have same charge dilepton to ensure the orthogonality to the signal region.
The normalization of backgrounds are obtained based on a fit to the number of hits in the innermost ID layer in CR,
and the transfer factors are computed from $Z+e$ simulated sample.

The \mfl shapes of reducible backgrounds are obtained from MC simulation in signal region, and then smoothed by an one-dimensional kernel estimation,
which models the input data as a superposition of Gaussian kernels, one for each data point with contributing $1/N$ to total integral $N$~\cite{Cranmer:2000du}.
The difference from using different smoothing strength ($\rho$) in kernel estimation is taken into account as additional shape uncertainties for these reducible backgrounds.

%\input{chapters/HMHZZ/modeling}
%\section{Systematic uncertainties}
\label{sec:hmhzz_sys}

This section describes the sources and values of theoretical and experimental systematic uncertainties considered in this analysis.

%% =========================================================================================================================
\subsection{Theoretical uncertainties}

The theoretical modelling uncertainties include the PDF variations, missing QCD higher-order corrections via the variations of factorisation and renormalization scales,
and the parton showering uncertainties.

\subsubsection{Theoretical uncertainties for signal}
\label{sec:hmhzz_theo_signal}

The PDF, QCD scale and parton showering uncertainties affecting the acceptance difference originating from analysis selection for signal are taken into account in different categories.
The acceptance uncertainties are calculated on the acceptance factor which extrapolates from the fiducial space to the full phase space by a simple ratio:
\begin{equation}
        A = \frac{N_{fiducial}}{N_{total}}
\end{equation}

For PDF uncertainties, the standard derivations of 100 PDF replicas of NNPDF3.0 NNLO, as well as comparison to two external PDF sets: MMHT2014 NNLO, CT14 NNLO are considered.
For missing QCD higher-order corrections, the effects are studied with truth events by comparing weights corresponding to
variations of the renormalization and factorization scale factors, up and down by a factor of two, and the envelop of different variations is used.
The parton showering uncertainties are estimated by comparing events with different setting via \textsc{Pythia8}.

Systematic uncertainties are studied for both cut- and MVA- based event categorizations, 
for cut-based analysis in two different categories: the inclusive ggF-CBA-enriched and VBF-CBA-enriched category,
and for MVA-based one in three different categories: inclusive ggF-MVA-high, ggF-MVA-low and VBF-MVA-enriched category.
This section shows the MVA-based results as an example.

Table~\ref{tab:acc-ggF-dnn} and ~\ref{tab:acc-VBF-dnn} show the theoretical uncertainties mentioned above for ggF and VBF signal respectively in MVA-based categorization.

\begin{table}[htbp]
  \centering
  \caption{Summary of acceptance uncertainties of PDF, QCD scale and parton shower variations for ggF production. The MVA-based categorization is used.}
  \label{tab:acc-ggF-dnn}
  \begin{spacing}{0.75}
  \begin{tabular}{cccc}
    \toprule
    Categories  & PDF    & QCD Scale  & Parton Shower \\
    \midrule
    ggF-MVA-high  & 0.40\% & 0.06\% & 2.03\% \\
    ggF-MVA-low   & 0.56\% & 0.07\% & 4.86\% \\
    VBF-MVA-enriched  & 0.53\% & 0.09\% & 3.43\% \\
    \bottomrule
  \end{tabular}
  \end{spacing}
\end{table}

\begin{table}[htbp]
  \centering
  \caption{Summary of acceptance uncertainties of PDF, QCD scale and parton shower variations for VBF production. The MVA-based categorization is used.}
  \label{tab:acc-VBF-dnn}
  \begin{spacing}{0.75}
  \begin{tabular}{cccc}
    \toprule
    Categories  & PDF    & QCD Scale  & Parton Shower \\
    \midrule
    ggF-MVA-high  & 0.18\% & 1.20\% & 0.41\% \\
    ggF-MVA-low   & 0.43\% & 0.26\% & 0.36\% \\
    VBF-MVA-enriched  & 0.23\% & 3.19\% & 0.85\% \\
    \bottomrule
  \end{tabular}
  \end{spacing}
\end{table}

%\textbf{Cut-based analysis} \\
%
%Table~\ref{tab:acc-ggF-cut} and ~\ref{tab:acc-VBF-cut} show the theoretical uncertainties for ggF and VBF signals respectively in cut-based categorization.
%The uncertainties are computed in two different categories: the inclusive ggF and VBF category.
%\begin{table}[htbp]
%  \centering
%  \caption{Summary of acceptance uncertainties of PDF, QCD scale and parton shower variations for ggF production. The cut-based categorization is used.}
%  \label{tab:acc-ggF-cut}
%  \begin{tabular}{cccc}
%    \toprule
%    Categories  & PDF    & QCD Scale  & Parton Shower \\
%    \midrule
%    ggF  & 0.44\% & 0.07\% & 0.22\% \\
%    VBF  & 0.61\% & 0.12\% & 3.33\% \\
%    \bottomrule
%  \end{tabular}
%\end{table}
%
%\begin{table}[htbp]
%  \centering
%  \caption{Summary of acceptance uncertainties of PDF, QCD scale and parton shower variations for VBF production. The cut-based categorization is used.}
%  \label{tab:acc-VBF-cut}
%  \begin{tabular}{cccc}
%    \toprule
%    Categories  & PDF    & QCD Scale  & Parton Shower \\
%    \midrule
%    ggF  & 0.18\% & 2.87\% & 0.52\% \\
%    VBF  & 0.08\% & 4.52\% & 0.72\% \\
%    \bottomrule
%  \end{tabular}
%\end{table}


\subsubsection{Theoretical uncertainties for SM background processes}

The theoretical uncertainties of irreducible $ZZ$ backgrounds are considered in terms of both the variations of shape of \mfl distributions
and the acceptance originating from the event selection.

The PDF and QCD scale uncertainties are considered by using the same method as described for signal.
The parton showering uncertainties for those \textsc{Sherpa} samples are evaluated by varying the resummation scale by a factor of 2, 
changing the CKKW setting and using different showering option, following the PMG recommendation in Ref.~\cite{twiki_pmgsyst},
and the quadratic sum between the uncertainties in different kinds of showering option is taken as final result of uncertainties.
Moreover, the shape uncertainty associated with electroweak higher-order correction for \qqZZ process is also taken into account.

Same as for signals, these theoretical uncertainties for irreducible backgrounds are studied for both cut- and MVA- based event categorizations.
The value of shape uncertainties vary from less than 1\% at low mass region to 50\% at high mass tail due to large statistic fluctuation.
As for the acceptance uncertainties, the values vary from about 1\% for PDF variations to 40\% for parton showering variations.
The VBF category has relative larger uncertainties.

Table~\ref{tab:acc-all-qqZZ_MVA} summarizes the acceptance uncertainties of PDF, QCD scale, and parton showering variations for the dominant background: \qqZZ.

\begin{table}[htbp]
  \centering
  \caption{Summary of acceptance uncertainties of PDF, scale, and parton showering variations for QCD \qqZZ background. The MVA-based categorization is used.}
  \label{tab:acc-all-qqZZ_MVA}
  \begin{spacing}{0.75}
  \begin{tabular}{cccc}
    \toprule
    Categories  & PDF    & QCD Scale   & Parton showering \\
    \midrule
    ggF-MVA-high  & 1.15\% & 10.16 \% & 3.71\% \\
    ggF-MVA-low   & 1.04\% & 3.26  \% & 3.80\% \\
    VBF-MVA-enriched  & 2.91\% & 27.90 \% & 23.82\% \\
    \bottomrule
  \end{tabular}
  \end{spacing}
\end{table}

%% =========================================================================================================================
\subsection{Experimental systematics}

The signal and background predictions used in this analysis are also affected by various sources of experimental systematic uncertainties.
Similar as described in section~\ref{sec:vbszz_exp_uncer}, the dominant experimental uncertainties in this analysis also come from the energy/momentum scales 
and reconstruction and identification efficiencies of the leptons and jets, as well as the luminosity uncertainty.
The systematic uncertainties are calculated using the recommendations from the Combined Performance (CP) groups of ATLAS experiment.
In addition, as mentioned in previous sections, the uncertainties of irreducible background modelling, reducible background shape smoothing procedure and signal yield difference between simulation and parameterization are all taken into account.
%Table~\ref{tab:np_list} summarizes the experimental systematics considered in this analysis that affect either the normalization of total event yield or the shape of \mfl distribution.
The impact of a few largest systematics and their value from statistical fit are studied in section~\ref{sec:hmhzz_result_4l}.

\iffalse
\begin{table}
  \centering
  \caption{
  A list of the experimental systematics considered in this analysis. The NPs have been separated by whether they only
  affect the normalisation (left column) or if they affect the shape (right column) of the \mfl distribution. They are
  further subdivided into the primary objects that they affect.
  }
  \begin{spacing}{0.65}
  \small
  \begin{tabular}{l|l}
    \toprule
    \multicolumn{1}{c}{Normalisation NPs} & \multicolumn{1}{c}{Shape NPs} \\
    \midrule
    \multicolumn{2}{c}{\textbf{Electrons}} \\
    \midrule
    \texttt{EL\_EFF\_ID\_CorrUncertaintyNP[0-15]}               & \texttt{EG\_RESOLUTION\_ALL} \\
    \texttt{EL\_EFF\_ID\_SIMPLIFIED\_UncorrUncertaintyNP[0-17]} & \texttt{EG\_SCALE\_ALLCORR} \\
    \texttt{EL\_EFF\_Iso\_TOTAL\_1NPCOR\_PLUS\_UNCOR}           & \texttt{EG\_SCALE\_E4SCINTILLATOR} \\
    \texttt{EL\_EFF\_Reco\_TOTAL\_1NPCOR\_PLUS\_UNCOR}          & \texttt{EG\_SCALE\_LARCALIB\_EXTRA2015PRE} \\
    ~                                                           & \texttt{EG\_SCALE\_LARTEMPERATURE\_EXTRA2015PRE} \\
    ~                                                           & \texttt{EG\_SCALE\_LARTEMPERATURE\_EXTRA2016PRE} \\
    \midrule
    \multicolumn{2}{c}{\textbf{Muons}} \\
    \midrule
    \texttt{MUON\_EFF\_ISO\_STAT}         & \texttt{MUON\_ID} \\
    \texttt{MUON\_EFF\_ISO\_SYS}          & \texttt{MUON\_MS} \\
    \texttt{MUON\_EFF\_RECO\_STAT}        & \texttt{MUON\_SAGITTA\_RESBIAS} \\
    \texttt{MUON\_EFF\_RECO\_STAT\_LOWPT} & \texttt{MUON\_SAGITTA\_RHO} \\
    \texttt{MUON\_EFF\_RECO\_SYS}         & \texttt{MUON\_SCALE} \\
    \texttt{MUON\_EFF\_RECO\_SYS\_LOWPT}  & ~ \\
    \texttt{MUON\_EFF\_TTVA\_STAT}        & ~ \\
    \texttt{MUON\_EFF\_TTVA\_SYS}         & ~ \\
    \midrule
    \multicolumn{2}{c}{\textbf{Jets}} \\
    \midrule
    ~                            & \texttt{JET\_BJES\_Response} \\
    ~                            & \texttt{JET\_EffectiveNP\_[1-7]} \\
    ~                            & \texttt{JET\_EffectiveNP\_8restTerm} \\
    ~                            & \texttt{JET\_EtaIntercalibration\_Modelling} \\
    ~                            & \texttt{JET\_EtaIntercalibration\_NonClosure\_highE} \\
    ~                            & \texttt{JET\_EtaIntercalibration\_NonClosure\_negEta} \\
    ~                            & \texttt{JET\_EtaIntercalibration\_NonClosure\_posEta} \\
    ~                            & \texttt{JET\_EtaIntercalibration\_TotalStat} \\
    ~                            & \texttt{JET\_Flavor\_Composition} \\
    ~                            & \texttt{JET\_Flavor\_Response} \\
    ~                            & \texttt{JET\_JER\_DataVsMC} \\
    ~                            & \texttt{JET\_JER\_EffectiveNP\_[1-6]} \\
    ~                            & \texttt{JET\_JER\_EffectiveNP\_7restTerm} \\
    ~                            & \texttt{JET\_Pileup\_OffsetMu} \\
    ~                            & \texttt{JET\_Pileup\_OffsetNPV} \\
    ~                            & \texttt{JET\_Pileup\_PtTerm} \\
    ~                            & \texttt{JET\_Pileup\_RhoTopology} \\
    ~                            & \texttt{JET\_PunchThrough\_MC16} \\
    ~                            & \texttt{JET\_SingleParticle\_HighPt} \\
    \midrule
    \multicolumn{2}{c}{\textbf{Other}} \\
    \midrule
    \texttt{HOEW\_QCD\_syst}    & ~ \\
    \texttt{HOEW\_syst}         & ~ \\
    \texttt{HOQCD\_scale\_syst} & ~ \\
    \texttt{PRW\_DATASF}        & ~ \\
    \bottomrule
  \end{tabular}
  \end{spacing}
  \label{tab:np_list}
\end{table}
\fi

%\subsection{Results in \llll channel}
\label{sec:hmhzz_result_4l}

%\input{chapters/HMHZZ/combination}

%\clearpage
%\section{Conclusion}

Searches of heavy resonances decaying into a pair of $Z$ boson to \llll final state are performed using
139~\ifb of 13~\tev~pp collision data collected by ATLAS experiment at the LHC.
The results are interpreted as 95\% CL upper limits on the production cross section of a spin-0 and spin-2 resonances under different theoretical models.
The search range of the hypothetical resonances is between 200~\gev~ to 2000~\gev~ depending on the signal model.

The spin-0 resonance is assumed to be a heavy Higgs like scalar produced predominantly from gluon–gluon fusion (ggF) and vector-boson fusion (VBF) decays, 
and it is studied under both the narrow-width approximation and with the large-width assumption.
For narrow-width approximation, the exclusion limits on cross section of heavy scalar decaying into two $Z$ bosons are set separately for ggF and VBF production modes, under MVA- and cut- based analysis.
In MVA-based analysis, the 95\% CL upper limit range is from 215 fb at $\mH = 240~\gev$ to 5.3 fb at $\mH = 2000~\gev$ for ggF production mode, 
and from 87 fb at $\mH = 255~\gev$ to 5.1 fb at $\mH = 1960~\gev$ for VBF production mode.
In cut-based analysis, the 95\% CL upper limit range is from 259 fb at $\mH = 245~\gev$ to 5.3 fb at $\mH = 2000~\gev$ for ggF production mode, 
and from 113 fb at $\mH = 240~\gev$ to 5.1 fb at $\mH = 2000~\gev$ for VBF production mode.
MVA-based analysis gains about 20\% improvement on upper limits at lower mass region comparing to the cut-based analysis, while for mass above 1500~\gev, both analyses perform closely.
For large-width approximation, limits are studied on ggF production rate at four different width assumptions: 1\%, 5\%, 10\% and 15\% of resonance's mass,
 with the interference between the heavy scalar and the SM Higgs boson as well as the heavy scalar and the SM \ggZZ continuum background taken into account.
The maximum and minimum of upper limits are obtained as 78 fb at $\mH = 400~\gev$ to 5.9 fb at $\mH = 2000~\gev$ for 1\% width;
98 fb at $\mH = 540~\gev$ to 6.4 fb at $\mH = 2000~\gev$ for 5\% width;
119 fb at $\mH = 540~\gev$ to 7.1 fb at $\mH = 2000~\gev$ for 10\% width;
133 fb at $\mH = 540~\gev$ to 7.5 fb at $\mH = 2000~\gev$ for 15\% width.
Last but not least, the framework of the Randall–Sundrum model with a graviton excitation spin-2 resonance with m($G_{KK}$) < 1500~\gev~ is excluded at 95\% CL.


\bibliography{bib/reference}

\appendix
%\input{chapters/complementary}

\backmatter
%% !TeX root = ../main.tex

\begin{publications}

\section*{已发表论文}

\begin{enumerate}
    \item ATLAS Collaboration. Search for electroweak diboson production in association with a high-mass dijet system in semileptonic final states in pp collisions at 13TeV with the ATLAS detector[J]. \textbf{Phys. Rev. D 100, 032007}.
    \item ATLAS Collaboration. Constraints on off-shell Higgs boson production and the Higgs boson total width in $ZZ \rightarrow \llll$ and $ZZ \rightarrow \llvv$ final states with the ATLAS detector[J]. \textbf{Physics Letters B, 2018, 786}.
    \item ATLAS Collaboration. Measurement of the four-lepton invariant mass spectrum in 13~\tev~ proton-proton collisions with the ATLAS detector[J]. \textbf{Journal of High Energy Physics, 2019, 2019(4): 48}.
    \item Chen K, et al. Design and Evaluation of the LAr Trigger Digitizer Board in the ATLAS Phase-I Upgrade[J]. \textbf{IEEE Transactions on Nuclear Science, 2019, 66(8): 2011-2016}.
\end{enumerate}

\section*{待发表论文}

\begin{enumerate}
    \item Observation of electroweak production of two jets and a Z-boson pair with the ATLAS detector at the LHC. \textbf{Submitted to Nature Physics, e-print: arXiv:2004.10612}.
    \item Search for heavy resonances decaying into a pair of Z bosons in $ZZ \rightarrow \llll$ and $ZZ \rightarrow \llvv$ final states using 139~\ifb~ of pp collisions at 13~\tev~ with the ATLAS detector. \textbf{Submitted to Eur. Phy. J. C, e-print: arXiv:2009.14791}.
\end{enumerate}

\section*{会议报告}
\begin{enumerate}
    \item Search for heavy resonances decaying into a pair of Z bosons with the ATLAS Detector. \textbf{China LHC Physics Workshop, online (Beijing, China), 2020.11}
    \item Measurement of Inclusive \llll and \llvv + 2-jet Cross Section and Observation of EW Component with the ATLAS detector. \textbf{Workshop on Connecting Insights in Fundamental Physics: Standard Model and Beyond, Corfu, Greece, 2019.9}
    \item Off-shell Higgs signal strength measurement in the high-mass $ZZ \rightarrow \llll$ and $ZZ \rightarrow \llvv$ final states with the ATLAS detector. \textbf{American Physics Society April Meeting, Columbus, USA, 2018.4}
\end{enumerate}

\end{publications}


\end{document}
