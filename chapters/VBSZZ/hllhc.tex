\section{Prospect study of EW-$ZZjj$ production in HL-LHC}

\subsection{Introduction}
The High-Luminosity Large Hadron Collider (HL-LHC) project aims to increase luminosity by a factor of 10 beyond the LHC’s design value 
to increase the potential for discoveries after 2025.
The designed luminosity will reach 3000~\ifb with the centre-of-mass energy of 14~\tev.

As introduced in previous sections, with full run-2 data of 139~\ifb collected by ATLAS detector at LHC, 
the EW-$ZZjj$ production is the last channel of observation for VBS processes with massive boson 
due to its very low cross section in $ZZ$ decay.
So we expect that this channel will benefit significantly from the increased luminosity at the high-luminosity LHC (HL-HLC),
and can be studied in great details for this known mechanism.

In this section, a prospect study has been performed for EW-$ZZjj$ production at the HL-LHC in the llll channel with the ATLAS detector.
The study uses 3000~\ifb of simulated pp collisions at a centre-of-mass energy of 14~\tev~that is expected to be recorded by the ATLAS detector at HL-LHC.
All simulated events are produced at particle-level, 
and the detector effects of lepton and jet reconstruction and identification are estimated by corrections, 
assuming the mean number of interactions per bunch crossing ($<\mu>$) of 200.

\subsection{The ATLAS detector at HL-LHC}

As the expectation of HL-LHC, the new Inner Tracker (ITk)\cite{Collaboration:2285585}
 will extend the acceptance capability of ATLAS detector to pseudorapidity ($|\eta|$) up to 4.0.
By including a forward muon trigger, the upgraded Muon Spectrometer\cite{Collaboration:2285580} is also expected to provide 
muon identification capabilities to $|\eta|$ up to 4.0.
The new high granularity timing detector (HGTD)\cite{Collaboration:2623663} designed to mitigate the pile-up (PU) effects 
is also foreseen in the forward region of $2.4 < |\eta| < 4.0$.
The expected performance of the upgraded ATLAS detector at HL-LHC has been studied as reported in Ref.\cite{ATL-PHYS-PUB-2016-026}.

\subsection{Simulation}

The analysis is performed using particle-level events for simulated samples.
The samples are generated at $\sqrt{s} = 14~\tev$~and with a fast simulation based on setting for ATLAS detector at HL-LHC.
The signal in this analysis is EW-$ZZjj$ process, while only the dominanted ireducible background of QCD-$ZZjj$ is considered.
Both signal and background are generated using \textsc{Sherpa} with NNPDF3.0NNLO PDF set.
The signal sample is generated with two jets at Matrix Element (ME) level.
The background is qq-initial process, in which events with up to one (three) outgoing partons are generated at NLO (LO) in perturbative QCD.
As a quick study, other irreducible backgrounds like fake backgrounds from \Zjet and top-quark processes, as well as Diboson without 4l final-state and Triboson processes are not included into this analysis.
Furthermore, for hard scattering events, the pile-up collisions are set with a mean value of 200 interactions per bunch crossing.
In studies, signal and background yields are then scaled to 3000~\ifb for HL-LHC.

\subsection{Event selection}

The analysis selection follows closely to the one in ATLAS run-2 analysis as described in section~\ref{sec:selection}.
Here are some changes according to the expectation of HL-LHC scenario for ATLAS detector:
\begin{itemize}
	\item Extend the lepton indentification in forward with $|\eta| <$ 4.0
	\item Pile-up (PU) jet suppression is applied with a PU rejection factor of 50 for all PU jets in the region of $|\eta| <$ 3.8, based on the expected ATLAS detector performance at the HL-LHC.
	\item The jets are reuiqred to have $\pT >$ 30 (70)~\gev~in the $|\eta| <$ 3.8 ($3.8 < |\eta| < 4.5)$ region.
	\item For two selected jets, tight the $\mjj$ requirement to > 600~\gev, and require $\Delta \eta_{jj} >$ 2.
\end{itemize}
In addition, a fiducial volume, which is used to study the expected precision of the cross-section measurements,
 is defined at particle-level with the same kinematic requirements listed above.

Table~\ref{tab:event_yield} summarized the number of selected signal and background events normalized to 3000~\ifb.
In addition to the \textit{baseline} selection listed above, two alternative selections are also studies:
\begin{itemize}
	\item Reduce the lepton $\eta$ region to 2.7, to uncerstand the effect due to forward lepton reconstruction and identification with the upgraded ATLAS detector.
	\item Only apply the PU jet suppression with region $|\eta| < 2.4$, to measure the improvement of \textit{baseline} by extending the rejection range of PU jets at the HL-LHC.
\end{itemize}
\begin{table}[htbp]
  \small
  \centering
  \begin{tabular}{|c|c|c|c|}
    \hline
    Selection & $N_{\mathrm{EW-ZZjj}}$ & $N_{\mathrm{QCD-ZZjj}}$ & $N_{\mathrm{EW-ZZjj}}$ / $\sqrt{N_{\mathrm{QCD-ZZjj}}}$ \\
    \hline
    Baseline                                 & 432 $\pm$ 21 & 1402 $\pm$ 37   & 11.54 $\pm$ 0.58 \\
    \hline
    Leptons with $|\eta|<$ 2.7               & 373 $\pm$ 19 & 1058 $\pm$ 33   & 11.46 $\pm$ 0.62 \\
    \hline
    PU jet suppression only in $|\eta|<$ 2.4 & 536 $\pm$ 23 & 15470 $\pm$ 120 &  4.31 $\pm$ 0.19  \\
    \hline
  \end{tabular}
  \caption{
    Comparison of event yields for signal ($N_{\mathrm{EW-ZZjj}}$) and background ($N_{\mathrm{QCD-ZZjj}}$) processes, 
    and expected significance of EW-$ZZjj$ processes,
    normalized to 3000~\ifb{} data at 14~\TeV{},
    with baseline and alternative selections.
    Uncertainties in the table refer to expected data statistical uncertainty at 14~\TeV{} with 3000~\ifb{}.
  }
  \label{tab:event_yield}
\end{table}
From this table, one can see the extended track coverage increases the \lllljj events by 15$~$30\%, by improving the lepton efficiency.
But the significance of searching for EW-$ZZjj$ process does not improve so much due to the large increment of background events.

Figure~\ref{fig:kine} shows the kinematic distributions of dijet invariant mass (\mjj), the ZZ invariant mass (\mzz) and 
the $\phi$ separation of two Z bosons ($|\Delta\phi(ZZ)|$) as well as the centrality of the ZZ system.
The ZZ centrality is defined as:
\begin{equation}
  ZZ~\text{centrality} = \frac{|y_{ZZ} - (y_{j1} + y_{j2})/2|}{|y_{j1} - y_{j2}|}
\end{equation}
To measure the event yield, the top panel shows the stack distribution for EW- and QCD-$ZZjj$ processes,
while bottom panel is the ratio between EW and QCD.
\begin{figure}[!htbp]
\centering
\subfloat[]{
\includegraphics[width=0.42\textwidth]{figures/VBSZZ/hllhc/TagJJM_final_noshape_0_ratio.pdf}
}
\subfloat[]{
\includegraphics[width=0.42\textwidth]{figures/VBSZZ/hllhc/MVV_noshape_0_ratio.pdf}
}
\\
\subfloat[]{
\includegraphics[width=0.42\textwidth]{figures/VBSZZ/hllhc/dPhiZZ_noshape_0_ratio.pdf}
}
\subfloat[]{
\includegraphics[width=0.42\textwidth]{figures/VBSZZ/hllhc/ZZCen_noshape_0_ratio.pdf}
}
\caption{
Detector-level distributions of EW- and QCD-$ZZjj$ processes with selected events in defined phase space at 14~\tev~of 
(a) \mjj,
(b) \mzz,
(c) $|\Delta\phi(ZZ)|$,
(d) ZZ centrality,
normalized to 3000~\ifb{}.
}
\label{fig:kine}
\end{figure}


\subsection{Systematics}

According to studies in section~\ref{sec:systematics}, the dominanted systematic in \llll channel is from theoretical systematic for QCD-$ZZjj$ background process.
Different sizes of systematics have been tested, at 5, 10 and 30\% level on background modeling.
The 5\% uncertainty is an optimal estimation when there is enough data events from QCD-enrich control region at HL-LHC to constrain the theoretical modeling on QCD-$ZZjj$ process.
The 30\% one is a conservative estimation, in which the uncertainties are calculated from different PDF sets and QCD renormalization and factorization
scales, following recommendation from PDF4LHC mentioned in section~\ref{sec:systematics}.

For experimental sources, the jet systematic has been checked following the setting provided by HL-LHC in Ref.\cite{ATL-PHYS-PUB-2016-026},
and the uncertainties are within 5\% level, which is smaller than run-2 measurement at 10\%.
Figure~\ref{fig:jet_uncer} depicts the up and down variations for jet uncertainty provided by HL-LHC performance tool as function of dijet invariant mass.
\begin{figure}
  \centering
  \includegraphics[width=0.42\textwidth]{figures/VBSZZ/hllhc/Uncer_baseline_TagJJM_ewk_linear.pdf}
  \includegraphics[width=0.42\textwidth]{figures/VBSZZ/hllhc/Uncer_baseline_TagJJM_qcd_linear.pdf}
  \caption{Jet variations on $\mjj$ distribution for EW-ZZjj (left) and QCD-ZZjj (right) processes
           with luminosity of 3000~\ifb at 14~\tev.
	   \textit{Upgrade Performance Function} is used to extract the uncertainties with \textit{baseline} setting.}
  \label{fig:jet_uncer}
\end{figure}
Therefore, a conservative 5\% uncertainty is used as experimental uncertainty.

The final results rely largely on the uncertainties, especially the theoretical uncertainties on QCD-$ZZjj$ production.
So results with different uncertainty conditions will be shown:
\begin{itemize}
	\item The case with statistical uncertainty of simulated samples only.
	\item The case with statistical and experimental uncertainties (5\%)
	\item The case with statistical, experimental and additional theoretical uncertainties at 5\%, 10\% and 30\% respectively.
\end{itemize}
Three different sources of uncertainties are treated as uncorrelated.

\subsection{Results}

In this analysis, the expected significance of EW-$ZZjj$ production is calculated as:
\begin{equation}
  \text{Significance} = \frac{S}{\sqrt{\sigma(B)_{stat.}^2 + \sigma(B)_{syst.}^2}},
\end{equation}
where $S$ presents the number of selected signal events,
and $\sigma(B)_{stat.}$ and $\sigma(B)_{syst.}$ denote the statistical and systematic (exp. + theo.) uncertainties from background processes.
The statistical is computed from expected yield at 3000~\ifb.

Base on baseline selection of $\mjj > 600~\gev$, a additional scan over different \mjj cuts are performed with a step of 50~\gev
for luminosity of 3000~\ifb under different systematic conditions, as shown in figure~\ref{fig:mjj_scan}.
\begin{figure}[!htbp]
\centering
\includegraphics[width=0.48\textwidth]{figures/VBSZZ/hllhc/significance_noshape_0_noratio.pdf}
\caption{
The expected significance of EW-ZZjj processes as a function of different \mjj cut with 3000~\ifb,
under conditions of different sizes of theoretical uncertainties on the QCD-ZZjj background modelling.
The statistical uncertainty is estimated from expected data yield at 14~\TeV{} with 3000~\ifb.
Different uncertainties are summed up quadratically.
}
\label{fig:mjj_scan}
\end{figure}

In addition, the expected differential cross section of EW-$ZZjj$ process is measured in the defined phase space at 14~\tev,
as a function of \mzz and \mjj, shown in figure~\ref{fig:xs_mjj_mzz}.
\begin{figure}[!htbp]
\centering
\includegraphics[width=0.48\textwidth]{figures/VBSZZ/hllhc/TagJJM_final_all_linear.pdf}
\includegraphics[width=0.48\textwidth]{figures/VBSZZ/hllhc/MZZ_all_linear.pdf}
\caption{
The projected differential cross-sections at 14~\TeV{} for the EW-ZZjj processes as a function of \mjj (left) and \mzz (right).
The top panel shows measurement with statistical only case,
where statistical uncertainty is estimated from expected data yield at 14~\TeV{} with 3000~\ifb.
The bottom panel shows impact of different sizes of systematic uncertainties.
}
\label{fig:xs_mjj_mzz}
\end{figure}
The expected differential cross sections are calculated as:
\begin{equation}
\begin{split}
  \sigma = \frac{N_{pseudo-data} - N_{QCD-ZZjj}}{L*C_{EW-ZZjj}}\\
  C_{EW-ZZjj} = \frac{N_{EW-ZZjj}^{det.}}{N_{EW-ZZjj}^{part.}}
\end{split}
\end{equation}
where $N_{pseudo-data}$ denotes the expected number of data events with 3000~\ifb{} luminosity at 14~\tev,
and $N_{QCD-ZZjj}$ and $N_{EW-ZZjj}$ are the number of predicted events of QCD-ZZjj and EW-ZZjj processes in particle-level.
The $C_{EW-ZZjj}$ factor represents the detector efficiency for EW-ZZjj processes introduced in section~\ref{sec:cf}.
The interference between EW- and QCD-ZZjj processes is ignored due to its minor contribution.

The number of expected integrated cross section as well as its uncertainty under different uncertainty conditions are shown in table~\ref{tab:xsec}
in 3000~\ifb luminosity at 14~\tev.
The statistical uncertainty is at 10\% level when with such large luminosity.
The result is dominanted by systematics and can reach 100\% level when theoretical modeling uncertainty is 30\% for QCD-$ZZjj$ processees.
\begin{table}[htbp]
  \small
  \centering
  \begin{tabular}{c|c|c|c|c|c|c}
    \hline
     & Cross section [fb] & Stat. only & Plus exp. & Plus $5\%$ theo. & Plus $10\%$ theo. & Plus $30\%$ theo. \\
    \hline
    EW-ZZjj & 0.21 & $\pm0.02$ & $\pm0.04$ & $\pm0.05$ & $\pm 0.08$ & $\pm 0.21$ \\
    \hline
  \end{tabular}
  \caption{
  Summary of expected cross-section measurements with different theoretical uncertainties.
  The statistical uncertainty is estimated from expected data yield at 14~\TeV{} with 3000~\ifb.
  Different uncertainties are summed up quadratically.
  }
  \label{tab:xsec}
\end{table}

