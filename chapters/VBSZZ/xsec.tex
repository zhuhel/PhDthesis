\section{Measurement of fiducial cross section}
\label{sec:xsec}

The fiducial cross section for the production of inclusive $ZZjj$, which includes both EW and QCD components, is then measured.

The defination of fiducial volume, which is used for cross section measurement, follows closely to the detector-level selection
but use physics objects in "particle-level", which are reconstructed in simulation from stable final-state particles,
prior to their interactions with the detector.
For electrons and muons, QED final-state radiation is for the most part recovered 
by adding the four-momenta of surrounding photons that are not originating from hadrons and within an angular distance $\Delta R < 0.1$
to the lepton four-momentum, called lepton "dressing".
Particle-level jets are built with anti-$k_{T}$ algorithm with radius parameter $R = 0.4$ using all final-state particles except leptons and neutrinos as inputs.
Comparing to the events selection in detector-level in section~\ref{sec:selection},
in particle-level, the selected dilepton pair mass required is relaxed to be within 60 to 120~\gev to reduce the migration effect
as well as be more compatibility with previous CMS publication\cite{2017682}.
All the other kinematics selection requirements are the same as the defination in detector-level.

\subsection{Calculation of C-factor}
\label{sec:cf}

C-factor is defined as the ratio between the number of selected events in detector-level and the number of particle-level events in fiducial volume (FV):
\begin{equation}
	\mathcal{C} = \frac{N_{detector-level}}{N_{FV.}}
\end{equation}
The C-factor value of each $ZZjj$ processes calculated from each individual simulation samples are listed in table~\ref{tab:xs_cf} as well as their systematics.
\begin{table}[H]
\begin{center}
   \begin{tabular}{|c|c|c|c|c|}
   \hline
   Process          & $\mathcal{C}$ & $\Delta$C(stats) & $\Delta$C(sys)        & $\Delta$C(theo)       \\
   \hline
   EWK ZZjj         & 0.663         & $\pm$0.002       & $\pm^{0.032}_{0.031}$ & NA                    \\
   \hline
   QCD \qqZZ        & 0.702         & $\pm$0.003       & $\pm^{0.061}_{0.051}$ & $\pm^{0.015}_{0.018}$ \\
   \hline
   QCD \ggZZ        & 0.741         & $\pm$0.021       & $\pm^{0.143}_{0.072}$ & $\pm{0.002}$          \\
   \hline
\end{tabular}
\end{center}
\caption{C Factor of different $ZZjj$ processes.}
\label{tab:xs_cf}
\end{table}

Then the $\mathcal{C}$ from different processes are combined together to be used as inputs for cross section calculation:
\begin{equation}
	\mathcal{C} = \Sigma_{i} \frac{N_{FV.}^{i}}{\Sigma_{j} N_{FV.}^{j}} \times \mathcal{C}_{i} = 0.699\pm0.003(stats.)\pm^{0.011}_{0.013}(theo.)\pm0.028(exp.)
\end{equation}
The stats. refers to the statistical uncertainty from MC simulation statistics.
The theo. and exp. denote the theoretical and experimental uncertainties described in section~\ref{sec:systematics}.

\subsection{Result of fiducial cross section}

The cross section in fiducial volume is computed as:
\begin{equation}\label{eq:xs}
	\sigma^{FV.} = \frac{N_{data} - N_{bkg}}{\mathcal{C} \times Lumi}
\end{equation}
where $N_{data}$ and $N_{bkg}$ denote the number of events selected from detector-level selection from data and sum of backgrounds,
and $\mathcal{C}$ is the C-factor calculated above, Lumi represents the integrated luminosity of data 2015$~$2018 of 139~\ifb.

As shown in table~\ref{tab:yield_prefit}, in inclusive measurement, only "Others" represents background, 
processes of EW-ZZjj, QCD-\qqZZ and QCD-\ggZZ are signals.
Table~\ref{tab:xs} shows the fiducial cross section for \llll channel measured from equation~\ref{eq:xs}, 
as well as the predicted cross section measured from signals MC directly.
\begin{table}[!htbp]
\begin{center}
\scalebox{0.90}{
\begin{tabular}{ c | c}
\hline
\hline\noalign{\smallskip}
Measured fiducial $\sigma$ [fb] & Predicted fiducial $\sigma$ [fb] \\
\noalign{\smallskip}\hline\noalign{\smallskip}
$1.27 \pm 0.12(\mathrm{stat}) \pm 0.02(\mathrm{theo}) \pm 0.07(\mathrm{exp}) \pm 0.01(\mathrm{bkg}) \pm 0.03(\mathrm{lumi})$ & $1.14 \pm 0.04(\mathrm{stat}) \pm 0.20(\mathrm{theo})$ \\
\noalign{\smallskip}\hline
\hline
\end{tabular}}
\end{center}
\caption{
Measured and predicted fiducial cross-sections in \lllljj final-state.
Uncertainties due to different sources are presented.
}
\label{tab:xs}
\end{table}
The measured cross section has a total uncertainty of 11\%, and is found to be compatible with SM prediction.
The data statitic is still dominanted for the measurement.
