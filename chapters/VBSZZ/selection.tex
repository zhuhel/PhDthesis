\section{Objects and Event selection}
\label{sec:vbszz_selection}

\subsection{Objects selection}

The selection of analysis relies on the definition of multiple objects: \textit{electrons}, \textit{Muons}, and \textit{jets}.
Details of definition for each object are described as below:

\textbf{Muon:} 
To increase the acceptance range in reconstruction (reco) -level for \llll channel, all four types of muons 
(CB, ST, CT, ME muons, described in section~\ref{sec:muon}) are used.
The identified muons are then required to pass $p_{T} > 7~\gev$ and $|\eta| < 2.7$,
and satisfy the \textit{Loose} identification criterion (see definition in sec~\ref{sec:muon}).
The impact parameter cuts are further applied to suppress the contribution from cosmic muons and non-prompt muons,
with the value of: $|d_{0}/\sigma(d_{0})| < 3.0$ and $|z_{0} sin\theta|$ < 0.5 mm,
where $d_{0}$ is the transverse impact parameter relative to the beam line, $\sigma(d_{0})$ is its uncertainty, 
and $z_{0}$ is the longitudinal impact parameter relative to the primary vertex.
In order to avoid muons associated with jets, all muons are required to be isolated and pass \textit{FixedCutLoose} isolation criteria,
which required $E_{T}^{topocone20} / p_{T} < 0.3$ and $p_{T}^{varcone30} / p_{T} < 0.15$.

\textbf{Electron:} 
As described in section~\ref{sec:electron}, electrons are reconstructed from energy deposits in the EM calorimeter matched to a track in the inner detector.
The electron candidates must satisfy the \textit{Loose} criterion valuing by the likelihood-based (LH) method.
And electrons are required to have $p_{T} > 7~\gev$ and $|\eta| < 2.47$.
Moreover, the impact parameter requirements of $|d_{0}/\sigma(d_{0})| < 5.0$ and $|z_{0} sin\theta|$ < 0.5 mm are applied.
Same as muon, all electrons are required to satisfy \textit{FixedCutLoose} isolation criteria
of $E_{T}^{topocone20} / p_{T} < 0.2$ and $p_{T}^{varcone20} / p_{T} < 0.15$.

\textbf{Jets:} 
Jet are key signatures for VBS processes. 
This analysis use the jets clustered using the anti-$k_t$ algorithm with radius parameter $R$ = 0.4, more details of jets' reconstruction can be found in section~\ref{sec:jet}.
The jets are required to satisfy $\pT >$ 30 (40)~\gev~in the $|\eta| <$ 2.4 ($2.4 < |\eta| < 4.5)$ region.
To further reduce the effects of pile-up jets, a jet vertex tagger (JVT) is applied to jets with $p_{T} <$ 60~\gev~and $|\eta| < 2.4$ to select jets from hard-scattering vertex~\cite{PERF-2014-03}.

\textbf{Overlap removal:} 
An overlap-removal procedure is applied to selected leptons and jets in this analysis.
To enhance the selection efficiency, leptons are given higher priority to be kept when overlapping with jets.
With this lepton preferred method, the events of EW signal after selection increases about 19\% while background only increases about 14\%.
More details of the strategy is summarized in table~\ref{tab:OR_4l}.
\begin{table}[htbp]
\begin{center}
\renewcommand\arraystretch{1.8}
\resizebox{\linewidth}{!}{
\small
\begin{tabular}{ l | c | c}
\hline
\centering & Reference objects & Criteria\\
\hline \hline
Remove electrons & electrons & Share a track or have overlapping calorimeter cluster. Keep higher \pt electron \\
\hline
Remove muons & electrons & Share track and muon is calo-tagged \\
\hline
Remove electrons & muons & Share track \\
\hline
\multirow{3}{*}{Remove jets} & electrons & $\Delta R_{e-jet}$ < 0.2  \\
\cline{2-3}
                   & \multirow{2}{*}{muons}     & $\Delta R_{\mu-jet}$ < 0.2 OR muon track is ghost-associated to jet \\
                   &       &  \textbf{AND} ($N_{Trk}(jet)<3$ OR ($p_T^{jet}/p_T^{\mu}<2$ and $p_T^{\mu}/\Sigma_{TrkPt}>0.7$)) \\
\hline\hline
\end{tabular}}
\end{center}
\caption{Overlap removal criteria between pre-selection objects for the \llll channel. The overlap removal follows the order shown in this table. Once an object has been marked as removed, it does not participate in the subsequent stages of the overlap removal procedure. }
\label{tab:OR_4l}
\end{table}

%% ========================================================================
\subsection{Event selection in reconstruction level}

The events are required to additionally be recorded by single or multi-lepton triggers, with transverse momentum ($p_{T}$) thresholds varying from 8 to 26~\gev.
The overall trigger efficiency for selected inclusive \lllljj signal events in the analysis region are from 95 to 99\%.

The \llll quadruplets are formed by two opposite-sign, same-flavour (OSSF) lepton pairs ($\ell^{+}\ell^{-}$),
in which leptons are required to be separated by $\Delta R > 0.2$ in table~\ref{tab:OR_4l}.
At most one muon is allowed to be ME or CT muon.
The $\pT$ threshold of first three leading leptons are 20, 20 and 10~\gev.
If more than one quadruplets are found, the one with minimum sum of difference between two dilepton pair masses and Z boson mass 
($|m_{l_{1}^{+}l_{1}^{-}} - m_{Z}| + |m_{l_{2}^{+}l_{2}^{-}} - m_{Z}|$) is selected.
Both two dilepton pair masses are required to be between 66 to 116~\gev.
In addition, the invariant masses of all possible OSSF pairs are required to be greater than 10~\gev~ to reject events from $J/\phi$ or $\Upsilon$ decay.

For VBS topology, the two most energetic jets in different detector side ($y_{j1} \times y_{j2} < 0$) are selected.
Furthermore, the invariant mass of two jets ($\mjj$) is required to be greater than 300~\gev, 
while $\dyjj$ is required to be larger than 2.
Table~\ref{tab:selection_reco} summarizes the above selection requirements, which is defined as signal region (SR).

\begin{table}[!htbp]
\begin{center}
\scalebox{0.75} {
\begin{tabular}{c c}
\hline
\hline \noalign{\smallskip}
\multirow{2}{*}{Electrons}     & $\pT >$ 7~\GeV{}, $|\eta| <$ 2.47            \\
                     & $|d_0/\sigma_{d_0}|<5$ and $|z_0\times\sin\theta|<0.5$ mm                                                             \\
\noalign{\smallskip}\hline\noalign{\smallskip}
\multirow{2}{*}{Muons}         & $\pT >$ 7~\GeV{}, $|\eta| <$ 2.7             \\
                     & $|d_0/\sigma_{d_0}|<3$ and $|z_0\times\sin\theta|<0.5$ mm                                                              \\
\noalign{\smallskip}\hline\noalign{\smallskip}
Jets                 & $\pT >$ 30 (40)~\GeV{} for $|\eta| <$ 2.4 ($2.4<|\eta|<4.5$)       \\
\noalign{\smallskip}\hline\noalign{\smallskip}
\multirow{5}{*}{$ZZ$ selection}  & $\pT >$ 20, 20, 10~\GeV{} for the leading, sub-leading and third leptons      \\
                     & Two OSSF lepton pairs with smallest $|m_{\ell^+\ell^-} - m_Z| + |m_{\ell^{'+}\ell^{'-}} - m_Z|$      \\
                     & $m_{\ell^+\ell^-} >$ 10~\GeV{} for lepton pairs                                            \\
                     & $\Delta R(\ell,\ell') >$ 0.2                                                               \\
                     & 66 $< m_{\ell^+\ell^-} <$ 116~\GeV{}                                                       \\
\noalign{\smallskip}\hline\noalign{\smallskip}
\multirow{2}{*}{Dijet selection}  & Two most energetic jets with $y_{j_1} \times y_{j_2} < 0$                                \\
                     & $\mjj >$ 300~\GeV{} and $\dyjj >$ 2                                                         \\
\noalign{\smallskip}\hline
\hline
\end{tabular}}
\end{center}
\caption{Summary of selection of physics objects and candidate events at detector level in the \lllljj signal region.}
\label{tab:selection_reco}
\end{table}
%Besides, a QCD Control region (CR) is defined by reverting either the $\mjj$ or $\dyjj$ requirements. 

