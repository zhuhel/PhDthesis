\section{Objects and Event selection}

\subsection{Objects defination}

The selection of analysis relies on the defination of multiple objects: \textit{electrons}, \textit{Muons}, and \textit{jets}.
Details of definations for each objects are described as below:

\textbf{Muon:} 
To increase the acceptance range in reco-level for \llll jj channel, all four types of muons 
(CB, ST, CT, ME muons, described in section~\ref{sec:muon}) are used.
The identified muons are then required to pass $p_{T} > 7 \gev$ and $|\eta| < 2.7$,
and satisfy the \textit{Loose} identification criterion (see defination in sec~\ref{sec:muon}).
The impact parameter cuts are further applied to suppress the contribution from cosmic muons and non-prompt muons,
with the value of: $|d_{0}/\sigma(d_{0})| < 3.0$ and $|z_{0} sin\theta| < 0.5 mm$.
In order to avoid muons associated with jets, all muons are required to be isolated that pass \textit{FixedCutLoose} isolation criteria.

\textbf{Electron:} 
As described in section~\ref{sec:electron}, electrons are reconstructed from energy deposits in the EM calorimeter matched to a track in the inner detector.
The electron candidates must satisfy the \textit{Loose} criterion valuing by the likelihood-based (LH) method.
And electrons are required to have $p_{T} > 7 \gev$ and $|\eta| < 2.47$.
Moreover, the impact parameter requirements of $|d_{0}/\sigma(d_{0})| < 5.0$ and $|z_{0} sin\theta| < 0.5 mm$ are applied.
