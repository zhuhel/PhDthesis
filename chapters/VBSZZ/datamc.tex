\section{Data and MC samples}

\subsection{Data samples}
\label{sec:vbszz_data}

The datasets for this analysis include the full run-2 pp collision data collected by the ATLAS experiment during the years from 2015 to 2018.
Data event is only used if it passed the latest Good Run List (GRL) released by the Data Quality group from ATLAS experiment,
corresponding to an integrated luminosity of $139.0~\pm~2.4$~\ifb.


%{\footnotesize
%\begin{verbatim}
%data15_13TeV.periodAllYear_DetStatus-v89-pro21-02_Unknown_PHYS_StandardGRL_All_Good_25ns.xml
%data16_13TeV.periodAllYear_DetStatus-v89-pro21-01_DQDefects-00-02-04_PHYS_StandardGRL_All_Good_25ns.xml
%data17_13TeV.periodAllYear_DetStatus-v99-pro22-01_Unknown_PHYS_StandardGRL_All_Good_25ns_Triggerno17e33prim.xml
%data18_13TeV.periodAllYear_DetStatus-v102-pro22-04_Unknown_PHYS_StandardGRL_All_Good_25ns_Triggerno17e33prim.xml
%\end{verbatim}
%}

%% =======================================================
\subsection{MC simulations}
\label{sec:mc}

The EW-$ZZjj$ production is modelled using \MGMCatNLO~2.6.1~\cite{Alwall:2014hca} with the matrix elements (ME) calculated in the leading-order (LO) approximation
in perturbative QCD (pQCD) and with the NNPDF2.3LO~\cite{Ball:2012cx} parton distribution functions (PDF).
The VBF Higgs process is also included.

The QCD-$ZZjj$ production is modelled using \textsc{Sherpa} 2.2.2~\cite{Gleisberg:2008ta} with the NNPDF3.0NNLO~\cite{ball2015parton} PDF,
where events with up to one (three) outgoing partons are generated at NLO (LO) in pQCD.
The production of $ZZjj$ from the gluon-gluon initial state with a four-fermion loop or with an exchange of the Higgs boson has an order of $\alpha_{S}^{4}$ in QCD,
and is not included in the \textsc{Sherpa} simulation.
A separate $gg$ induced $ZZ$ + 2jets sample is modelled using \textsc{Sherpa} 2.2.2 with the NNPDF3.0NNLO PDF
and with an additional 1.7 k-factor~\cite{PhysRevD.92.094028} being applied.
Then the interference between EW- and QCD-$ZZjj$ is modelled with \MGMCatNLO~2.6.1 calculated at LO.

The diboson productions from QCD $WW \rightarrow \ell \nu qq$ as well as QCD and EW $WZ \rightarrow \ell\ell qq$ are modelled using \textsc{Sherpa} 2.2.2 with the NNPDF3.0NNLO PDF.
The productions of semileptonic decays ($WW \rightarrow \ell\nu qq$ and $WZ \rightarrow qq\ell\ell$) are modelled using \textsc{Powheg-Box}~v2~\cite{Frixione:2007nw} with the CT10 PDF~\cite{Lai:2010vv}.
%Other diboson processes are not included due to negligible contributions.
The triboson production is modelled using \textsc{Sherpa} 2.2.2 with the NNPDF3.0NNLO PDF.

For top-quark pair (\ttbar) production, the \textsc{Powheg-Box}~v2 is used with the CT10 PDF.
The single top-quark production in $t$-channel, $s$-channel and $Wt$-channel are simulated using the \textsc{Powheg-Box}~v1 event generator~\cite{Alioli:2009je,Frederix:2012dh,Re:2010bp}.
The productions of \ttbar~in association with vector boson(s) ($ttV$) are modelled with \MGMCatNLO~2.3.3 for $ttW$ and $ttZ$ with $Z \rightarrow \nu\nu/qq$ decays,
with \textsc{Sherpa} 2.2.1 for $ttZ$ where the $Z$ decays to dilepton,
and with \MGMCatNLO~2.2.2 for $ttWW$ respectively.

The \Zjet processes are modelled using \textsc{Sherpa} 2.2.1 with the NNPDF3.0NNLO PDF, 
in which the ME is calculated for up to two partons with next-to-leading-order (NLO) accuracy in pQCD and up to four partons with LO accuracy.

For all the samples except those from \textsc{Sherpa}, 
the parton showering is modelled with \textsc{Pythia8}~\cite{Sjostrand:2007gs} using the NNPDF2.3~\cite{Ball:2012cx} PDF set,
and the A14 set of tuned parameters~\cite{ATL-PHYS-PUB-2014-021}.
While for \textsc{Sherpa} samples, the parton showering is simulated within the programme.

All simulated events are processed with detector response simulation based on \textsc{Geant4} described in section~\ref{sec:simulation_framework}.
In addition, simulated inelastic pp collisions are overlaid to model additional pp collision in the same and neighbouring bunch crossings (pile-up),
and reweighted to match the pile-up conditions in data.
Moreover, all simulated events are processed using the same reconstruction algorithms as data.
And the leptons and jets reconstruction, energy scale and resolution, and the leptons identification, isolation, trigger efficiencies for simulated events,
as described in section~\ref{sec:reconstruction}, are all corrected to match the data measurements.
