\chapter{The Large Hadron Collider and the ATLAS Detector}

\section{The Large Hadron Collider}
Located near the French-Swiss border at the European Organization for Nuclear Research (CERN),
the Large Hadron Collider (LHC) is the largest and most powerful facility for particle physics in the world.
It's the proton-proton collider with the centre-of-mass energy designed up to 14~\tev.
The beams inside the LHC are made to collide at four locations around its 27-kilometer accelerator ring, 
corresponding to four particle experiments - the ATLAS, CMS, ALICE and LHCb.
With its unprecedented energy, the LHC is designed to observe physics that involve highly massive particles,
which have never been observed in previous accelerators with lower energies.

\subsection{Operation history and machine layout}

%% ================================= history ==============================
\textbf{Operation history}

The LHC \cite{Bruning:2004ej, Buning:2004wk, Benedikt:2004wm, Evans_2008} 
is a two-ring-superconducting-hadron accelerator and collider lies in a tunnel about 100 metres underground.
It's designed to provide proton-proton collisions at the centre-of-mass energy up to 14~\tev~
with a unprecedented luminosity of $10^{34} cm^{-2} s^{-1}$.
In the meantime, it can also collide heavy (Pb) ions with an energy of 2.8~\tev~ per nucleon and a peak luminosity of $10^{27} cm^{-2} s^{-1}$.
Table~\ref{tab:LHC_parameters} shows the main design parameters of the LHC for proton-proton collisions.
\begin{table}[htbp]
  \centering
  \caption{Summary of design parameters of the LHC for pp collisions.}
  \label{tab:LHC_parameters}
  \begin{tabular}{cc}
    \hline
    Circumference	& 26.7 km\\
    Beam energy at collision	& 7 ~\tev\\
    Beam energy at injection	& 0.45~\tev \\
    Dipole field at 7~\tev	& 8.33 T \\
    Luminosity		& $10^{34} cm^{-2} s^{-1}$ \\
    Beam current	& 0.56 A \\
    Protons per bunch	& $1.1 \times 10^{11}$ \\
    Number of bunches	& 2808 \\
    Nominal bunch spacing	& 24.95 ns \\
    Normalized emittance	& 3.75 $\mu$m \\
    Total crossing angle	& 300 $\mu$rad \\
    Energy loss per turn	& 6.7 keV \\
    Critical synchrotron energy	& 44.1 eV \\
    Radiated power per beam	& 3.8 kW \\
    Stored energy per beam	& 350 MJ \\
    Stored energy in magnets	& 11 GJ \\
    Operating temperature	& 1.9 K \\
    \hline
  \end{tabular}
\end{table}

The LHC was built from 1998 to 2008. 
It started its first beam in September 2008, but then was interrupted by a quench incident only after a few days running.
Then it resumed the operation in November 2009 with a low energy beams.
From March 2010, physics runs took place at the centre-of-mass energy of 7~\tev.
Later on, this energy was increased in 2012 to $\sqrt{s} = 8~\tev$, with an integrated luminosity of 20.3~\ifb, and this period is called ``run-1".
After run-1, the LHC was shut down for two years for hardware maintenance and upgrade, starting from February 2013.

The second operation period with higher centre-of-mass energy at 13~\tev~ started from 2015 called ``run-2".
And it continued to the end of 2018 with total integrated luminosity reaching about 147 $fb^{-1}$ for ATLAS experiment.
Figure~\ref{fig:lumi_vs_month} shows the cumulative luminosity as a function of time in month delivered to ATLAS experiment during stable beams 
in years from 2011 to 2018.
\begin{figure}[!htb]
  \centering
  \includegraphics[width=0.6\textwidth]{figures/Detector/intlumivsyear.pdf}
  \caption{Cumulative luminosity vs time in the years from 2011 to 2018 for ATLAS detector.}
  \label{fig:lumi_vs_month}
\end{figure}

%% ======================================= layout ==============================
\textbf{Machine layout}

The layout of CERN accelerator complex is shown in figure~\ref{cern_layout}.
The protons are accelerated by a series of machines before being injected into the main ring.
At beginning, the 50~\mev~ protons are produced in the linear particle accelerator LINAC2, 
and further accelerated to 1.4~\gev~ in Proton Synchrotron Booster (PSB).
The protons are then injected into the Proton Synchrotron (PS) to gain the energy of 26~\gev~ and further accelerated to 450~\gev~ in Super Proton Synchrotron (SPS).
At the end, they are injected into the main ring, and can reach a maximum energy of 7~\tev.

\begin{figure}[!htb]
  \centering
  \includegraphics[width=1.0\textwidth]{figures/Detector/LHC_v2017.png}
  \caption{Layout of CERN LHC complex \cite{Mobs:2197559}.}
  \label{cern_layout}
\end{figure}

The collisions can occur in 4 points, with corresponding 4 major detector experiments that are briefly described as follows:
\begin{itemize}
	\item \textbf{ATLAS}: A Toroidal LHC ApparatuS, one general-purpose particle detector experiment and the detector with largest volume at the LHC. It is designed to search for the Higgs boson, test the stardand model of particle physics and search for possible beyond SM physics.
	\item \textbf{CMS}: Compact Muon Solenoid, another large general-purpose particle physics detector, with the same physics goal as ATLAS and also cross check with ATLAS.
	\item \textbf{ALICE}: A Large Ion Collider Experiment, it is optimized to study heavy-ion (Pb-Pb nuclei) collisions at a centre-of-mass energy of 2.76~\tev~ per nucleon pair.
	\item \textbf{LHCb}: Large Hadron Collider beauty, it is a specialized b-physics experiment, designed primarily to measure the parameters of CP violation in the interactions of b-hadrons.
\end{itemize}



%\subsection{Operation history and machine layout}
\subsection{Luminosity and pile-up}

\textbf{Luminosity}

In beam–beam collisions, the event rate for a process is written as~\cite{Evans_2008}:
\begin{equation}
	N = \mathcal{L} \sigma
\end{equation}
where $\sigma$ is the cross section of the process, and $\mathcal{L}$ is the luminosity.
To study rare events, $\mathcal{L}$ must be as high as possible.
The luminosity only depends on the beam parameters as:
\begin{equation} \label{eq:lumi}
	\mathcal{L} = \frac{ N_{b}^{2} n f_{r} \gamma}{4\pi \epsilon_{n} \beta^{*}}
\end{equation}
in which $N_{b}$ represents the number of particles per bunch, $n$ denotes the number of bunches per beam,
$f_{r}$ is the revolution frequency, and $\gamma$ is relativistic $\gamma$ factor, 
$\epsilon_{n}$ is the normalized transverse emittance and $\beta^{*}$ denotes the $\beta$ function at the collision point.
To reduce the beam-beam interaction effects, the bunches must have a crossing angle,
which produces a geometrical luminosity reduction factor $F$:
\begin{equation}
	F = 1 / \sqrt{1 + \left( \frac{\theta_{c}\sigma_{Z}}{2\sigma^{*}} \right) }
\end{equation}
where $\theta_{c}$ denotes the crossing angle at the interaction point, $\sigma_{Z}$ is the root mean square (RMS) bunch length
and $\sigma^{*}$ is the transverse RMS beam size at crossing point.

The luminosity expressed in Eq.~\ref{eq:lumi} is normally the instantaneous luminosity.
In fact the running conditions usually vary with time, so the luminosity can change as well.
To take into account the time dependence, integrated luminosity is invited, by integraling the instantaneous luminosity over time:
\begin{equation}
	L = \int \mathcal{L}(t) dt
\end{equation}
The unit of integrated luminosity we commonly use is $b^{-1}$ that satisfying $1 b^{-1} = 10^{24} cm^{-2}$.
Figure~\ref{fig:lumi_vs_time} shows integrated luminosity as a function of time delivered to ATLAS (green), 
recorded by ATLAS (yellow), and certified to be good quality data (blue) during run-2 pp collisions.
For most physics analysis, the data with good quality (require to satisfy \textit{Good Run List}) is used.
\begin{figure}[!htb]
  \centering
  \includegraphics[width=0.6\textwidth]{figures/Detector/intlumivstimeRun2DQall.pdf}
  \caption{Integrated luminosity vs delivered month from 2015 to 2018 in ATLAS experiment.}
  \label{fig:lumi_vs_time}
\end{figure}

\textbf{Pile-up}

In collisions, multiple interactions can happen in one single bunch crossing, which is called ``\textit{pile-up}".
The variable $\left< \mu \right>$, representing the average number of interactions per bunch crossing that used to describe pile-up effect, is defined as:
\begin{equation}
	\left< \mu \right> = \frac{L_{tot}\sigma}{f_{r}n_{bunch}}
\end{equation}
where $L_{tot}$ is the instantaneous luminosity, $\sigma$ denotes the inelastic cross section,
$f_{r}$ represents the LHC revolution frequency and $n_{bunch}$ is the number of colliding bunches.
Usually, with increasing luminosity, the pile-up becomes more significant.
Figure~\ref{fig:run2_mu} shows the luminosity-weighted distribution of the mean number of interactions per crossing
for pp collision data from 2015 to 2018 (full run-2), the challenge of pile-up increased in each year.
\begin{figure}[!htb]
  \centering
  \includegraphics[width=0.6\textwidth]{figures/Detector/mu_2015_2018.pdf}
  \caption{Number of interactions per crossing weighted bt luminosity from 2015 to 2018 in ATLAS experiment.}
  \label{fig:run2_mu}
\end{figure}

%\subsection{Luminosity and pile-up}

\section{ATLAS detector}
\subsection{Detector overview}

ATLAS (A Toroidal LHC ApparatuS) is the largest volume detector ever constructed for a particle collider.
It has the dimensions of a cylinder with 46 meters long, 25 meters in diameter, and sits in a cavern 100 meters below ground.
The detector contains about 3000km of cables and it weights 7000 tonnes.

This paragraph briefly summarizes the coordinate system and nomenclature used to describe the ATLAS detector \cite{Collaboration_2008}.
As shown in figure~\ref{fig:coordinate}, we define the nominal interaction point as the origin of the coordinate system, the beam direction as the \textit{z}-axis and the \textit{x-y} plane is transverse to the beam direction.
The positive \textit{x}-axis is defined to be the direction pointing to the center of the LHC ring, 
while the positive \textit{y}-axis is pointing upwards.
\begin{figure}[!htb]
  \centering
  \includegraphics[width=0.9\textwidth]{figures/Detector/Coordinate_system_atlas.png}
  \caption{Coordinate system used by the ATLAS experiment at the LHC \cite{Perez:phdthesis}.}
  \label{fig:coordinate}
\end{figure}
There are two sides of detector A and C, in which A(C)-side is defined as with positive (negative) \textit{z}.
The azimuthal angle $\phi$ is measured as usual around the beam axis, while the polar angle $\theta$ is the angle from the beam axis
In physics analysis, we usually use the pseudorapidity instead of $\theta$ angle, which is designed as $\eta = - ln [tan\left( \frac{\theta}{2}\right)]$. 
For massive objects (eg. jets), the rapidity $y = \frac{1}{2} ln[ \frac{E+p_{z}}{E-p_{z}}]$ is used.
In addition, the \textit{transverse} momentum $p_{T}$, \textit{transverse} energy $E_{T}$ and the missing \textit{transverse} energy $E_{T}^{miss}$ are defined in \textit{x-y} plane.
The commonly used distance measurement $\Delta R$, is defined in the pseudorapidity-azimuthal angle space as $\Delta R = \sqrt{ \Delta\eta^{2} + \Delta\phi^{2}}$.

The overall ATLAS layout is shown in figure~\ref{fig:atlas_layout}, which is forward-backward symmetric with respect to the interaction point.
\begin{figure}[!htb]
  \centering
  \includegraphics[width=1.0\textwidth]{figures/Detector/atlas_layout.jpg}
  \caption{Cut-away view of the ATLAS detector \cite{Pequenao:1095924}.}
  \label{fig:atlas_layout}
\end{figure}
The magnet configuration comprises a thin superconducting solenoid surrounding the inner-detector cavity, 
and three large superconducting toroids (one barrel and two end-caps) arranged with an eight-fold azimuthal symmetry around the calorimeters.

\textbf{The inner detector}, which is the innermost part of ATLAS, is immersed in a 2 T solenoidal magnetic field.
It's used for pattern recognition, momentum and vertex measurements and electron identification, with the combination of tracking system.

\textbf{The calorimeter} is outside the inner detector, for electromagnetic and hadronic energy measurements.
High granularity liquid-argon (LAr) electromagnetic sampling calorimeters is used to measure energy and position resolution with range up to $|\eta| < 3.2$ for electrons and photons.
For hadronic calorimetry, a scintillator-tile calorimeter is used in the range of $|\eta| < 1.7$.
The LAr forward calorimeters provide both electromagnetic and hadronic energy measurements with the coverage up to $|\eta| = 4.9$.

\textbf{The muon spectrometer} is in the outermost side.
The air-core toroid system, with a long barrel and two inserted end-cap magnets, provides strong bending power in a large volume within a light and open structure.
Multiple-scattering effects are minor, and excellent muon momentum resolution can be achieved.

%\subsection{Dector overview}
\subsection{Physics requirement}

As mentioned previously, ATLAS is one of two general-purpose particle detector experiment at LHC.
It's designed to take advange of the unprecedented energy at LHC.
The Higgs boson was discovered as one of its benchmark, and lots of precise tests and measurements of SM is on going.
In the meantime, ATLAS is also designed to abserve the phenomena that involve highly massive particles, such as heavy beyond standard model (BSM) gauge bosons $Z'$ and $W'$.
It can also explore the possibility of extra dimensions proposed by several models in TeV region.
To fulfil many diverse physics goals, a set of general requirements are needed:
\begin{itemize}
	\item The speed-fast and radiation-hard electronics are required due to the experimental conditions at LHC. 
	\item High detector granularity is needed to reduce the overlapping events and handle the particle fluxes.
	\item Large acceptance in pseudorapidity and azimuthal angle coverage is needed.
	\item For inner detector, good charged-paricle momentum resolution and reconstruction efficiency are crucial. And the vertex detectors close to the interaction region are required to be able to observe secondary vertices for offline tagging of $\tau$-lepton and $b$-jets.
	\item Good electromagnetic (EM) calorimetry for electron and photon, as well as full-coverage hadronic calorimetry for accurate jet and missing transverse energy measurements, are importantly required, since these measurements form the basis of many studies.
	\item Good muon spectrometer is also required for muon identification and momentum resolution measurement over a wide range of momenta.
	\item Highly efficient but with sufficient background rejection triggers are also needed and extramely important for objects with low transverse-momentum. 
\end{itemize}

%Table~\ref{tab:detector_goal} summarizes the major performance for different parts of ATLAS detector.
More detailed descriptions of each sub-system will be given in the following subsections.

%\subsection{Physics requirement}

\subsection{Magnet system}

A strong magnetic field is required for precise measurement of charged particle momenta.
The ATLAS detector uses two large superconducting magnet systems, a hybrid system of a central superconducting solenoid and three outer superconducting toroids, to bend charged particles~\cite{McFayden:phdthesis}.
The total magnet system is 22 m in diameter and 26 m in length as shown in figure~\ref{fig:megnet_sys}.
\begin{figure}[!htb]
  \centering
  \includegraphics[width=0.7\textwidth]{figures/Detector/magnetSystems.png}
  \caption{Schematic diagram of the ATLAS magnet system.}
  \label{fig:megnet_sys}
\end{figure}

The central solenoid produces two Tesla (T) magnetic field surrounding the inner Detector.
When obtaining such high field strength, at the same time, the solenoid needs to be thin in order to reduce the material in front of the calorimeter.

The outer toroid system comprises one barrel superconducting toroid and two end-caps.
The barrel one is composed of eight coils encased in individual racetrack-shaped, stainless-steel vacuum vessels and produces the magnetic field in the cylindrical volume surrounding the calorimeters.
Each end-cap toroid consists of a single cold mass built up from eight flat, square coil units and eight keystone wedges and provides a magnetic field of approximately 1 T for the muon detectors in the end-cap regions.

%\subsection{Magnet system}
\subsection{Inner detector}

The inner detector, as shown in figure~\ref{fig:inner_dec}, is the detector closest to beam pipe.
It's used to measure the position of charged particle tracks in high precision together with good momentum resolution,
in which the measurement of primary and secondary vertices and electron identification are especially important.
Due to the extremely high luminosity produced by the LHC, the precise measurements of vertex and momentum becomes tough and fine-granularity detectors are crucial.
The inner detector consists of three subdetectors that will be described as below.
\begin{figure}[!htb]
  \centering
  \includegraphics[width=0.8\textwidth]{figures/Detector/ID_newTRT_d3.png}
  \caption{Schematic diagram of the ATLAS inner detector\cite{Aad:1698966}.}
  \label{fig:inner_dec}
\end{figure}

%% ============================== Pixel detector ====================================
\textbf{Pixel detector}

The pixel detector is the innermost part of ATLAS tracking system.
With finest granularity of materials, it has the best spatial resolution and 3-dimensional space-point measurement in inner detector.
ATLAS Pixel Detector for the LHC run-2 is composed of 4 layers of barrel pixel detector and two end-caps with three pixel disks each, as shown in figure~\ref{fig:inner_pixel}.
There are three outer layers that originally installed for run-1 and one additional layer called Insertable B-Layer (IBL) that newly constructed in run-2~\cite{Mullier:2016}.
Now the 4-layer pixel detector has very good reconstruction of primary and secondary vertices, which is even crucial for long-lived particles like $\tau$-lepton and b-quark.
\begin{figure}[!htb]
  \centering
  \includegraphics[width=1.0\textwidth]{figures/Detector/inner_pixel.png}
  \caption{Schematic diagram of the ATLAS 4-Layer Pixel Detector.}
  \label{fig:inner_pixel}
\end{figure}

%% ================================ SCT ===========================================
\textbf{Semiconductor Tracker}

The Semiconductor Tracker (SCT) is the middle component of the inner detector that outside the pixel detector.
It has similar function as pixel detector but with long and narrow strips instead of small pixels, which makes a much larger coverage than pixel detector.
The SCT consists of 4088 modules, it contains four concentric layers in barrel (2112 modules) and nine disks in each of the two end-caps (1976 modules) as shown in figure~\ref{fig:inner_sct}.
And it measures particles over a large area with 6.3 million readout channels and a total area of 61 square meters.
The SCT is the most critical part of the inner detector for 2D track hit reconstruction.
In barrel, the hit precision is 17 $\mu$m in the \textit{r}-$\phi$ coordinate and 580 $\mu$m in \textit{z} coordinate.
In end-caps, it have accuracies of 17 $\mu$m in the \textit{z}-$\phi$ coordinate and 580 $\mu$m in \textit{r} coordinate.
\begin{figure}[!htb]
  \centering
  \includegraphics[width=0.8\textwidth]{figures/Detector/inner_SCT.png}
  \caption{SCT (a) barrel module and (b) end-cap\cite{Sultan:phdthesis}.}
  \label{fig:inner_sct}
\end{figure}

%% ================================= TRT ======================================
\textbf{Transition radiation tracker}

The transition radiation tracker (TRT)\cite{TRT_2008} is the outermost part of inner detector.
It has a very different design with the two previously sub-detectors. It's composed of thin-walled drift tubes called straw, also in three parts: a barrel and two end-cap regions.
There are 73 barrel layers and 224 end-cap layers (112 in each) with 372000 straws in total, and about 351000 readout channels for TRT.
The TRT provides better \textit{z} resolution but much worse \textit{r}-$\phi$ resolution (about 130 $\mu$m) compared to the pixel detector and SCT per straw.
But the straw hits still make significant contributions to momentum measurement, since its lower precision per point (compared to silicon) can be compensated by the large number of measurements and long track length.

%\subsection{Inner dector}
\subsection{Calorimeters}

The calorimeters are designed to measure the energy from particles by absorbing them.
They are located outside the solenoidal magnet that surrounds the inner detector.
The ATLAS calorimeters are comprised of a number of sampling calorimeters with full $\phi$-symmetry and the pseudorapidity range of $|\eta|<4.9$.
Figure~\ref{fig:calo_dec} shows the layout of the ATLAS calorimeter system.
As mentioned in overview section, there are two basic calorimeter systems: an inner electromagnetic (EM) calorimeter and an outer hadronic calorimeter.
The EM calorimeter is designed for precise measurements for electrons and photons, so that with fine granularity;
while the hardronic one with relative coarser granularity but satisfied the physics requirements for jets reconstructions and $E_{T}^{miss}$ measurements.
Two different sampling techniques are used, the EM calorimeter is purely based on liquid-argon (LAr) technology, hardronic calorimeter use both LAr and scintillating tiles calorimeters. 
More details are described as belows.
\begin{figure}[!htb]
  \centering
  \includegraphics[width=0.8\textwidth]{figures/Detector/calo_layout.png}
  \caption{Cut-away view of the ATLAS calorimeters. The LAr calorimeters are seen inside the scintillator- based Tile hadronic calorimeters\cite{Buchanan:2008}.}
  \label{fig:calo_dec}
\end{figure}

%% ================================ electromagnetic calorimeter ===================
\textbf{Liquid Argon calorimeter}

%Liquid-argon calorimeter is used for EM calorimeter in barrel and end-caps regions and for hardronic calorimeter in end-caps.
The Liquid Argon sampling calorimeter technique with "accordion-shaped" electrodes is used for all electromagnetic calorimetry covering the pseudorapidity range of $|\eta|<3.2$;
and for hadronic calorimetry from $|\eta| = 1.4$ up to the acceptance limit $|\eta| = 4.9$\cite{CERN-LHCC-96-041}.
Figure~\ref{fig:calo_lar} shows the shape of a barrel module as accordion geometry.
For barrel EM calorimeter, the absorbing material is lead-liquid argon, while the hadronic end-cap calorimeter use copper plates as the absorbing material.
In addition, the forward calorimeter is splited into three parts, an EM sector in which copper is used as absorbing material and two hadronic sectors using tungsten ouside the EM sector.
\begin{figure}[!htb]
  \centering
  \includegraphics[width=0.6\textwidth]{figures/Detector/calo_lar.png}
  \caption{Diagram of a LAr EM calorimeter barrel module\cite{Sanchez:2010}.}
  \label{fig:calo_lar}
\end{figure}


%\subsection{Calorimeters}
\subsection{Muon spectrometer}

Muon spectrometer~\cite{CERN-LHCC-97-022} is the outermost part of the ATLAS detector with an extremely large tracking system.
It measures a large range of muon momentum, and the accuracy is about 3\% at 100 GeV and 10\% at 1 TeV.
The muon spectrometer comprises three main parts: a magnetic field produced by three toroidal magnets;
a set of chambers measuring the tracks of muons with high spatial precision; and triggering chambers with accurate time-resolution. 
Figure~\ref{fig:muon_dec} shows the schematic of ATLAS muon spectrometer that consists of four types of muon chambers 
(\textit{MDT, CSC, RPC, TGC}) as well as the magnet systems (barrel and end-cap toroid).
\begin{figure}[!htb]
  \centering
  \includegraphics[width=0.8\textwidth]{figures/Detector/muon_all.png}
  \caption{Cut-away view of the ATLAS muon spectrometer\cite{Sliwa:2013oua}.}
  \label{fig:muon_dec}
\end{figure}

More details of four chambers are given as below:
\begin{itemize}
	\item \textbf{Monitored Drift Tubes (MDT)}. MDTs provide the precise momentum measurement with the $|\eta|$ range up to 2.7, except in the innermost end-cap layer where the coverage is limited to $|\eta| < 2.0$. The chambers comprises three or four layers of drift tubes, with a diameter of 29.970 mm, operated with Ar/CO2 gas (93/7) at 3 bar. The average resolution can reach 80 $\mu$m per tube and 30 $\mu$m per chamber.
	\item \textbf{Cathode strip chambers (CSC)}. CSCs are used in the forward region of $2 < |\eta| < 2.7$ in the innermost tracking layers, due to their good time resolution and high rate capability. The CSCs are multi-wire proportional chambers (MWPC) with the cathode planes segmented into strips in orthogonal directions, which allows both coordinates to be measured from the induced-charge distribution. The resolution of a chamber is about 40 $\mu$m for bending plane and 5 mm for the transverse plane.
	\item \textbf{Resistive plate chambers (RPC)}. The RPCs serves as fast triggers in the barrel region of $|\eta| < 1.05$ due to the high rate capability and good spatial and time resolution. It is a gaseous parallel electrode-plate detector without any wires. There are three concentric cylindrical layers around the beam axis, as three trigger stations. Each stations consists of two independent layers to measure the transverse coordinates of $\eta$ and $\phi$.
	\item \textbf{Thin gap chambers (TGC)}. TGCs are used as trigger system for the end-cap region of $1.5 < |\eta| < 2.4$, and works based on the same principle as multi-wire proportional chambers. In addition, they can also provide the second azimuthal coordinate to complement the measurement of MDT in bending direction.
\end{itemize}

%\subsection{Muon spectrometer}
\subsection{Trigger system}

Trigger system in ATLAS is a very essential component, which is responsible for deciding whether to keep a given collision event for later study or not.
In LHC run-2, higher energy, luminosity and pile-up lead to an large increase of event rate by up to a factor of five, which cause to a even larger challenge and more strict requirement of trigger system.

The trigger system in run-2 is comprised of a hardware-based first level trigger (Level-1) and a software-based high level trigger (HLT) \cite{Ruiz-Martinez:2133909}.
As depicted in figure~\ref{fig:trig_syst}, in Level-1, the inputs from coarse granularity calorimeter and muon detector information together with some other subsystems are sent to the Central Trigger Processor to determine Regions-of-Interest (RoIs) in the detector. 
\begin{figure}[!htb]
  \centering
  \includegraphics[width=0.9\textwidth]{figures/Detector/tdaq-run2-schematic2017.png}
  \caption{Schematic diagram of the ATLAS trigger and data acquisition system in Run-2.}
  \label{fig:trig_syst}
\end{figure}
The events rate can be reduced by Level-1 triggers from 30 MHz to 100 kHz. 
After that, the RoI information from Level-1 is sent to HLT, in which more sophisticated selection algorithms are run for regional reconstruction.
The HLT reduces the rate from Level-1 of 100 kHz to about 1 kHz on average.
At the end, the events that accepted by HLT are transfered to local storage at experimental site for offline reconstruction.
Details about Level-1 and HLT trigger systems will be described as belows.

\textbf{Level-1 trigger}

Substantial upgrades have been delivered in ATLAS Level-1 trigger system for Run-2 data taking.
The upgrades took place in both hardware and detector readout, allows the trigger rate increasing from 70 kHz (run-1) to 100 kHz (run-2).
As mentioned above, there are two major parts of Level-1 triggers, which include Level-1 calorimeter (L1calo) trigger and Level-1 muon (L1mu) trigger.

Level-1 Calorimeter trigger uses the reduced granularity information from the electromagnetic and hadronic calorimeters to search for electrons, photons, taus and jets and missing transverse energy ($E_{T}^{miss}$).
It can identify an Region-of-Interest (RoI) as a $2 \times 2$ trigger tower cluster in the EM calorimeter as shown in figure~\ref{fig:trig_tower}, 
and $4 \times 4$ or $8 \times 8$ trigger tower for Jet RoIs.
\begin{figure}[!htb]
  \centering
  \includegraphics[width=0.6\textwidth]{figures/Detector/trig_tower.png}
  \caption{An examples of L1 calorimeter trigger tower for electron and photon triggers\cite{Pasztor:2063746}.}
  \label{fig:trig_tower}
\end{figure}
One important upgrade is that, the new FPGA-based (field-programmable gate array) Multi-Chip Modules are used to replace the ASICs (application-specific integrated circuits) included in the modules used in run-1,
which allows the use of auto-correlation filters to suppress pile-up.

The Level-1 Muon trigger system includes one barrel section (RPC) and two enf-cap section (TGC), which provides fast trigger signals from the muon detectors for the Level-1 trigger decision.
By requiring a coincidence with hits from the innermost muon chambers, it can reduce the $L1_MU15$ rate by about 50\% in the region of $1.3 < |\eta| < 1.9$ while only loss around 2\% signal efficiency.
In addition, the coverage is extended by around 4\% due to installing new chambers in the feet region of the muon detector.

\textbf{High Level Trigger}

The ATLAS trigger system separated the Level-2 and Event Filter computer clusters in run-1, but for run-2, they have been merged into a single HLT event processing.
The new arrangement helps to reduce the complexity and duplication of algorithm, which leads to a more flexible high level trigger system.
During the long-shutdown between LHC run-1 and run-2, lots of re-optimizations have been done for trigger reconstruction algorithms as well as the offline analysis selections,
which can improve the efficiency by more than a factor of two in some cases like in hadronic tau triggers.
For some triggers, the HLT processing performed within RoIs can also allows to agregrate from RoIs to single objects. 
This improvement reduces the CPU processing for events with overlapping RoIs, and the average output rate has been increased from 400 Hz to 1 kHz.
The HLT reconstruction algorithm can be divided into fast and precision online reconstruction steps. 
As depicted by figure~\ref{fig:trig_alg}, the initial fast reconstruction helps to reduce the event rate early, and be seeded into precision reconstruction.
Then the final online precision reconstruction is improved and uses offline-like algorithms as much as possible.
In particular, multivariate analysis techniques (based on machine learning) have been introduce online in many aspects.
\begin{figure}[!htb]
  \centering
  \includegraphics[width=0.5\textwidth]{figures/Detector/trig_alg.png}
  \caption{ The HLT trigger algorithm sequence\cite{Pasztor:2063746}.}
  \label{fig:trig_alg}
\end{figure}

%\subsection{Trigger system}
