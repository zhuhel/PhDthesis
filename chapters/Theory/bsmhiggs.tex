\subsection{Beyond the SM Higgs sector}
\label{bsmhiggs}

After the discovery of the Higgs boson by the ATLAS and CMS Collaborations at the LHC~\cite{20121, 201230} in 2012,
one question comes out: if this Higgs boson at around 125~\gev~ is fully responsible for the unitarization of the scattering amplitudes?
%If not, additional new physics must be present to play this role.
The possibility that this discovered particle is just a part of the extended Higgs sector by various extensions cannot be ruled out.
Many models, motivated by hierarchy and naturalness arguments, 
predicted the extended Higgs sector, such as the electroweak-singlet model and the two-Higgs-doublet models (2HDM).

\textbf{Singlet scalar extension of the SM} \\
The electroweak singlet model can be considered as the minimal extension of the SM Higgs sector~\cite{Profumo_2007}, encompassing a single gauge singlet real scalar field $S$.
In this model, a heavy, real singlet is introduced in addition to the SM one.
The associated zero temperature, tree-level scalar potential can be written as:
\begin{equation}
    V = V_{SM} + V_{HS} + V_{S}
\end{equation}
where
\begin{equation}
\begin{split}
    V_{SM} =  \mu^{2} \left( H^{\dagger}H \right) + \bar{\lambda_{0}} \left( H^{\dagger}H \right) \\
    V_{HS} = \frac{a_{1}}{2}\left( H^{\dagger}H \right)S + \frac{a_{2}}{2}\left( H^{\dagger}H \right)S^{2} \\
    V_{S} = \frac{b_{2}}{2} S^{2} + \frac{b_{3}}{3} S^{3} + \frac{b_{4}}{4} S^{4} \\ 
\end{split}
\end{equation}
where $H$ stands for the SM scalar field of the original Higgs mechanism.
After electroweak symmetry breaking, this model gives rise to two $CP$-even Higgs bosons, in which the lighter one is the Higgs boson that has been discovered at around 125~\gev.
And the new heavy scalar ($S$) is allowed to have both SM and non-SM decays.
One would expect to see suppressions of the branching ratio to SM Higgs decay modes, as the branching ratio to the pair of singlet-like scalars would be considerable. 

\textbf{Two Higgs Doublet Model} \\
The two-Higgs-doublet model (2HDM)~\cite{BRANCO20121} is another extension of SM Higgs sector carried by an additional scalar doublet.
In this model, through electroweak symmetry breaking, there are five physical Higgs bosons: two CP-even, one CP-odd, and two charged ones.
The most general CP-conserving 2HDM has seven free parameters:
\begin{itemize}
    \item The masses of five Higgs bosons: $m_{h}$, $m_{H}$, $m_{A}$ and $m_{H^{\pm}}$.
    \item $tan \beta$: $v_1/v_2$, where $v_1$ and $v_2$ are the two Higgs doublets' vacuum expectation values.
    \item $\alpha$: the two neutral CP-even Higgs bosons mixing angle .
    \item $m_{12}^{2}$: the potential parameter mixing the two Higgs doublets.
\end{itemize}
where the $m_{h}$ can be identified as the mass of observed Higgs boson at around 125~\gev, and $m_{H}$ is another heavy scaler with similar properties to $h$ boson.
The coupling of the neutral Higgs bosons to either $WW$ or $ZZ$ follows the rules~\cite{BRANCO20121}: 
\begin{enumerate}
    \item The coupling of the light Higgs ($h$) equals to the Standard Model coupling times $sin(\beta - \alpha)$ 
    \item The coupling of the heavier Higgs ($H$) equals to the Standard Model coupling times $cos(\alpha - \beta)$.
    \item The coupling of the pseudoscalar ($A$) to vector bosons is zero.
\end{enumerate}
The two Higgs doublets, $\Phi_{1}$ and $\Phi_{2}$, can couple to fermions (leptons and up- and down-type quarks) in several ways, which leads to several types of 2HDM models:
\begin{itemize}
    \item Type-\uppercase\expandafter{\romannumeral1} model: all quarks and leptons couple only to $\Phi_{2}$.
    \item Type-\uppercase\expandafter{\romannumeral2} model: down-type quarks and leptons couple to $\Phi_{1}$, and up-type quarks couple to $\Phi_{2}$.
    \item The ``lepton-specific" model: leptons couple to $\Phi_{1}$, while all quarks couple to $\Phi_{2}$.
    \item The ``flipped" model: down-type quarks couple to $\Phi_{1}$, while up-type quarks and leptons couple to $\Phi_{2}$.
\end{itemize}
