\section{Data and MC samples}

\subsection{Data samples}

The data used in theses searches are collected by ATLAS detector at the centre-of-mass energy of 13 TeV during the year of 2015 to 2018.
Only events pass the latest Good Run List (GRL) released by the Data Quality group from ATLAS experiment as listed in section~\ref{sec:vbszz_data} 
corresponding to an integrated luminosity of $139.0~\pm~2.4$~\ifb are used.
Table~\ref{tab:data_info} listed the recorded integrated luminosity, average and peak pile up of each year's data.
\begin{table}[htbp]
  \centering
  \caption{Summary of the recorded integrated luminosity (lumi), average and peak pile up (PU) of data from 2015 to 2018.}
  \label{tab:data_info}
  \begin{tabular}{ccccc}
    \toprule
    Year & recorded integrated lumi  & lumi after GRL & average PU & peak PU  \\
    \midrule
    2015 & 3.86~\ifb & \multirow{2}{*}{36.2~\ifb} & 13.4 & 28.1 \\
    2016 & 35.6~\ifb & & 25.1 & 52.2 \\
    2017 & 46.9~\ifb & 44.3~\ifb & 37.8 & 79.8 \\
    2018 & 60.6~\ifb & 58.5~\ifb & 36.1 & 88.6 \\
    \bottomrule
  \end{tabular}
\end{table}

\subsection{Background MC simulations}

Background processes considered in this analysis include $ZZ$ (\qqZZ, \ggZZ), triboson ($WWZ$, $WZZ$, $ZZZ$), \Zjet and top-quark (\ttbar, ttV) processes.
The QCD \qqZZ process is modelled using \textsc{Sherpa} 2.2.2~\cite{Gleisberg:2008ta} with the NNPDF3.0NNLO~\cite{ball2015parton} PDF,
in which events with up to one (three) outgoing partons are generated at NLO (LO) in pQCD.
The production of $ZZ$ from the gluon-gluon initial state with a four-fermion loop or with an exchange of the Higgs boson, which has an order of $\alpha_{S}^{4}$ in QCD, is not included in the \textsc{Sherpa} simulation.
So a separate $gg$ induced $ZZ$ sample including the continuum background, the SM Higgs boson, and the interference contribution 
is modelled using \textsc{Sherpa} 2.2.2 with the NNPDF3.0NNLO PDF set,
and with an additional 1.7 k-factor~\cite{PhysRevD.92.094028} being applied.
The EW-$ZZjj$ production is simulated using \textsc{Sherpa} 2.2.2 with the NNPDF3.0NNLO PDF, and the $ZZZ \rightarrow \llll qq$ process is also taken into account in this sample.

The \Zjet events are generated using \textsc{Sherpa} 2.2.2 with the NNPDF3.0NNLO PDF,
in which the ME is calculated for up to two partons with next-to-leading-order (NLO) accuracy in pQCD and up to four partons with LO accuracy.
The \Zjet events are normalized using the next-to-next-to-leading-order (NNLO) cross section.
The triboson processes with full leptonic decays and at least four prompt charged leptons are generated using \textsc{Sherpa} 2.1.1.
For top-quark pair (\ttbar) production and the single top-quark productions in $t$-channel, $s$-channel and $Wt$-channel, the \textsc{Powheg-Box}~v2 is used with the CT10 PDF.
The productions of \ttbar~in association with $Z$ boson(s) ($ttZ$) is modelled with \MGMCatNLO.

\subsection{Signal MC simulations}

One model considered in this analysis is heavy spin-0 resonance under Narrow Width Approximation (NWA) simulated using \textsc{Powheg-Box}~v2 MC event generator with the CT10 PDF.
The gluon-gluon fusion (ggF) production mode and vector-boson fusion (VBF) production mode are calculated separately with matrix elements up to NLO in QCD.
The \textsc{Powheg-Box} is interfaced to \textsc{Pythia8} for parton showering, and for decaying the Higgs boson into the $H \rightarrow ZZ \rightarrow \llll$ final states.
Events under NWA are generated at mass points between 200~\gev~to 2000~\gev~using the step of 100 (200)~\gev~up to (above) 1~\tev~in both ggF and VBF productions.

In addition, heavy Higgs boson events under Large Width Approximation (LWA) with widths of 1\%, 5\%, 10\% and 15\% of the boson mass are generated using \MGMCatNLO~2.3.2 interfaced to \textsc{Pythia8}.
Only ggF production is considered.
Mass points between 400~\gev~to 2000~\gev~are simulated with 100 (200)~\gev step up to (above) 1~\tev.
To describe jet multiplicity, \MGMCatNLO~is used to simulated process of $pp\to H + \geq2\text{jets}$ at NLO in QCD with the FxFx merging scheme~\cite{Frederix2012}.

Spin-2 Kaluza–Klein (KK) gravitons (\Graviton) from the Bulk Randall–Sundrum model~\cite{graviton} are also studies in this analysis.
Events are generated by \MGMCatNLO at LO in QCD, which is then interfaced to \textsc{Pythia8}.
The \Graviton-gluon coupling \kOverMpl, where $k$ is the curvature scale of the extra dimension and \Mpl~is the reduced Planck mass, is set to 1.
And the mass of the \Graviton~is the only free parameter in this simplified model.
