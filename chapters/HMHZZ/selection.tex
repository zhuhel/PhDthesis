\section{Objects and Event selection}
\label{sec:hmhzz_selection}

\subsection{Objects selection}

Similar as described in section~\ref{sec:vbszz_selection}, the selection of this analysis relies on the definition of multiple objects: \textit{electrons}, \textit{Muons}, and \textit{jets}.
Details of definitions for each objects are described as below:

\textbf{Electron:}
The reconstruction of electrons is described in section~\ref{sec:electron}.
In this analysis, the electron candidates satisfying \textit{Loose} working point (WP) are selected,
with a selection efficiency ranging from 90\% for transverse momentum \pt = 20~\gev to 96\% for \pt > 60~\gev.
Also, electrons are required to have $p_{T} > 7 \gev$ and $|\eta| < 2.47$.

\textbf{Muon:}
To increase the acceptance range in reco-level for \lllljj channel, all four types of muons
(CB, ST, CT, ME muons, described in section~\ref{sec:muon}) are used.
But at most one CT, ST or ME muon is allowed in one \llll quadruplet.
The Muon candidates are required to pass $p_{T} > 7 \gev$ and $|\eta| < 2.7$,
and satisfy the \textit{Loose} identification criterion with an efficiency of 98.5\%.

\textbf{Jets:}
Jets are clustered using the anti-kt algorithm with radius parameter R = 0.4 implemented in the package, with the inputs of particle flow (PFlow) objects
