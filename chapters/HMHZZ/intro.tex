\section{Introduction}

A new particle was discovered by the ATLAS and CMS Collaborations at the LHC~\cite{20121, 201230} in 2012.
Both experiments have confirmed that the properties including spin, couplings and parity of this new particle are consistent with 
Higgs boson predicted in the Standard Model (SM), which is an important milestone in understanding of the mechanism of EWSB.
Nevertheless, the possibility that this newly discovered particle is just a part of the extended Higgs sector
as predicted by various extensions in the SM cannot be ruled out.
Many models predicted the existence of new heavy resonances decaying into dibosons, 
such as a heavy spin-0 neutral Higgs boson~\cite{PhysRevD.36.3463}
and the two-Higgs-doublet models (2HDM)~\cite{BRANCO20121}, 
as well as the spin-2 Kaluza–Klein (KK) excitations of the graviton ($G_{KK}$)~\cite{DAVOUDIASL200043}.

Though with smaller branching ratio compared to semileptonic or fully hadronic decay channels, the \llll final state has its unique sensitivity in mass range smaller than 1~\tev~region 
due to its good mass resolution and relative smaller experimental and theoretical systematics.
This section presents the search for heavy resonance decaying into a pair of $Z$ bosons to the \llll final state, in which $\ell$ denotes to either an electron or a muon~\cite{Aaboud:2017rel, Aad:2020fpj}. 
Several signal hypothesises are considered.
The first hypothesis is a heavy Higgs boson (spin-0 resonance) under the narrow-width approximation (NWA).
Then as several theoretical models prefer non-negligible natural widths, the models under large-width approximation (LWA), 
assuming widths of 1\%, 5\%, 10\% and 15\% of the resonance mass, are also studied.
In addition, the graviton excitations (spin-2 resonance) under the Randall–Sundrum model are also searched.
It is assumed that the heavy resonance is produced predominantly via the gluon-gluon Fusion (ggF) and the Vector Boson Fusion (VBF) productions, 
but with the unknown ratio of two production rates.
So the results are separated for ggF and VBF production modes.
To gain more sensitivity, the \llll events are classified into  ggF- and VBF-enriched categories.
Moreover, for the NWA model, the categorizations are studied under both cut-based and multivariate (MVA) -based methods, the details of categorization are shown in following sections.

The search uses the four-lepton invariant mass in the range of 200~\gev~to~2000~\gev~for signal hypothesis of spin-0 resonance under the NWA model,
and from 400~\gev~to 2000~\gev~for the one under the LWA models.
And the spin-2 graviton signals are searched in the mass range from 600~\gev~ to 2000~\gev.
The data collected by ATLAS detector at the LHC from 2015 to 2018 at the centre-of-mass energy of 13~\tev~is used.
In case of no excess, upper limits on the production rate of different signal hypotheses are computed from statistical fits to $m_{4l}$ distribution.

