% !TeX root = ../main.tex

\begin{abstract}
论文介绍了本人在粒子物理领域基于大型强子对撞机(LHC)上ATLAS 实验做的研究工作。 
大型强子对撞机是当今世界上最大的、能量最高的对撞机,是建立在理论和实验之间的重要桥梁。
而ATLAS实验是LHC上的一个通用例子探测器实验,同时也是体积最大的探测器。
基于ATLAS实验在LHC上收集到的亮度为139~\ifb~能量为13~\tev~的质子-质子对撞数据,本文重点介绍了两个$Z$玻色子衰变到四轻子末态过程的一系列研究。
包括,标准模型(SM)下$ZZ$到四轻子过程截面的测量、矢量玻色子散射(VBS)过程在$ZZ$到四轻子末态的观测,和寻找重共振态衰变到$ZZ$到四轻子末态的过程。

$ZZ$到四轻子过程截面的测量结果为$\sigma_{ZZjj}^{tot} = 1.27 \pm 0.12 (\mathrm{stat}) \pm 0.02 (\mathrm{theo}) \pm 0.07 (\mathrm{exp}) \pm 0.01 (\mathrm{bkg}) \pm 0.03 (\mathrm{lumi})$,
总体相对误差为11\%。
在误差范围内,该结果和标准模型预言值$1.14 \pm 0.04 (\mathrm{stat}) \pm 0.20 (\mathrm{theo})$相吻合。
同时,在两个$Z$玻色子伴随着两个喷注(jets)末态的电弱相互作用过程的寻找中, 我们观测到偏离本底假说超过5倍标准差(5.5 $\sigma$) 的明显偏差。
在此基础上,本文也介绍了对于下一代高亮度大型强子对撞机(HL-LHC)在两个$Z$玻色子伴随着两个喷注(jets)末态的电弱相互作用过程的模拟预言。

另一方面,本文介绍了在一对$Z$玻色子衰变至四轻子末态过程中寻找重共振态的实验。根据不同的信号模型,寻找的粒子质量区间设置在200~\gev~ 到 2000~\gev~之间。
基于该测量结果,没有证据可以证明重共振态的存在。因此,研究给出了基于不同信号模型的截面上限,包括在不同衰变宽度假说下自旋为0的共振态,以及基于Randall–Sundrum模型的自旋为2的引力子(graviton)。
在该分析中,我们认为,信号主要可通过gluon-gluon Fusion (ggF) 和Vector Boson Fusion (VBF) 过程产生。
在自旋为0的窄衰变宽度模型下,我们对ggF 和VBF两个过程都进行了研究。而对于大宽度模型,由于在质量很高的区间分辨率很差以及VBF过程的统计量太小等客观原因,只对ggF过程进行了研究。
对于自旋为2的模型,实验给出了Randall–Sundrum模型的引力子(graviton)的理论质量下限,为1500~\gev。
\end{abstract}

\begin{enabstract}
This dissertation presents my research in the field of Particle Physics with the ATLAS experiment at the Large Hadron Collider (LHC). 
The LHC is the world's largest and most powerful collider, and it was built as a bridge between the theories and the experiment.
The ATLAS experiment is a general-purpose particle detector experiment with the largest volume at the LHC.
This dissertation focus on the studies with two $Z$ bosons production decaying into \llll final state, where $\mathrm{\ell}$ stands for electron or muon, using 139~\ifb of 13~\tev~proton-proton (pp) collision data collected by ATLAS experiment at the LHC.
The $ZZ$ production in \llll channel provides a most clean and sensitive tool to test the Standard Model (SM) at the energy frontier and to study the \textit{Higgs} physics.
Studies including the measurement on SM $ZZjj$ production cross section, the observation of Vector Boson Scattering (VBS) process as well as the searches of heavy resonances in $ZZ$ production decaying into \llll final state
are reported in this dissertation.

The fiducial cross section for SM $ZZjj$ production is measured to be 
$\sigma_{ZZjj}^{tot} = 1.27 \pm 0.12 (\mathrm{stat}) \pm 0.02 (\mathrm{theo}) \pm 0.07 (\mathrm{exp}) \pm 0.01 (\mathrm{bkg}) \pm 0.03 (\mathrm{lumi})$
with a total relative uncertainty of 11\% for the \llll final state, and found to be compatible with the SM prediction of $1.14 \pm 0.04 (\mathrm{stat}) \pm 0.20 (\mathrm{theo})$.
The electroweak production of two jets in association with a Z-boson pair (EW-$ZZjj$) is observed with
a significant deviation from the background-only hypothesis corresponding to a statistical significance of 5.5 $\sigma$.
Following with the observation, the prospect study for the EW-$ZZjj$ production at the High luminosity LHC (HL-LHC) using 3000~\ifb simulated pp collision data at a centre-of-mass energy of 14~\tev~is presented,
with a expected significance of around 7 $\sigma$.

A search for heavy resonances decaying into a pair of $Z$ bosons to \llll final state is also conducted in this dissertation.
Different mass ranges for the hypothetical resonances are considered, depending on the signal models and spanning between 200~\gev~and 2000~\gev.
Data is found to agree with a background-only hypothesis, thus, the results are interpreted as upper limits on production cross section for sevaral different models, 
including heavy Higgs like (spin-0) narrow-width approximation (NWA) and large-width approximation (LWA), as well as the Randall–Sundrum model with a graviton excitation spin-2 resonance (RSG).
The signal is assumed to generate dominatly via gluon-gluon Fusion (ggF) production mode and Vector Boson Fusion (VBF) production mode.
Both ggF and VBF channels are studied in NWA, while for LWA, only ggF channel is studied due to worse resolution in higher mass region and the lack of statistic for VBF process.
In addition, mass of RS Graviton is constrained, m($G_{KK}$) < 1500~\gev~ is excluded at 95\% CL by $ZZ \rightarrow \llll$ analysis.

\end{enabstract}
