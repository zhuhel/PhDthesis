% !TeX root = ../main.tex

\chapter{Introduction}

The goal of particle physics is to understand how our universe works at its most fundamental level. It can be accomplished by pursuing the mysteries of the basic construction of matter and energy, probing the interactions between elementary particles, and exploring the basic nature of space and time itself. 

\textbf{Elementary particles}

From around the 6th century BC, ancient Greek philosophers Leucippus, Democritus, and Epicurus brought up a philosophical idea that everything is composed of "uncuttable" elementary particles. In the 19th century, John Dalton, through his work on stoichiometry, concluded that each element of nature was composed of a single, unique type of particle. The particle was named as "atom" after the Greek word atomos, with the meaning of "indivisible". However this Dalton's atom theory was strongly challenged later. Near the end of 19th century, physicists discovered that Dalton's atoms are not, in fact, the fundamental particles of nature, but conglomerates of even smaller particles. Electron was discovered by J. J. Thomson in 1897, and then its charge was carefully measured by Robert Andrews Millikan and Harvey Fletcher in their "oil drop experiment" of 1909. In early 20th-century, Rutherford's "gold foil experiment" showed that the atom is mainly empty space, with almost all its mass concentrated in a tiny positive charge atomic nucleus. Then the discoveries of anti-particles (the positron in 1932) and other particles (e.g. the muon in 1936) shows that more discoveries could be expected in future experiments.

Starting from 1950s, more accelerator facilities were put into service. Throughout the 1950s and 1960s, a bewildering variety of particles were found in collisions of particles from increasingly high-energy beams. It was referred to informally as the "particle zoo".
In 1964, the quark model was independently proposed by physicists Murray Gell-Mann and George Zweig, and experimentally confirmed of their existence in mid-1970s. In 1970s, the establishment of quantum chromodynamics  (QCD) postulated the fundamental strong interaction, experienced by quarks and mediated by gluons.

The well-known Standard model (SM) was developed in stages throughout the latter half of the 20th century. Since then, confirmation of the top quark (1995), the tau neutrino (2000), and the Higgs boson (2012) have added further credence to the Standard Model.
Now, the quarks, leptons and gauge bosons are the elementary constituents in a framework of Standard Model of particle physics, which theoretically describes three of the four known fundamental forces (the electromagnetic, weak, and strong interactions, and not including the gravitational force) in the universe, as well as classifies all known elementary particles.

\textbf{Higgs mechanics and electroweak symmetry breaking}

In 1961, Sheldon Glashow brought forward a unified electroweak theory to combine the electromagnetic and weak interactions. In the standard model, at energy high enough that electroweak symmetry is unbroken, all elementary particles are massless. But measurements show the fact that the W and Z bosons actually have masses. Later on, the Higgs mechanics resolves this conundrum. The simplest description of the mechanism adds a Higgs field that permeates all space to the Standard Model. Below some extremely high energy, the field causes spontaneous symmetry breaking during interactions. All massive particles in the Standard Model, including the W and Z bosons, interact with Higgs boson to acquire their mass.

Over the past few decades, with the combination of electroweak theory, Higgs mechanics and strong interactions has been widely accepted. But the Higgs boson, which is essential to explain the mechanics of the property "mass" for gauge bosons and fermions, had been the final missing piece in the Standard Model of particle physics for the time being. The mass of Higgs boson was not be specifically predicted, and it has been searched in several large experiments (eg. LEP at CERN, Tevatron at Fermilab, and LHC at CERN). In 2012, the discovery of Higgs boson was finally announced by the ATLAS and CMS collaborations at the Large Hadron Collider (LHC) with its mass round 125 GeV. Peter Higgs and Francois Englert were award the 2013's Nobel Prize in Physics for their theoretical discovery of a mechanism that contributes to our understanding of the origin of mass of subatomic particles.

\textbf{Contents of this thesis}

This thesis is organized as follows. Section 2 briefly introduces the Standard Model of particle physics, the Higgs mechanism related to the thesis and the LHC phenomenology. Section 3 gives an overview of the LHC and the ATLAS detector. The detector simulation and the reconstruction of physics objects are described in section 4. Section 5 focuses on the Standard model ZZ production cross section measurement in $ZZ \rightarrow 4l$ channel, and the observation of its electroweak component as well as its further prospects in High luminosity LHC (HL-LHC). Section 6 present the search of possible heavy Higgs in $H \rightarrow ZZ \rightarrow 4l$ channel. In the end, section 7 gives the summary and outlook for future physics in LHC.


