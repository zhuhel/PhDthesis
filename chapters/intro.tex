% !TeX root = ../main.tex

\chapter{Introduction}

The research of particle physics is aiming to understand how our universe works at its fundamental level. It can be accomplished by pursuing the mysteries of the basic construction of matter and energy, probing the interactions between elementary particles, and studying the nature of time and space. 

\textbf{Elementary particles}

From around the 6th century BC, ancient Greek philosophers Leucippus, Democritus, and Epicurus brought up a philosophical idea that everything is composed of ``uncuttable" elementary particles. 
In the 19th century, John Dalton, through his work on stoichiometry, concluded that each element of nature was composed of a single, unique type of particle. 
The particle was named as ``atom" after the Greek word atomos, with the meaning of ``indivisible". 
However this Dalton's atom theory was strongly challenged later. Near the end of 19th century, physicists discovered that Dalton's atoms are not, in fact, the fundamental particles of nature, 
but conglomerates of even smaller particles. 
Electron was discovered by J. J. Thomson in 1897~\cite{Thomson1897}, and then its charge was then carefully measured by Robert Andrews Millikan and Harvey Fletcher in their ``oil drop experiment" of 1909~\cite{PhysRev.2.109}. 
In early 20th-century, Rutherford's ``gold foil experiment" showed that the most mass of atom is concentrated in a small positive charge nucleus~\cite{Rutherford1911}. 
Then the discoveries of anti-particles (the positron in 1932) and other particles (e.g. the muon in 1936) indicate that more discoveries could be expected in future experiments.

Starting from 1950s, more accelerator facilities were put into service. 
During the 1950s and 1960s, various particles were found in particle collisions from increasingly high-energy beams, informally referred to as the ``particle zoo".
In 1964, the quark model was independently proposed by physicists Murray Gell-Mann and George Zweig, and experimentally confirmed of their existence in mid-1970s. 
In 1970s, the establishment of quantum chromodynamics  (QCD) postulated the fundamental strong interaction, experienced by quarks and mediated by gluons.

The well-known Standard model (SM) was developed during the latter half of the 20th century. 
At that time, confirmation of the top quark (1995), the tau neutrino (2000), and the Higgs boson (2012) have added further credence to the SM.
Now, the quarks, leptons and gauge bosons are the elementary particles in the framework of Standard Model of particle physics, 
that theoretically describes three of the four known fundamental forces (the electromagnetic, weak, and strong interactions, and not including the gravitational force) in the universe, 
as well as classifies all known elementary particles.

\textbf{Higgs mechanics and electroweak symmetry breaking}

In 1961, Sheldon Glashow, Steven Weinberg and Abdus Salam together brought forward a unified electroweak theory to combine the electromagnetic and weak interactions. 
In the standard model, under the condition that the energy is high enough but electroweak symmetry is unbroken, all elementary particles are massless. 
But measurements show the fact that the W and Z bosons actually have masses. 
Later on, the Higgs mechanics resolves this conundrum. 
The description of the mechanism adds a Higgs field in all space of the Standard Model, 
where the field causes spontaneous symmetry breaking during interactions,
and all massive particles in the Standard Model, including the W and Z bosons, interact with Higgs boson to acquire their mass.

Over the past few decades, with the combination of electroweak theory, the Higgs mechanics has been widely accepted. 
But the Higgs boson, the essential part to explain this mechanics of the property ``mass" for gauge bosons and fermions, had been the final missing piece in the Standard Model of particle physics at that time. 
The mass of Higgs boson was not specifically predicted by the SM, and it has been searched in several large experiments (eg. LEP at CERN, Tevatron at Fermilab, and LHC at CERN) with different energy. 
In 2012, the discovery of Higgs boson was finally announced by the ATLAS and CMS collaborations at the Large Hadron Collider (LHC) with its mass round 125~\gev. 
Peter Higgs and Francois Englert were award the Nobel Prize in Physics in the year of 2013 for their theoretical discovery of 
this mechanism that contributes to our understanding of the origin of mass.

\textbf{Contents of this thesis}

This dissertation is organized as follows. 
Section 2 briefly introduces the Standard Model of particle physics, the Higgs mechanism related to the dissertation and the LHC phenomenology. 
Section 3 gives an overview of the LHC and the ATLAS detector. 
The detector simulation and the reconstruction of physics objects are described in section 4. 
And then section 5 focuses on the Standard model ZZ production cross section measurement in $ZZ \rightarrow \llll$ channel, where $\mathrm{\ell}$ stands for electron or muon, 
and the observation of its electroweak component as well as its further prospects in High luminosity LHC (HL-LHC). 
Section 6 present the search of possible heavy resonances in $ZZ \rightarrow \llll$ channel. 
In the end, section 7 gives the summary and outlook for future physics in LHC.
