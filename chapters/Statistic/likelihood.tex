\section{The likelihood function}

The likelihood function is defined as the product of a set of the probability density functions (pdfs) of variables $x$, that used to evaluate the probability of the observed dataset:
\begin{equation}\label{eq:likelihoodf}
    \mathcal{L} (x_{1}, ..., x_{N}; \theta_{1}, ..., \theta_{M}) = \prod_{i}^{N} f(x_{i}; \theta_{1}, ..., \theta_{M})
\end{equation}
where $\theta_{1}$, ..., $\theta_{M}$ are the nuisance parameters that can be written as $\pmb{\theta}$,
and $x_{1}$, ..., $x_{N}$ denote the observables of dataset. 
Usually one measures the variable $x$ by constructing a histogram $\pmb{n} = (n_{1}, ..., n_{N})$~\cite{Cowan:2010js}.
The expectation value of the ith bin $n_{i}$ can be written as:
\begin{equation}\label{eq:discri}
    E[n_{i}] = \mu s_{i} + b_{i}
\end{equation}
where $\mu$ is the signal strength, $s_{i}$ and $b_{i}$ are the number of signal and background events in that bin.
In addition to the histogram $\pmb{n}$, in some cases, one would like to use subsidiary measurements to help further constrain the nuisance parameters.
For instance, due to the lack of background simulation or the mismodelling issue of one MC sample, one can choose a control region and construct another histogram $\pmb{m} = (m_{1}, ..., m_{M})$ to constrain the contribution of one certain background in data.
For this measurement, the expectation value of the ith bin $m_{i}$ can be written as:
\begin{equation}\label{eq:cr}
    E[m_i] = u_i(\pmb{\theta})
\end{equation}

In most particle experiments, the number of these events observed in one bin follows the Poisson distribution,
by combining the equation~\ref{eq:discri} and ~\ref{eq:cr}, one can get the likelihood function for all bins as:
\begin{equation}
    \mathcal{L} (\mu, \pmb{\theta}) = \prod_{i=1}^{N} \frac{(\mu s_{i} + b_{i})^{n_i}}{n_i !} e^{-(\mu s_{i} + b_{i})}
    \prod_{k=1}^{M} \frac{u_k^{m_k}}{m_k !} e^{-u_k}
\end{equation}

Then to test the hypothesized value of $\mu$, the profile likelihood ratio is defined as:
\begin{equation} \label{eq:lambda}
    \lambda (\mu) = \frac{\mathcal{L}(\mu, \pmb{\hat{\hat{\theta}}})}{\mathcal{L}(\hat{\mu}, \pmb{\hat{\theta}})}
\end{equation}
where numerator denotes to a local maximum-likelihood for a specific $\mu$, $\pmb{\hat{\hat{\theta}}}$ is the value of $\pmb{\theta}$ that maximizes the numerator.
And the denominator is the global maximum-likelihood with the $\hat{\mu}$ and $\pmb{\hat{\theta}}$ as their best fit value.
