\section{The CLs upper limit}
\label{sec:CLs}

For a signal hypothesized value $\mu$, one can compute the probability that this hypothesis (called S+B hypothesis) gives a \textbf{greater} test statistic value than the observed one $q_{obs}$ as:
\begin{equation}
    p_{s+b} = \int_{q_{obs}}^{\infty} f(q_{\mu}|\mu) d q_{\mu}
\end{equation}
In the meantime, the probability that the background-only hypothesis gives a \textbf{smaller} test statistic than observed data can also be calculated as:
\begin{equation}
    1 - p_{b} = \int_{-\infty}^{q_{obs}} f(q_{\mu}|0) d q_{\mu}
\end{equation}
Then we define the CLs~\cite{Read_2002} of a hypothesized value $\mu$ as:
\begin{equation}
    CLs = \frac{p_{s+b}}{1-p_{b}}
\end{equation}
For purpose of excluding a signal hypothesis, a threshold CLs of 0.05 is often used.
For this reason, usually under the circumstance that no significant derivation between data and background-only hypothesis is found,
one would like to find the value of hypothesized signal strength $\mu$ by requiring its $CLs = 0.05$ (called 95\% CLs upper limit) for exclusion. 

The sensitivity of an experiment to exclude a new signal process is quantified by \textit{median upper limit}~\cite{Bellos:2725027},
which is obtained using ``Asimov dataset".
The Asimov dataset is defined such that when one uses it to evaluate the estimators for all parameters, one obtains the true parameter values.
Moreover, it is useful to use Asimov dataset to compute how much the sensitivity is expected to vary, given the expected fluctuations in the data.
The $\hat{\mu}$ is assumed to follow a Gaussian distribution with a mean value of $\mu '$ and the standard deviation of $\sigma$.
First of all, the test statistic from profile likelihood ratio can be approximated as~\cite{Cowan:2010js}:
\begin{equation}
	-2 ln \lambda(\mu) = \frac{(\mu - \hat{\mu})^2}{\sigma^2} + \mathcal{O}(1/\sqrt{N})
\end{equation}
Given that the Asimov dataset corresponding to a signal strength $\mu'$, one finds:
\begin{equation}
	-2 ln \lambda_{A}(\mu) \approx \frac{(\mu - \mu')^2}{\sigma^2} = q_{\mu,A}
\end{equation}
where $q_{\mu,A} = -2ln\lambda_{A}(\mu)$ is the observed test statistic of Asimov dataset.
Then the standard derivation can be computed as:
\begin{equation}
	\sigma_A^2 = \frac{(\mu - \hat{\mu})^2}{q_{\mu,A}}
\end{equation}
In a special situation where one wants to find the median exclusion significance for the hypothesis $\mu$ assuming that there is no signal ($\mu' = 0$),
one gets:
\begin{equation}
        \sigma_A^2 = \frac{\hat{\mu}^2}{q_{0,A}}
\end{equation}
