\subsection{Physics requirement}

As mentioned previously, ATLAS is one of two general-purpose particle detector experiment at LHC.
It's designed to take advange of the unprecedented energy at LHC.
The Higgs boson was discovered as one of its benchmark, and lots of precise tests and measurements of SM is on going.
In the meantime, ATLAS is also designed to abserve the phenomena that involve highly massive particles, such as heavy beyond standard model (BSM) gauge bosons $Z'$ and $W'$.
It can also explore the possibility of extra dimensions proposed by several models in TeV region.
To fulfil many diverse physics goals, a set of general requirements are needed:
\begin{itemize}
	\item The speed-fast and radiation-hard electronics are required due to the experimental conditions at LHC. 
	\item High detector granularity is needed to reduce the overlapping events and handle the particle fluxes.
	\item Large acceptance in pseudorapidity and azimuthal angle coverage is needed.
	\item For inner detector, good charged-paricle momentum resolution and reconstruction efficiency are crucial. And the vertex detectors close to the interaction region are required to be able to observe secondary vertices for offline tagging of $\tau$-lepton and $b$-jets.
	\item Good electromagnetic (EM) calorimetry for electron and photon, as well as full-coverage hadronic calorimetry for accurate jet and missing transverse energy measurements, are importantly required, since these measurements form the basis of many studies.
	\item Good muon spectrometer is also required for muon identification and momentum resolution measurement over a wide range of momenta.
	\item Highly efficient but with sufficient background rejection triggers are also needed and extramely important for objects with low transverse-momentum. 
\end{itemize}

%Table~\ref{tab:detector_goal} summarizes the major performance for different parts of ATLAS detector.
More detailed descriptions of each sub-system will be given in the following subsections.
