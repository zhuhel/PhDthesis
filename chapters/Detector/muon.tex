\subsection{Muon spectrometer}

Muon spectrometer~\cite{CERN-LHCC-97-022} is the outermost part of the ATLAS detector with an extremely large tracking system.
It measures a large range of muon momentum, and the accuracy is about 3\% at 100 GeV and 10\% at 1 TeV.
The muon spectrometer comprises three main parts: a magnetic field produced by three toroidal magnets;
a set of chambers measuring the tracks of muons with high spatial precision; and triggering chambers with accurate time-resolution. 
Figure~\ref{fig:muon_dec} shows the schematic of ATLAS muon spectrometer that consists of four types of muon chambers 
(\textit{MDT, CSC, RPC, TGC}) as well as the magnet systems (barrel and end-cap toroid).
\begin{figure}[!htb]
  \centering
  \includegraphics[width=0.8\textwidth]{figures/Detector/muon_all.png}
  \caption{Cut-away view of the muon spectrometer in ATLAS~\cite{Sliwa:2013oua}.}
  \label{fig:muon_dec}
\end{figure}

More details of four chambers are given as below:
\begin{itemize}
	\item \textbf{Monitored Drift Tubes (MDT)}. MDTs offer precise measurement of momentum with the $|\eta|$ range up to 2.7.
        %except in the innermost end-cap layer where the coverage is limited to $|\eta| < 2.0$. 
        The chambers include three to eight layers of drift tubes, with a diameter of 29.970 mm, operated with Ar/CO2 gas (93/7) at 3 bar. The average resolution can reach 80 $\mu$m per tube and 30 $\mu$m per chamber.
	\item \textbf{Cathode strip chambers (CSC)}. CSCs are used in the forward region of $2 < |\eta| < 2.7$ in the innermost tracking layers, because of their good time resolution and high rate capability. 
        They are multi-wire proportional chambers (MWPC), in which the cathode planes are segmented into strips in orthogonal directions, allowing both coordinates to be measured based on the induced-charge distribution. 
        The resolution in the bending plane is about 40 $\mu$m and 5 mm in the transverse plane.
	\item \textbf{Resistive plate chambers (RPC)}. The RPCs serve as fast triggers in the barrel region of $|\eta| < 1.05$ due to its high rate capability as well as its good time and spatial resolution. 
        They are gaseous parallel electrode-plate detector without any wires. 
        There are three concentric cylindrical layers around the beam axis working as three trigger stations, while each of them is composed of two independent layers to measure the transverse coordinates of $\eta$ and $\phi$.
	\item \textbf{Thin gap chambers (TGC)}. TGCs are used as trigger system for the end-cap region of $1.05 < |\eta| < 2.4$, and works based on the same principle as multi-wire proportional chambers. 
        In addition to the measurement of MDT in bending direction, they also offer the second azimuthal coordinate as supplement.
\end{itemize}
