\subsection{Detector overview}

ATLAS is the world's largest volume particle detector.
It is a cylinder with 46 meters long, 25 meters in diameter, and sits in a cavern 100 meters below ground.
The detector contains about 3000 km of cables and it weights 7000 tonnes.

The coordinate system and nomenclature used to describe the ATLAS detector \cite{Collaboration_2008} is depicted in figure~\ref{fig:coordinate}. We define the nominal interaction point as the origin of the coordinate system, the beam direction as the \textit{z}-axis and the \textit{x-y} plane is transverse to the beam direction.
The positive \textit{x}-axis is given as the direction from interaction point to the centre of the LHC ring, 
while the positive \textit{y}-axis points upward.
\begin{figure}[!htb]
  \centering
  \includegraphics[width=0.9\textwidth]{figures/Detector/Coordinate_system_atlas.png}
  \caption{Coordinate system used by the ATLAS experiment at the LHC \cite{Perez:phdthesis}.}
  \label{fig:coordinate}
\end{figure}
There are two sides of detector A and C, in which A (C) -side is in the positive (negative) \textit{z} direction.
The polar angle $\theta$ is measured from the beam axis, while the azimuthal angle $\phi$ is obtained around the beam axis.
In physics analysis, we usually use the pseudorapidity $\eta$ designed as:
\begin{equation}
    \eta = - ln \left[ tan\left( \frac{\theta}{2}\right) \right]
\end{equation}
instead of $\theta$ angle. 
And for massive objects (eg. jets), the rapidity is used:
\begin{equation}
    y = \frac{1}{2} ln \left[ \frac{E+p_{z}}{E-p_{z}} \right]
\end{equation}

In addition, the transverse momentum $p_{T}$, transverse energy $E_{T}$ and the missing transverse energy $E_{T}^{miss}$ are defined in \textit{x-y} plane.
The $\Delta R$, a commonly used distance measurement, is defined in the pseudorapidity-azimuthal angle space as:
\begin{equation}
    \Delta R = \sqrt{ \Delta\eta^{2} + \Delta\phi^{2}}.
\end{equation}

The overall ATLAS layout is shown in figure~\ref{fig:atlas_layout}, which is forward-backward symmetric with respect to the interaction point.
\begin{figure}[!htb]
  \centering
  \includegraphics[width=1.0\textwidth]{figures/Detector/atlas_layout.jpg}
  \caption{Layout view of the ATLAS detector \cite{Pequenao:1095924}.}
  \label{fig:atlas_layout}
\end{figure}
The magnet configuration has a thin superconducting-solenoid surrounding the inner-detector, 
and three large superconducting-toroids (one barrel and two end-caps) around the calorimeters.

\textbf{The inner detector}, which is the innermost part of ATLAS, is surrounded by a 2 T solenoidal magnetic field.
It's used for pattern recognition, momentum and vertex measurements and electron identification, with the combination of tracking system in the region of $\eta$ up to 2.5.

\textbf{The calorimeter} is outside the solenoid, for electromagnetic and hadronic energy measurements.
The high granularity liquid-argon (LAr) electromagnetic sampling calorimeters is used to measure energy and position with range up to $|\eta| < 3.2$ for electrons and photons.
For hadron, a scintillator-tile calorimeter is used in the range of $|\eta| < 1.7$, and the liquid-argon hadronic endcap calorimeters (HEC) is used in end-cap region.
And then the LAr forward calorimeters provide both electromagnetic and hadronic energy measurements with the coverage in forward region up to $|\eta| = 4.9$.

\textbf{The muon spectrometer} is the outermost layer.
It's a air-core toroid system, with a long barrel and two inserted end-cap magnets that provides strong bending power in a large volume within a light and open structure.
A set of chambers measuring the tracks of muons with high spatial precision and accurate time-resolution are used.
Multiple-scattering effects are minor, and excellent muon momentum resolution can be achieved.
