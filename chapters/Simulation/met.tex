\subsection{Missing transverse energy}

Many interesting physics processes are with the inveloment of neutrinos.
Since they do not interact with any materials in the detector, neutrinos cannot be detected directly;
but instead, they can result in imbalance in the plane transverse to the beam axis, in which momentum conservation is assumed.
It is known as the missing transverse momentum denoted as $E_{T}^{miss}$,
which is obtained from the negative vector sum of the momenta of all particles detected in a proton-proton collision event.

The $E_{T}^{miss}$ is measured using selected, reconstructed and calibrated hard objects in an event.
Its x- and y- conponents can be calculated as follow\cite{ATL-PHYS-PUB-2015-027}:
\begin{equation} \label{eq:met_xy}
	E_{x(y)}^{miss} = E_{x(y)}^{miss, e} + E_{x(y)}^{miss, \gamma} + E_{x(y)}^{miss, \tau} + E_{x(y)}^{miss, jets} + E_{x(y)}^{miss, \mu} + E_{x(y)}^{miss, soft}
\end{equation}
where each object term is given by the negative vectorial sum of the momenta of the respective calibrated objects.
The calorimeter signals are associated with the reconstructed objects in the following order: electrons, photons, hadronically decaying taus, jets, muons.
The soft term is reconstructed from detected objects not overlap with any object passing the above selection cuts.

Based on $E_{x(y)}^{miss}$, the magnitude of $E_{T}^{miss}$ and the azimuthal angle $\phi^{miss}$ are computed:
\begin{equation}
\begin{split}
	E_{T}^{miss} &= \sqrt{ \left(E_{x}^{miss}\right)^{2} + \left(E_{y}^{miss}\right)^{2} } \\
	\phi^{miss} &= arctan \left(E_{y}^{miss}/E_{x}^{miss}\right)
\end{split}
\end{equation}

In equation~\ref{eq:met_xy}, each objects are required to pass certain reconstruction and calibrated criteria before taken as inputs.
