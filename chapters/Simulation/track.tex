\subsection{Track}
\label{sec:track}

The ATLAS detector is composed of two independent tracking systems: the Inner Detector (ID) close to the interaction point, and the Muon Spectrometer (MS) located in the outermost region.
The reconstructed charged-particle trajectories in the ID and MS are referred to as ID tracks and MS tracks respectively.
The challenge of ID reconstruction is that it needs to handle high track density that imposes a large number of combinatorial track candidates, 
while the MS reconstruction is however largely limited by the huge amount of inert material, the large background and the highly inhomogeneous magnetic field~\cite{Cornelissen:1020106}.
More details of these two types of track are given as below:

\textbf{Inner detector track}

Figure~\ref{fig:track_ID} sketches the ID system used for detecting charge-particle tracks.
The ID track reconstructions contains two sequences: \textit{inside-out} track reconstruction and \textit{outside-in} one.
\begin{figure}[!htb]
  \centering
  \includegraphics[width=0.7\textwidth]{figures/Simulation/track_ID.png}
  \caption{Schematic view of the ATLAS inner detector showing all the corresponding components.}
  \label{fig:track_ID}
\end{figure}

For inside-out tracking, it exploits the high granularity of the pixel and SCT detectors to discover prompt tracks originating from the interaction point.
In first step, the track seeds are formed by combining the information of space-points in the three pixel layers and the first SCT layer.
Then, these seeds are extended throughout the SCT to build track candidates.
After that, these candidates are fitted with some quality cuts applied to remove the outlier clusters, reject the fake tracks and resolve ambiguities in the cluster-to-track association.
The selected tracks are then further extended to TRT, and refitted with the full information from pixel, SCT and TRT detectors.

Another complementary approach, outside-in, searches for unused track segments start from TRT instead.
These segments are then extended into the SCT and pixel detectors to improve the tracking efficiency for secondary tracks from conversions or decays of long-lived particles.

\textbf{Muon spectrometer track}

The MS track reconstruction\cite{Aad:2016jkr} starts from searching hit patterns inside each muon chamber to form segments.
In each MDT chamber and nearby trigger chamber, a Hough transform\cite{ILLINGWORTH198887} is used to search the hits lie on a certain trajectory in the bending plane of the detector.
The MDT segments are reconstructed by performing a linear fit to the hits found in each layer.
The RPC or TGC hits can be built by measuring the coordinate orthogonal to the bending plane.
And the segments of CSC can be built using a separate combinatorial search in the $\eta$ and $\phi$ detector planes.

Then muon track candidates are built by fitting hits from segments in different layers together.
This task makes use of the algorithm by performing a segment-seeded combinatorial search, which starts by using the segments generated in the middle layers of the detector where more trigger hits are available as seeds.
The search is then extended to use the segments as seeds from the inner and outer layers.
The segments are selected based on criteria of hit multiplicity and fit quality, and are matched using their relative positions and angles.
To build a track, at least two matching segments are required, except in the barrel-endcap transition region where a single high-quality segment with $\eta$ and $\phi$ information can be used to build a track.
At beginning, the same segment can be used to build more than one track candidates.
Later on, an overlap removal algorithm is performed to select the best assignment to a single track, or decide whether allows the certain segment to be shared between two tracks.

The hits associated with each track candidate are then fitted using a global $\chi^{2}$ fit.
The algorithm accepts the track candidate if its fitting $\chi^{2}$ passes the selection criteria.
Hits contribute largely to $\chi^{2}$ are removed and the track fit is repeated.
In addition, the algorithm performs a hit recovery procedure that looks for additional hits consistent with the candidate trajectory, and the track candidate is refit if additional hits are found.
